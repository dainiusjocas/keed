%% 4. Rezultatai, išvados ir pasiūlymai. Išdėstomi pagrindiniai darbo rezultatai ir išvados,  praktikos darbo privalumai ir trūkumai, aprašomos įgytos žinios ir patirtis praktikos metu, duodamas universitete įgytų žinių atitikimo praktikos užduočiai atlikti įvertinimas, pateikiami argumentuoti pasiūlymai, kaip geriau organizuoti darbo ir valdymo procesus praktikos atlikimo vietoje ir mokymą Universitete (1-2 psl.).

\section{REZULTATAI, IŠVADOS IR PASIŪLYMAI}
\label{rezultatai_isvados_pasiulymai}

Profesinės praktikos metu susipažinau su matų atrinkimo daugiamačiuose duomenyse problematiką nagrinėjančia literatūra, suprogramavau pagrindinius matų atrinkimo algoritmus bei atlikau suprogramuotų algoritmų palyginamąją analizę. Įvykdęs profesinei praktikai keltus uždavinius, galiu tvirtinti, kad matų atrinkimas daugiamačiuose duomenyse yra sudėtinga problema, nes vieni matų atrinkimo metodai pvz. AW-SVM, greitai atrenka matus, kurie pagerina klasifikavimo tikslumą, tačiau atrinktas matų poaibis nėra stabilus; arba konsensuso grupėmis grįstas matų atrinkimas yra labai lėtas, tačiau jis atrenka stabilų matų poaibį, kuris pagerina klasifikavimo procesą. Todėl vienareikšmiškai teigti, kuris matų atrinkimo algoritmas yra absoliučiai geriausias, negalima -- matų atrinkimo algoritmą visada reikia pasirinkti atsižvelgiant į pačius tiriamus duomenis ir tiriamai problemai keliamus uždavinius.

Didžiausias praktikos darbo privalumas yra tai, kad visą darbo dieną galima skirti konkrečių uždavinių įgyvendinimui ir planuoti darbų atlikimo laiką. To pasiekti galima, nes atvažiavus į profesinės praktikos vietą ryte iki pat vakaro nereikia gaišti laiko kelionėms mieste. Be to, atliekant profesinę praktiką yra proga pasisemti patirties iš kolegų. 

Dirbdamas su VU MII mokslininkais įgijau daugiamačiais duomenimis apibūdinamų procesų, duomenų analizės žinių bei pagerinau programavimo įgūdžius. Daugiamačiais duomenimis apibūdinamų procesų žinių sėmiausi MII rengiamų seminarų metu. Duomenų analizės žinių sėmiausi iš mokslinės literatūros, bei konsultacijų su kolegomis. Profesinės praktikos metu reikėjo programuoti statistinei analizei skirta programavimo kalba \textit{R}. Dž

Mokymą universitete siūlyčiau gerinti peržiūrint studijų tvarkaraštį. Profesinė praktika yra naudinga studijų procesui, bet ji prasideda iškart po labai įtemptos sesijos, kuri prasideda dar prieš šv. Kalėdas, o prieš šv. Kalėdas yra semestro pabaiga, kurios metu reikia užbaigti semestro darbus, parašyti visus kontrolinius, bei pasirūpinti Kalėdinėmis dovanomis artimiesiems. Kitaip tariant, po poros sunkių mėnesių (gruodis, sausis), prasideda praktika, kurios pradžioje studentas tiesiog natūraliai ir pagrįstai norėtų pailsėti. O, savo ruožtu, praktikoje reikia skubėti spręsti praktikai išsikeltus uždavinius, nes laiko gali ir greičiausiai jiems išspręsti pritrūks. Žinoma, laiko pritrūks, jei profesinė praktika atliekama ne nuolatinėje darbovietėje. Atliekant praktiką nuolatinėje darbovietėje laiko nepritrūks, nes darbai dažniausiai yra planuojami ilgesniam nei 11 savaičių periodui. Tačiau darbas nuolatinėje darbovietėje studijų metų nėra tai ką turėtų daryti studentas - studentas turi studijuoti universitete.

Mano siūlomas studijų tvarkaraštis būtų toks: rudens semestras prasideda mėnesiu vėliau nei įprasta - spalio pirmą ar rugsėjo paskutinę savaitę; šventiniu laikotarpiu (šv. Kalėdos, Naujieji metai) studentams yra duodamos dviejų ar triejų savaičių atostogos; rudens semestro pabaiga perkeliama į sausio pabaigą; tada mėnuo sesijai; savaitė ar dvi atostogų; praktika ir bakalaurinis darbas arba pavasario semestras; atsiskaitymai už pavasario semestrą.