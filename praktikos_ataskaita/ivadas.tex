\addcontentsline{toc}{section}{{\c I}VADAS}
\section*{ĮVADAS}
% Įvadas. Išdėstomi praktikos vietos pasirinkimo motyvai, praktikos užduotis,
%  jos tikslas, spręstieji uždaviniai, pateikiama praktinės veiklos  planas 
% praktikos atlikimo eiga (2-3 psl.). 

Profesinei praktikai atlikti pasirinkau Vilniaus universiteto matematikos ir 
informatikos instituto (MII) sistemų analizės skyrių dėl keletos priežasčių. Visų pirma, norėjau pasinaudoti 
galimybe profesinės praktikos metu tęsti bakalauriniame darbe atliekamą tyrimą. 
Bakalauriniame darbe nagrinėjama biomedicininių daug atributų turinčių - 
daugiamačių duomenų suskirstymo į pageidaujamas kategorijas pagal vidinę duomenų
struktūrą - klasifikavimo problema.

Norint pradėti spręsti bakalauriniame darbe iškeltą problemą reikia ir tais 
daugiamačiais duomenimis 
apibūdinamų procesų dalykinės srities, ir duomenų analizės, ir programavimo 
žinių. Todėl antroji mano pasirinkimo profesinę praktiką atlikti MII priežastis yra ta, kad
dirbdamas MII, turėsiu galimybę konsultuotis su daugiamačių 
duomenų analizės problematiką tiriančiais mokslininkais. Jų žinių bagažas
labai palengvino ir pagreitino mano susipažinimą su nagrinėjama problematika.

%% Profesinei praktikai kelta užduotis buvo programuoti stabilius dimensijų 
%% atrinkimo algoritmus ir juos palyginti naudojant empirinius duomenis. 

Profesinės praktikos tikslas - ištirti bazinius ir multikriterinius dimensijų atrinkimo metodus.
% ir, jei pavyks,pasiūlyti naują dimensijų atrinkimo algoritmą.
Siekiant užsibrėžto tikslo profesinės praktikos metu buvo sprendžiami šie 
uždaviniai:
\begin{enumerate}
  \item Suprogramuoti bazinius dimensijų atrinkimo metodus: 
fišerio įvertį \textit{(angl. fisher score)}, 
atpalaidavimo koeficientą \textit{(angl. relief)},
asimetrinį priklausomybės koeficientą \textit{(angl. asymetric
dependency coefficient)},  atraminių vektorių klasifikatoriumi (SVM) grįstu absoliučių 
svorių metodą (AW-SVM) \textit{(angl. absolute weight support vector machines)}.
  \item Suprogramuoti multikriterinius dimensijų atrinkimo metodus\cite{5611484}: svoriais 
grįstą multikriterinį suliejimą \textit{(angl. score-based multicriterion fusion)},
reitingais grįstą multikriterinį suliejimą \textit{(ranking-based multicriterion fusion)},
ir svoriais ir reitingais grįstą multikriterinį suliejimą, bei ir svoriais ir
reitingais grįsto multikriterinį rekursyvų dimensijų eliminavimą;
  \item Suprogramuoti  stabiliomis konsensuso grupėmis grįsto dimensijų atrinkimo
metodą\cite{loscalzo2009consensus} \textit{(angl. consensous group
 stable feature selection)}
  \item Palyginti suprogramuotų metodų skaičiavimo laiką, klasifikavimo tikslumą 
bei dimensijų atrinkimo stabilumą.
\end{enumerate}

Praktinės veiklos planas susideda iš dviejų dalių: suprogramuoti pasirinktus 
dimensijų atrinkimo algoritmus ir palyginti suprogramuotus algoritmus 
skaičiavimo laiko, klasifikavimo tikslumo bei dimensijų atrinkimo stabilumo 
atžvilgiais tarpusavyje. Du penktadaliai numatyto profesinės praktikos laiko 
buvo skirta dimensijų atrinkimo algoritmų programavimui, dar du penktadaliai 
buvo algoritmų palyginimui, o likęs laikas dalyvavimui MII rengiamuose 
seminaruose.

Profesinę praktiką atlikinėti pradėjau 2012 metų vasario 6 dieną. Profesinė 
praktika truko 11 savaičių ir baigėsi 2012 metų balandžio 20 dieną. Ilgiau nei 
planuota užtruko dimensijų atrinkimo metodų programavimo darbai, todėl teko
sumažinti dimensijų atrinkimo algoritmų lyginamųjų eksperimentų apimtis.

