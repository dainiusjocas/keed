\addcontentsline{toc}{section}{{\c I}VADAS}
\section*{ĮVADAS}
% Įvadas. Išdėstomi praktikos vietos pasirinkimo motyvai, praktikos užduotis,
% jos tikslas, spręstieji uždaviniai, pateikiama praktinės veiklos planas 
% praktikos atlikimo eiga (2-3 psl.). 	

Profesinei praktikai atlikti pasirinkau Vilniaus universiteto (VU) Matematikos ir informatikos instituto (MII) sistemų analizės skyrių dėl dviejų priežasčių. Pirma, norėjau pasinaudoti galimybe profesinės praktikos metu tęsti bakalauriniame darbe atliekamą tyrimą. Bakalauriniame darbe nagrinėjama biomedicininių, daug atributų turinčių (daugiamačių) duomenų suskirstymo į pageidaujamas kategorijas pagal vidinę duomenų struktūrą -- klasifikavimo -- problema.

Antroji mano pasirinkimo profesinę praktiką atlikti MII priežastis buvo ta, kad dirbdamas MII, turėsiu galimybę konsultuotis su daugiamačių duomenų analizės problematiką tiriančiais mokslininkais, nes norint pradėti spręsti bakalauriniame darbe iškeltą problemą reikia ir tais daugiamačiais duomenimis apibūdinamų procesų dalykinės srities, ir duomenų analizės, ir programavimo žinių. MII mokslininkų sukauptas žinių bagažas labai palengvino ir pagreitino mano susipažinimą su nagrinėjama problematika.

Profesinės praktikos tikslas -- atlikti informatyvių matų (toliau \textit{matų}) atrinkimo metodų palyginamąją analizę. Siekiant užsibrėžto tikslo profesinės praktikos metu buvo sprendžiami šie uždaviniai:
\begin{enumerate}
  \item Susipažinti daugiamačių duomenų matų atrinkimo problematika bei moksline literatūra;
  \item Suprogramuoti matų atrinkimo metodus: 
\textit{Fisher}, \textit{Relief},
asimetrinį priklausomybės koeficientą (angl. \textit{asymetric dependency coefficient}, ADC),
atraminių vektorių klasifikatoriumi (SVM) grįstą absoliučių svorių metodą (angl. \textit{absolute weight support vector machines, AW-SVM}),
svoriais grįstą multikriterinį suliejimą (angl. \textit{score-based multicriterion fusion}),
reitingais grįstą multikriterinį suliejimą (angl. \textit{ranking-based multicriterion fusion}),
svoriais ir reitingais grįstą multikriterinį rekursyvų matų eliminavimą\cite{5611484},
konsensuso grupėmis grįsto stabilių matų atrinkimo metodą\cite{loscalzo2009consensus} (angl. \textit{consensous group stable feature selection})
  \item Palyginti suprogramuotų matų atrinkimo metodų skaičiavimo laiką, klasifikavimo tikslumą bei stabilumą.
\end{enumerate}

Praktinės veiklos planas buvo sudarytas iš dviejų dalių: 
\begin{enumerate}
 \item suprogramuoti pasirinktus matų atrinkimo algoritmus;
 \item palyginti suprogramuotus algoritmus skaičiavimo laiko, klasifikavimo tikslumo bei matų atrinkimo stabilumo atžvilgiais tarpusavyje.
\end{enumerate}
Du penktadaliai numatyto profesinės praktikos laiko buvo skirta matų atrinkimo algoritmų programavimui, dar du penktadaliai buvo numatyti algoritmų palyginimui, o likęs laikas susipažinimui su dalykinės srities literatūra bei dalyvavimui MII rengiamuose seminaruose.

Profesinę praktiką pradėjau 2012 metų vasario 6 dieną. Ji truko 11 savaičių ir baigėsi 2012 metų balandžio 20 dieną. Ilgiau nei planuota užtruko matų atrinkimo metodų programavimo darbai, todėl teko sumažinti matų atrinkimo algoritmų lyginamųjų eksperimentų apimtis.

Likusi praktikos ataskaitos dalis yra organizuota taip: skyriuje \ref{imones_apibudinimas} glaustai aprašau įstaigą, kurioje atlikau profesinę praktiką; skyriuje  \ref{praktikos_veiklos_aprasymas} aprašau praktikos veiklas ir praktikos užduotis; skyriuje \ref{rezultatai_isvados_pasiulymai} aprašau profesinės praktikos darbo rezultatus bei padarytas išvadas, praktikos darbo privalumus bei trūkumus, įgytas žinias bei patirtis, taip pat pateikiu pasiūlymų, kaip galima būtų geriau organizuoti darbo ir valdymo procesus praktikos atlikimo vietoje ir mokymą Vilniaus universitete.