\addcontentsline{toc}{section}{{\c I}VADAS}
\section*{ĮVADAS}
% Įvadas. Išdėstomi praktikos vietos pasirinkimo motyvai, praktikos užduotis,
% jos tikslas, spręstieji uždaviniai, pateikiama praktinės veiklos planas 
% praktikos atlikimo eiga (2-3 psl.). 	

Profesinei praktikai atlikti pasirinkau Vilniaus universiteto matematikos ir 
informatikos instituto (MII) sistemų analizės skyrių dėl keletos priežasčių. Visų pirma, norėjau pasinaudoti 
galimybe profesinės praktikos metu tęsti bakalauriniame darbe atliekamą tyrimą. 
Bakalauriniame darbe nagrinėjama biomedicininių daug atributų turinčių - 
daugiamačių duomenų suskirstymo į pageidaujamas kategorijas pagal vidinę duomenų
struktūrą - klasifikavimo problema.

Norint pradėti spręsti bakalauriniame darbe iškeltą problemą reikia ir tais 
daugiamačiais duomenimis apibūdinamų procesų dalykinės srities, ir duomenų analizės, ir programavimo 
žinių. Todėl antroji mano pasirinkimo profesinę praktiką atlikti MII priežastis yra ta, kad
dirbdamas MII, turėsiu galimybę konsultuotis su daugiamačių 
duomenų analizės problematiką tiriančiais mokslininkais. Jų žinių bagažas
labai palengvino ir pagreitino mano susipažinimą su nagrinėjama problematika.

Profesinės praktikos tikslas - atlikti dimensijų atrinkimo metodų palyginamąją analizę. Siekiant užsibrėžto tikslo profesinės praktikos metu buvo sprendžiami šie 
uždaviniai:
\begin{enumerate}
  \item Susipažinti daugiamačių duomenų dimensijų atrinkimo problematika;
  \item Suprogramuoti dimensijų atrinkimo metodus: 
,,fisher`` įvertis, 
,,relief`` koeficientas,
asimetrinį priklausomybės koeficientą (angl. \textit{asymetric dependency coefficient}),
atraminių vektorių klasifikatoriumi (SVM) grįstu absoliučių svorių metodą (AW-SVM) \textit{(angl. absolute weight support vector machines)},
svoriais grįstą multikriterinį suliejimą \textit{(angl. score-based multicriterion fusion)},
reitingais grįstą multikriterinį suliejimą \textit{(ranking-based multicriterion fusion)},
ir svoriais ir reitingais grįsto multikriterinį rekursyvų dimensijų eliminavimą\cite{5611484},
konsensuso grupėmis grįsto stabilių dimensijų atrinkimo metodą\cite{loscalzo2009consensus} \textit{(angl. consensous group stable feature selection)}
  \item Palyginti suprogramuotų dimensijų atrinkimo metodų skaičiavimo laiką, klasifikavimo tikslumą bei stabilumą.
\end{enumerate}

Praktinės veiklos planas buvo sudarytas iš dviejų dalių: suprogramuoti pasirinktus 
dimensijų atrinkimo algoritmus ir palyginti suprogramuotus algoritmus 
skaičiavimo laiko, klasifikavimo tikslumo bei dimensijų atrinkimo stabilumo 
atžvilgiais tarpusavyje. Du penktadaliai numatyto profesinės praktikos laiko 
buvo skirta dimensijų atrinkimo algoritmų programavimui, dar du penktadaliai 
buvo numatyti algoritmų palyginimui, o likęs laikas susipažinimui su dalykinės srities literatūra bei dalyvavimui MII rengiamuose seminaruose.

Profesinę praktiką atlikinėti pradėjau 2012 metų vasario 6 dieną. Ji truko 11 savaičių ir baigėsi 2012 metų balandžio 20 dieną. Ilgiau nei 
planuota užtruko dimensijų atrinkimo metodų programavimo darbai, todėl teko sumažinti dimensijų atrinkimo algoritmų lyginamųjų eksperimentų apimtis.

Likusi praktikos ataskaitos dalis yra organizuota taip: skyriuje nr. \ref{imones_apibudinimas} glaustai aprašysiu įstaigą, kurioje atlikau profesinę praktiką; skyriuje nr. \ref{praktikos_veiklos_aprasymas} aprašysiu praktikos veiklas ir praktikos užduoties atlikimą profesinės praktikos metu; skyriuje nr. \ref{rezultatai_isvados_pasiulymai} aprašysiu profesinės praktikos darbo rezultatus bei padarytas išvadas, praktikos darbo privalumus bei trūkumus, įgytas žinias bei patirtis, taip pat pateiksiu pasiūlymų, kaip galima būtų geriau organizuoti darbo ir valdymo procesus praktikos atlikimo vietoje ir mokymą Vilniaus universitete.