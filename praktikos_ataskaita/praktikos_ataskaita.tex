\documentclass[12pt, a4paper]{article}

\usepackage[utf8x]{inputenc}
\usepackage[lithuanian]{babel}
\usepackage[L7x]{fontenc}
\usepackage{lmodern}
\usepackage{verbatim}
\usepackage{tocloft} 
\usepackage{url}
\usepackage{setspace}
\usepackage{caption}
\usepackage{parskip}
\usepackage{graphicx}
\usepackage{fixltx2e} % šitas dėl \textsubscript{}
\usepackage{color} % spalvinsim tekstą
\usepackage{mathtools} % pades rašyti matematinius intarpus
\usepackage{titlesec}
\usepackage{indentfirst} % atitraukta pirmoji eilute
\usepackage{longtable} % darysim didelias lenteles
\usepackage{algorithm}
\usepackage{lipsum}% http://ctan.org/pkg/lipsum
\usepackage{geometry}
\usepackage{changepage}
\usepackage{pdfpages} % su šitua paketu galima įkelti pdf kaip puslapius
\linespread{1.5}
\parindent = 1cm
\titlelabel{\thetitle.\quad} % Padeda tašką po skyriaus numerio
\renewcommand{\cftsecleader}{\cftdotfill{\cftdotsep}}

\begin{document}
% Titulinis puslapis
\includepdf{images/praktikos_ataskaitos_titulinis.pdf}

\let \savenumberline \numberline
\def \numberline#1{\savenumberline{#1.}}
\tableofcontents 
\newpage

%% Įvade aprašomi darbo tikslai, nurodomas temos aktualumas, aptariamos teorinės darbo prielaidos bei metodika, apibrėžiamas tiriamasis objektas, apibūdinami su tema susiję literatūros ar kitokie šaltiniai, temos analizės tvarka, darbo atlikimo aplinkybės, pateikiama žinių apie naudojamus instrumentus (programas ir kt.). Darbo įvadas neturi būti dėstymo santrauka. Įvado apimtis 3–4 puslapiai.

\newpage
\section*{ĮVADAS}

%% JG: Idėja - naudok terminus "matai" ir "mėginiai". Kiekvienam mėginiui atliekama labai daug matavimų, iš to ir "daugiamatiškumas"

% JG: Paminėk, kad šiame darbe mes gilinamės į biomedicinoje kaupiamų genetinių duomenų analizės specifika. Šie duomenys ypatingi dėl daugiamatiškumo, mažo mėginių kiekio, triukšmingų matavimų.

Nuolat vystosi technologijos skirtos gauti biomedicininius duomenis, pvz. genomo sekvenavimas\cite{pettersson2009generations}, o tai reiškia, kad didėja gaunamų duomenų detalumas. Detalumas reiškia, kad daugėja biomedicininius duomenis abibūdinančių faktorių arba matų skaičius. Duomenys, kurių kiekvienas mėginys aprašomas dideliu skaičiumi matų, yra vadinami daugiamačiais duomenimis.

Šiame darbe yra nagrinėjama biomedicinoje kaupiamų genetinių daugiamačių duomenų analizės specifika. Šie duomenys yra ypatingi tuo, kad jie įprastai turi šimtus kartų daugiau matų nei mėginių. Santykinai mažas mėginių skaičius turimas, nes mėginio gavimo kaina yra aukšta. Biomedicininių duomenų analizę apsunkina ir tai, kad matavimai, kuriais tie duomenys gaunami, įneša atsitiktinių duomenų - triukšmo. Triukšmas matavimo metu gali atsirasti dėl įvairių priežasčių, pvz. netinkamai paruoštų cheminių preparatų. Kai duomenys yra triukšmingi, didėja tikimybė duomenyse rasti atsitiktinių priklausomybių. Tai yra viena priežasčių, kodėl biomedicininių duomenų analizės procesas yra sudėtingas.

%% JG: Būdai neatsiranda, o vystosi technologija. Jie nėra tikslesni, bet detalesni, t.y. kiekvienam mėginiui atliekama daugiau matavimų. 

%% JG: Nors matavimų kiekis didėja, mėginio kaina išlieka gana aukšta. Todėl biomedicinos eksperimentuose gaunami duomenys ypatingi tuo, kad matų visados ženkliai daugiau nei mėginių.

Klasifikavimu\cite{fisher1936use} yra vadinamas duomenų analizės procesas, kai duomenys suskirstomi į grupes pagal tam tikrus jų požymius. Algoritmai arba funkcijos, kurios turimus duomenis priskiria iš anksto žinomoms grupėms - atlieka klasifikavimą - yra vadinami klasifikatoriais. Klasifikatoriai paruošiami naudojant turimus mėginius - treniravimosi duomenis - ir informaciją apie jų būklę (sveikas ar sergantis). Klasifikatoriaus ruošimo procesas yra vadinamas apmokymu. Apmokyti klasifikatoriai paprastai naudojami nustatant naujų, dar nematytų, mėginių - testavimo duomenų - būklę. Pagal tai, kokią dalį visų testavimo mėginių klasifikatorius priskiria neteisingai klasei, yra nustatomas klasifikatoriaus tikslumas. 

Biomedicininių duomenų tipinė klasifikavimo užduotis yra atskirti sveikų pacientų mėginius nuo sergančiųjų. Klasifikavimu siekiama nustatyti, kurie matai veikdami drauge, geriausiai paaiškina skirtumą tarp ligos paveiktų ir sveikų mėginių. Labiausiai ligą paaiškinančių matų nustatymas galėtų palengvinti tiriamų ligų diagnozės ar gydymo metodų kūrimą.

Biomediciniuose duomenyse dažniausiai turime tik kelias dešimtis mėginių, todėl norint geriau įvertinti klasifikatoriaus tikslumą yra naudojami pakartotinio mėginių poaibio atrinkimo (angl. \textit{resampling}) metodai: kryžminio patikrinimo (angl. \textit{cross-validation}) arba įkelčių (angl. \textit{bootstrap\footnote{Terminas \textit{bootstrap} ,,įkelties`` prasme pradėtas naudoti dar Rudolfo Ericho Raspės knygoje ,,Barono Miunchauzeno nuotykiai``, kurioje Baronas Minchauzenas užkėlė save ant arklio tempdamas į viršų savo batų raištelius (angl. \textit{bootstraps}).}}). Šių metodų naudojimas naudojimas su duomenimis, kurių tikrasis pasiskirstymas nėra žinomas, padeda įvertinti klasifikavimo variabilumą (angl. \textit{variance}) ir sisteminį nuokrypį (angl. \textit{bias}).

Naudojant kryžminio patikrinimo metodą, daug kartų sudaromos skirtingos treniravimosi ir testinės mėginių imtys. Taikant atskirą šio metodo variantą, kryžminį patikrinimą išbraukiant po vieną mėginį (angl. \textit{leave-one-out cross-validation}), iš treniravimosi imties išbraukiamas vienas mėginys ir apmokomas klasifikatorius, kuris klasifikuoja išbrauktąjį mėginį. Procesas tęsiamas tol, kol suklasifikuojami visi objektai. Kitais kryžminio patikrinimo metodo variantais iš treniravimosi mėginių yra išmetama po keletą mėginių. Pagal tai, kiek testinių mėginių klasifikatorius priskyrė klaidingai kategorijai, yra nustatoma vidutinė klaidingo klasifikavimo tikimybė. Šiuo metodu gauti įverčiai pasižymi didele dispersija \cite{braga2004cross}.

Naudojant įkelčių metodą, iš $n$ dydžio mėginių aibės yra paimama tokio pačio dydžio atsitiktinių mėginių imtis su pasikartojimais, kuri vadinama įkelties treniravimosi imtimi. Į šią imtį nepaimti mėginiai yra priskiriami testavimo imčiai. Naudojant įkelties treniravimosi mėginių imtį yra apmokomas klasifikatorius, kuris klasifikuoja testavimo imtį. Procesą kartojant gaunama klaidingo klasifikavimo tikimybės įverčių imtis. Šios imties vidurkis yra klaidingo klasifikavimo tikimybės įvertis. Dažniausiai naudojamas ,“0.623 įkelčių`` (angl. \textit{0.623\footnote{0.623 yra tikimybė mėginiui būti įtrauktam į treniravimosi imtį.} bootstrap}) įverčiu. Šiuo metodu gautas klaidingo klasifikavimo tikimybės įverti pasižymi maža dispersija \cite{michie1994machine}.

Biomedicininių duomenų kontekste galima daryti prielaidą, kad ne visi matai yra susiję su tiriama problema, pvz. gaubtinės žarnos vėžiu, dėl tokių faktorių, kaip triukšmas duomenyse. Paprastai nagrinėjamai problemai svarbus yra mažas, palyginus su visu, matų kiekis. Ši biomedicininių duomenų ypatybė veda prie ,,daugiamatiškumo prakeiksmo`` (angl. \textit{the curse of dimentionality})\cite{bellman1966adaptive} - didėjant matų
kiekiui mėginiai pasidaro panašūs, o bandymas juos klasifikuoti tolygus spėliojimui. Todėl biomedicininių duomenų daugiamatiškumui sumažinti yra naudojami informatyviausių dimensijų atrinkimo metodai\cite{guyon2003introduction} (angl. \textit{feature selection}). Pagal tai, kaip susiję su klasifikatoriumi, matų atrinkimo metodai skirstomi į tris kategorijas\cite{saeys2008robust}: filtruojantys (angl. \textit{filter}), 
prisitaikantys (angl. \textit{wrapper}) ir įterptiniai (angl. \textit{embedded}) metodai. Dimensijų atrinkimas yra svarbi biomedicininių duomenų apdorojimo (angl. \textit{preprocessing}) etapo dalis. Naudojant dimensijų atrinkimo metodus galima kovoti su ,,daugiamatiškumo prakeiksmu`` dimensijų skaičių priartinant prie mėginių skaičiaus.

%% JG: Kodėl klasifikuojama? Norima nustatyti, kokie matai, veikdami drauge, geriausiai paaiškina skirtumą tarp ligos paveiktų ir sveikų mėginių.

%% JG: Kokios yra tradicinės klasifikavimo strategijos ir kodėl jos neveikia daugiamačių duomenų atveju? Skaičiavimo laikas nėra problema - Random forests veikia visai neblogai tokiais atvejais. Daugiamatiškumas veda prie the curse fo dimensionality.

% JG: sumažinus naudojamų matų kiekį, matų kiekis priartėja prie mėginių kiekio ir tokiu būdu apeinamas daugiamatiškumo prakeiksmo problema. Biomedicinos duomenų kontekste, galima daryti prielaidą, kad dauguma matų yra beprasmiai, pvz., tik kai kurie genai įtakoja ligą, todėl matų mažinimimas yra prasmingas. Taip pat, kuriant medicininius diagnostikos įrankius, naudojamų matų kiekis įtakoja įrankio kainą. Todėl pageidautina turėti kuo mažiau matų.

Kadangi biomedicininiuose duomenyse reikšmingų dimensijų kiekis tiriamai problemai yra nedidelis, todėl norima žinoti, kuris dimensijų poaibis yra svarbus tai problemai. Tokioje situacijoje tampa svarbu, kaip varijuoja atrenkamų dimensijų aibė, kai dimensijų atrinkimas vykdomas su vis kitu mėginių poaibiu. Dimensijos, kurios keičiant mėginių, kurie naudojami dimensijų atrinkime, poaibį, yra vėl ir vėl atrenkamos yra vadinamos stabiliomis dimensijomis \cite{needcitation}. Tačiau skirtingi dimensijų atrinkimo metodai tiems patiems mėginiams gali atrinkti skirtingas dimensijas. Taip pat, suskaidžius duomenis į persidengiančius poaibius ir atrinkus tą patį kiekį dimensijų tuo pačiu metodu, gaunami skirtingi dimensijų poaibis. Tačiau, norint geriau suprasti biomedicininius duomenis, itin svarbu fokusuoti dėmesį į sąlyginai nedidelį dimensijų poaibį. Dimensijų aibės sumažinimas paspartina biomedicininių duomenų tyrimus - tyrėjams reikia atlikinėti bandymus su mažesniu mėginių skaičiumi. Mažesnio skaičiaus mėginių tyrimas kainuoja mažiau, nes mažiau reikia žmonių darbo laiko, mažiau reikia ir cheminių reagentų. Todėl stabilių dimensijų atrinkimas dirbant su biomedicininiais duomenimis yra \textit{(angl. robustness)}.

% JG: Žiūrėkim į stabilumą kaip į šalutinį matų atrinkimo efektą, kurį svarbu pažaboti. Stabilumas svarbus, nes, analizuojant duomenis norima ne tik nustatyti, koks būtų vidutinis klasifikatoriaus tikslumas, bet ir sukurti tą vidutinį klasifikatorių. Norint pastarąjį sukurti, reikia žinoti, kuriuos konkrečius matus naudoti. 

% JG: Šitam paragrafe suplaki daug svarbių dalykų į krūvą ir juos turėtum būti paaiškinęs jau anksčiau. Mėginių trūkumas nėra tavo sprendžiama problema. Tačiau, kuo mažiau duomenų, tuo nestabilesni atrenkami matai. 

%Taigi, dirbant su daugiamačiais duomenimis, reikia atsižvelgti į keletą kriterijų:
%\begin{enumerate}
% \item Klasifikavimo tikslumą;
% \item Dimensijų atrinkimo stabilumą, atsižvelgiant į klasifikavimo rezultatus;
% \item Triukšmo lygį duomenyse;
% \item Skaičiavimo išteklių naudojimo racionalumą.
%\end{enumerate}
%Reikalavimas vienu metu atsižvelgti į keletą kriterijų užduotį daro sudėtinga. Klasifikuojant daugiamačius duomenis uždavinys yra surasti geriausius rezultatus duodančią strategiją, kuri geriausiai atsižvelgia į minėtus kriterijus.

%Darbo eksperimentinei daliai reikalingus skaičiavimo išteklius, suteikė VU MIF skaitmeninių tyrimų ir skaičiavimų centras \cite{mif2012stsc}. Eksperimentuose buvo naudojami laisvai internete prieinami biomedicininių duomenų rinkiniai (angl. \textit{datasets}). Biomedicininių duomenų apdorojimo algoritmų implementavimui buvo naudojama R \cite{r2012statistics} programavimo kalba. Eksperimentai atlikti profesinės praktikos MII metu.

Dimensijų atrinkimo stabilumo problemą Yang ir Mao \cite{yang2011robust} siūlė spręsti reitinguojant dimensijas remiantis keletos dimensijų atrinkimo metodų rezultatais. Galutinis dimensijų reitingų sąrašas gaunamas, kai po kiekvieno dimensijų atrinkimo yra išmetama viena žemiausią reitingą turinti dimensija iš dimensijų aibės, ir dimensijų atrinkimas yra kartojamas tol, kol nebelieka dimensijų. Tačiau dimensijų atrinkimo metodų kiekis yra ribotas ir skirtingų metodų dažnai negalima atlikti paraleliai. Tai riboja šio metodo pritaikomumą daugiamačių duomenų analizėje.

Dimensijų atrinkimo stabilumo problemą siūlyta spręsti surandant dimensijų grupių tankio centrus ir naudoti dimensijas, kurios artimiausios tiems centrams \cite{yu2008stable}. Pasiūlytas grupių tankių algoritmas užtrunka $O(\lambda n^2m)$ laiko, kur n yra dimensijų kiekis, o m - mėginių skaičius. Vėliau Loscalzo ir kt. pasiūlė mokymo duomenis skaidyti poaibiais ir kiekviename poaibyje ieškoti tankių grupių, o tada imti sprendimą balsavimo principu \cite{loscalzo2009consensus}. Nors šie metodai siūlo stabilų dimensijų atrinkimą, tačiau šių metodų panaudojamumą daugiamačiuose duomenyse riboja skaičiavimo sudėtingumas.

Šiame bakalauriniame darbe remiantis Yang, Mao bei Loscalzo darbuose pateiktomis įžvalgomis, bus stengiamasi pasiūlyti tyrimų kryptis, kurios galėtų padėtų sukurti metodus, skirtus spręsti stabilių dimensijų atrinkimo problemą. Idėja yra sugrupuoti dimensijas pagal greitą klasterizacijos algoritmą, išrinkti reprezentatyviausias dimensijas, transformuoti dimensijų erdvę ir joje vykdyti dimensijų atrinkimą remiantis keletu dimensijų atrinkimo metodų.

Šio darbo tikslas yra išanalizuoti darbo su daugiamačiais duomenis ypatybes. Šiam darbui yra keliamos tokios užduotys:
\begin{enumerate}
 \item Susipažinti su naujausiais klasifikavimo ir dimensijų atrinkimo metodais;
 \item Atlikti dimensijų atrinkimo metodų palyginimo eksperimentus;
 \item Pasiūlyti kryptis, kaip dabartiniai metodai gali būti patobulinti ir paruošti naujųjų metodų prototipus.
\end{enumerate}

Tolimesnė bakalaurinio darbo struktūra yra tokia: skyriuje


\newpage

% 2.	Įmonės/įstaigos apibūdinimas. Glaustai aprašoma įmonė/įstaiga, kurioje buvo
% atliekta praktika: jos veiklos sritis, organizacinė struktūra, teikiamos 
% paslaugos ir kt. Apibūdinamos praktikos vietoje sudarytos darbo sąlygos
% (1-2 psl.).

\section{ĮSTAIGOS APIBŪDINIMAS}
\label{imones_apibudinimas}

Vilniaus universiteto matematikos ir informatikos institutas (MII) nuo 2010 m. yra Vilniaus universiteto padalinys užsiimantis tyrimais matematikos ir informatikos srityse. Instituto įkūrimo data laikoma 1965 m. spalio 1d., kai buvo panaikintas Lietuvos mokslų akademijos Fizikos ir technikos institutas ir įkurti trys nauji institutai, tarp kurių buvo Fizikos ir matematikos institutas, kuris laikomas MII pirmtaku. 

Pagrindinė instituto veikla - moksliniai tyrimai ir eksperimentinė plėtra. Kitos veiklos sritys yra: mokslininkų ugdymas (doktorantūros studijos) (MII suteikta teisė ruošti matematikos, informatikos ir informatikos inžinerijos sričių mokslininkus); mokslo organizacinė veikla - konferencijos, seminarai, parodos, mokslinių knygų redagavimas; leidyba; mokymas, moksleivių ugdymas, švietimas. Mokslinė veikla sukoncentruota 12-oje mokslinių padalinių. Institute yra 5 matematikos krypties padaliniai, 7 informatikos bei informatikos inžinerijos padaliniai:
\begin{enumerate}
  \item Atpažinimo procesų skyrius;
  \item Atsitiktinių procesų skyrius;
  \item Informatikos metodologijos skyrius;
  \item Kompiuterinių tinklų laboratorija;
  \item Matematinės logikos sektorius;
  \item Programų sistemų inžinerijos skyrius;
  \item Sistemų analizės skyrius (SAS);
  \item SAS optimizavimo sektorius;
  \item SAS operacijų tyrimo sektorius;
  \item Skaičiavimo metodų skyrius (SMS);
  \item SMS diferencialinių lygčių sektorius;
  \item Tikimybių teorijos ir statistikos skyrius;
\end{enumerate}

MII organizuoja moksleivių ugdymą: veikia jaunųjų programuotojų neakivaizdinė mokykla, rengiamos lietuvos moksleivių informatikos ir matematikos olimpiados, rengiamas informacinių technologijų konkursas ,,Bebras``. MII yra vienas iš Lietuvos jaunųjų matematikų mokyklos steigėjų, jaunųjų matematikų konkurso ,,Kengūra`` rengėjas. Taip pat MII prisideda prie kompiuterijos naudotojų švietimo ir mokymo: dirba informatikos terminijos komisija, multimedijos centras humanitarams, palaikomas tinklalapis apie lietuviškų rašmenų naudojimą elektroninio pašto laiškuose.

MII leidybos skyrius atsakingas už visą eilę recenzuojamų periodinių leidinių: ,,Informatica``, ,,Informatics in Education``, ,,Lithuanian Mathematical Journal``, ,,Lietuvos matematikos rinkinys. LMD darbai``, ,,Mathematical Modelling and Analysis``, ,,Nonlinear Analysis. Modelling and Control``, ,,Olympiads in Informatics``. MII taip pat yra išleidusi mokslinių bei mokslo populiarinimo knygų lietuvių ir anglų kalbomis, mokymo priemonių, interaktyvių kompaktinių diskų bei sukūrusi įvairių internetinių informacinių sistemų (pvz. enciklopedinis kompiuterijos terminų žodynas).

MII man, kaip ir kiekvienam darbuotojui, parūpino: darbo vietą, prisijungimo prie vietinio tinklo, galimybe naudotis skaičiavimo ištekliais, galimybe su nuolaida pietauti vietinėje valgykloje. MII darbuotojai buvo kolegiški, todėl apsipratimas MII įvyko labai greitai. Todėl jau nuo pat pirmosios profesinės praktikos dienos galėjau pradėti spręsti užsibrėžtus uždavinius.

\newpage

\section{DIMENSIJŲ ATRINKIMAS}
%\section{PAGRINDINIAI DIMENSIJŲ ATRINKIMO METODAI}

Biomedicininių duomenų kontekste galima daryti prielaidą, kad ne visi matai yra susiję su tiriama problema, pvz. gaubtinės žarnos vėžiu, dėl tokių faktorių, kaip triukšmas duomenyse. Paprastai nagrinėjamai problemai svarbus yra mažas, palyginus su visu, matų kiekis. Ši biomedicininių duomenų ypatybė veda prie ,,daugiamatiškumo prakeiksmo`` (angl. \textit{the curse of dimentionality}) \cite{bellman1966adaptive} - didėjant matų kiekiui mėginiai pasidaro panašūs, todėl bandymas juos klasifikuoti tolygus spėliojimui. Vienas iš būdų kovoti su ,,daugiamatiškumo prakeiksmu`` yra naudoti dimensijų atrinkimo metodus. Dimensijų atrinkimas yra svarbus etapas biomedicininių duomenų pirminiam apdorojimui (angl. \textit{preprocessing}). Dimensijų atrinkimas dažniausiai yra naudojamas surasti mažiausią dimensijų poaibį, kuris maksimaliai pagerina klasifikatoriaus tikslumą.

Pagal tai, kaip dimensijų atrinkimo metodai yra susiję su klasifikatoriumi, dimensijų atrinkimo metodus galima skirstyti į tris kategorijas \cite{saeys2008robust}:
\begin{enumerate}
 \item Filtruojantys metodai (angl. \textit{filter methods}), pvz. \textit{Fisher} įvertis. Jie dirba tiesiogiai su duomenimis, o jų rezultatas gali būti dimensijų įvertinimas svoriais, dimensijų reitingavimas ar tiesiog geriausių dimensijų poaibis, kuriuo remiantis vėliau apmokomas klasifikatorius. Tokių metodų pagrindinis privalumas yra tai, kad jie yra greiti, tinka paskirstytų skaičiavimų aplinkoms ir nepriklausomi nuo klasifikavimo  metodo, tačiau remiantis atrinktosiomis dimensijomis nebūtinai bus sukurtas geriausias klasifikatorius.
 \item Prisitaikantieji metodai (angl. \textit{wrapper methods}). Pirma, apmokomas klasifikatorius su visomis dimensijomis, antra, parenkamas dimensijų poaibis ir apmokomas klasifikatorius, tada po daugkartinio dimensijų aibių įvertinimo pagal klasifikavimo rezultatus yra nusprendžiama, kuris dimensijų poaibis yra labiausiai tinkamas klasifikavimui. Įterptinių metodų atveju dimensijų atrinkimo procesas yra neatsiejamas nuo klasifikavimo proceso - pats klasifikatorius įvertina dimensijas. Jie dažnai duoda geresnius rezultatus negu filtravimo metodai, bet yra reiklūs resursams.
 \item Įterptiniai metodai (angl. \textit{embedded methods}), pvz. AW-SVM\cite{vapnik2000nature}. Jie dimensijų atrinkimui naudoja vidinius klasifikatoriaus duomenis (pvz. dimensijų svoriai gauti pagal SVM). Šie metodai dažnai siūlo gerą santykį tarp klasifikavimo tikslumo ir skaičiavimų sudėtingumo.
\end{enumerate}

Šiame skyriuje nagrinėsiu pagrindinius dimensijų atsirinkimo metodus:
\begin{enumerate}
 \item \textit{Fisher} įvertis (angl. \textit{Fisher ratio})\cite{Pavlidis:2001:GFC:369133.369228};
 \item \textit{Relief} metodas\cite{DBLP:journals/ml/Robnik-SikonjaK03};
 \item Asimetrinis priklausomybės koeficientas\cite{Shannon:2001:MTC:584091.584093} (angl. \textit{Asymmetric Dependency Coefficient, ADC});
 \item Absoliučių svorių SVM\cite{vapnik2000nature} (AW-SVM) (angl. \textit{Absolute Weight SVM})
 \item Rekursyvus dimensijų eliminavimas pagal SVM\cite{Guyon:2002:GSC:599613.599671} (SVM-RFE) (angl. \textit{Recursive Feature Elimination by SVM})
\end{enumerate}

\subsection{\textit{Fisher} įvertis}

\textit{Fisher} įvertis vertina individualias dimensijas pagal dimensijos klasių atskiriamąją galią. Dimensijos įvertis yra sudarytas iš tarpklasinio skirtumo santykio su vidiniu klasės pasiskirstymu:
\begin{equation}
 FR(j) = \frac{(\mu_{j1} - \mu_{j2})^2}{\sigma_{j1}^2 + \sigma_{j2}^2},
\end{equation}
kur, 
$j$ - yra dimensijos indeksas, 
$\mu_{jc}$ - dimensijos $j$ reikšmių vidurkis klasėje $c$, 
$\sigma_{jc}^2$ - dimensijos $j$ reikšmių standartinis nuokrypis klasėje $c$, kur $c={1,2}$. Kuo didesnis yra \textit{Fisher} įvertis, tuo geriau ta dimensija atskiria klases.

\subsection{\textit{Relief} metodas}

\textit{Relief} metodas iteratyviai skaičiuoja dimensijų ,,susietumą``. Pradžioje
,,susietumas`` visoms dimensijoms yra lygus nuliui. Kiekvienoje
iteracijoje atsitiktinai\footnote{Pastebėtina, kad dėl atsitiktinumo faktoriaus klasifikavimo ir  dimensijų atrinkimo stabilumo
rezultatai varijuoja.} pasirenkamas objektas iš mėginių aibės, surandami
artimiausi kaimynai iš tos pačios ir kitos klasės, ir atnaujinamos visų 
dimensijų ,,susietumo`` reikšmės. Dimensijos įvertis yra vidurkis visų objektų
atstumų iki artimiausių kaimynų iš tos pačios ir kitos klasės:
\begin{equation}
 W(j)=W(j) - \frac{diff(j, x, x_H)}{n} + \frac{diff(i, x, x_M)}{n},
\end{equation}
kur 
$W(j)$ - $j$-osios dimensijos ,,susietumo`` įvertis, 
$n$ - mėginių aibės dydis, 
$x$ - atsitiktinai pasirinktas mėginys, 
$x_H$ - artimiausias $x$ kaimynas iš tos pačios klasės (angl. \textit{nearest-Hit}), 
$x_M$ - artimiausias $x$ kaimynas iš kitos klasės(angl. \textit{nearest-Miss}),
$diff(j, x, x')$ - $j$-osios dimensijos reikšmių skirtumas tarp laisvai pasirinkto objekto $x$ ir atitinkamo kaimyno, kur skirtumą į intervalą $[0, 1]$ normalizuojanti funkcija yra:
\begin{equation}
 diff(j, x, x')=\frac{|x_j- x_j'|}{x_{j_{max}} - x_{i_{min}}},
\end{equation}
kur $x_{j_{max}}$ ir $x_{j_{min}}$ yra maximali ir minimali $j$-osios dimensijos reikšmės. ,,Susietumo`` reikšmių atnaujinimas yra vykdomas $n$ kartų ir kuo didesnė galutinė reikšmė, tuo svarbesnė dimensija. Pastebėtina, kad aprašyta algoritma versija yra skirta dirbti su dviejų klasių atveju, tačiau yra ir multiklasinis algoritmo variantas.

\subsection{Asimetrinis priklausomybės koeficientas}

Asimetrinis priklausomybės koeficientas (ADC) yra dimensijų reitingavimo motodas, kuris matuoja klasės $Y$ etiketės (angl. \textit{label}) tikimybinę priklausomybę $j$-ąjai dimensijai, naudodamas informacijos prieaugį \cite{kent1983information} (angl. information gain):
\begin{equation}
 ADC(Y, j) = \frac{MI(Y, X_j)}{H(Y)},
\end{equation}
kur $H(Y)$ - klasės $Y$ entropija \cite{Shannon:2001:MTC:584091.584093}, o $MI(Y, X_j)$ - yra bendrumo informacija \cite{Shannon:2001:MTC:584091.584093} (angl. mutual information) tarp klasės etiketės $Y$ ir $j$-osios dimensijos
\begin{equation}
 H(Y)=-\sum_y{p(Y=y)log{p(Y=y)}}, 
\end{equation}
\begin{equation}
 H(X_j)=-\sum_x{p(X_j=x) log{p(X_j=x)}},
\end{equation}
\begin{equation}
 MI(Y, X_j) = H(Y) + H(X_j) - H(Y, X_j),
\end{equation}
\begin{equation}
 H(Y, X_j) = -\sum_{y,x_j}{p(y, x_j)log{p(y, x_j)}},
\end{equation}
Kuo didesni ADC įverčiai, tuo dimensija yra svarbesnė, nes turi daugiau informacijos apie mėginių klases.

\subsection{Absoliučių svorių SVM}

Atraminių vektorių metodas (SVM) yra vienas populiariausių klasifikavimo algortimų, nes jis gerai susidoroja su daugiamačiais duomenimis \cite{guyon2002gene}. Yra keletas bazinių SVM variantų \cite{vapnik2000nature}, bet šiame darbe naudosime tiesinį SVM, nes jis demonstruoja gerus rezultatus analizuojant genų ekspresijos duomenimis. Tiesinis SVM yra hiperplokštuma apibrėžta kaip:
\begin{equation}
 \sum_{j=1}^{p}{w_jx_j + b_0 = 0},
\end{equation}
kur $p$ - dimensijų kiekis, $w_j$ - j-osios dimensijos svoris, $x_j$ - j-osios
dimensijos kintamasis, $b_0$ - konstanta. Dimensijos absoliutus\footnote{Svorį
reikia imti absoliutaus dydžio, nes neigiamas svoris implikuoja priklausomybę 
vienai klasei, o teigiamas kitai klasei.} svoris $w_j$ gali būti panaudotas
dimensijų reitingavimui. Pastebėtina, kad svorių nustatymas yra atliekamas tik 
vieną kartą\footnote{SVM-RFE dimensijų atrinkimo metodas svorius nustato daug kartų.}.

\subsection{Rekursyvus dimensijų eliminavimas pagal SVM}

Rekursyvus dimensijų eliminavimas pagal SVM \cite{guyon2002gene} yra vienas populiariausių dimensijų
atrinkimo algoritmų. Todėl, jis yra naudojamas, kaip atskaitos taškas (angl. \textit{benchmark})
vertinant kitus dimensijų atrankos metodus. Iš esmės šis metodas yra daugkartinis 
absoliučių svorių SVM metodo taikymas nuolat išmetinėjant dimensijas su 
mažiausiais svoriais. Rekursyvus dimensijų eliminavimas mums padeda surasti 
tinkamą dimensijų poaibį, kas nevisada pavyksta su dimensijų reitingavimo 
metodais. Bendroji rekursyvaus dimensijų eliminavimo procedūra:
\begin{algorithm}
\caption{Rekursyvus dimensijų eliminavimas}
\label{RFE}
 \begin{enumerate}
 \item Turime pilną dimensijų rinkinį $F_0$, nustatome $i=0$;
 \item Įvertiname kiekvienos dimensijos kokybę dimensijų aibėje $F_i$;
 \item Išmetame mažiausiai kokybišką dimensiją iš $F_I$ tam, kad gautume
 dimensijų rinkinį $F_{i+1}$;
 \item Nustatome $i=i+1$ ir grįžtame į antrąjį žingsnį kol nėra patenkinta 
 algoritmo pabaigos sąlyga.
\end{enumerate}
\end{algorithm}
Jei trečiajame algoritmo žingsnyje iš dimensijų aibės yra pašalinama tik viena dimensija, tai gauname dimensijų reitingavimą, o jei pašalinamas fiksuotas skaičius ar dalis (pvz. 50\%) dimensijų, tai dimensijų reitingavimo negauname. Pastebėtina, kad rekursyvus dimensijų eliminavimas labai padidina algoritmo sudėtingumą. Algoritmo pabaigos sąlyga gali būti koks nors konkretus dimensijų skaičius arba tiesiog dimensijų aibę mažinti tol, kol dimensijų visai nebeliks.

\section{STABILIŲ DIMENSIJŲ ATRINKIMO METODAI}
\label{stabiliu_dimensiju_atrinkimo_metodai}

Naudodami dimensijų atrinkimo metodus, biomedicininius duomenis tiriantys mokslininkai susiduria su atrinktųjų dimensijų aibės stabilumo problema - atrenkant dimensijas pagal kitą mėginių poaibį, gaunamas kitas dimensijų poaibis. Dimensijų atrinkimo nestabilumas yra sąlygotas šių veiksnių:
\begin{enumerate}
 \item Duomenys yra triukšmingi ir kai kurios dimensijos gali būti palaikytos informatyviomis grynai dėl atsitiktinių priežasčių;
 \item Daugiamačiuose duomenyse tikėtina, kad dalis dimensijų koreliuoja, todėl, kuri iš koreliuojančių dimensijų bus pasirinkta, priklauso nuo to, kuriuos mėginius pasirinksime klasifikatoriaus apmokymui;
 \item Kiekvienas dimensijų atrinkimo algoritmas daro skirtingas prielaidas apie tai, kurios dimensijos yra informatyvios.
\end{enumerate}
Galime teigti, kad skirtingi metodai tiems patiems duomenims gali atrinkti skirtingas dimensijas. Taip pat, suskaidžius turimus duomenis į atskiras persidengiančias aibes ir atrinkus tą patį kiekį dimensijų tuo pačiu metodu, gaunamos skirtingos dimensijų aibės. Be to, kuo triukšmingesni duomenys, kuo mažiau turima mėginių ir kuo daugiau yra dimensijų, tuo ryškesnė yra ši problema \cite{loscalzo2009consensus}. 

Vienas iš būdų didinti dimensijų atrinkimo stabilumą yra naudoti multikriterinius dimensijų atrinkimo metodus. Jų esmė yra panaudoti kelis dimensijų atrinkimo metodus suliejant jų rezultatus į vieną bendrą rezultatą. Yra skiriamos trys priežastys, kodėl keletas agreguotų silpnų ir nestabilių dimensijų atrinkimo metodų gali duoti stabilesnius dimensijų atrinkimo rezultatus \cite{dietterich2000ensemble}:
\begin{enumerate}
 \item Keletas skirtingų, bet vienodai optimalių hipotezių gali būti teisingos, ir kriterijų agregavimas sumažiną tikimybę, kad bus pasirinkta neteisinga  hipotezė;
 \item Atskiri dimensijų metodai gali dirbti skirtinguose lokaliuose optimumuose, tuo tarpu agregavimas gali geriau reprezentuoti tikrąją  duomenis generuojančią funkciją;
 \item Tikroji duomenų funkcija negali būti reprezentuojama jokia hipoteze paskiro algoritmo hipotezių erdvėje ir agreguojant pavienių metodų rezultatus galima praplėsti hipotezių erdvę.
\end{enumerate}
Apibendrinant galima sakyti, kad suliejant keletą skirtingų dimensijų atrinkimo metodų rezultatų suliejamos gerosios pavienių dimensijų atrinkimo metodų savybės, taip bandant kompensuoti tų algoritmų silpnybes.

Šiame skyriuje aptarsiu dimensijų atrinkimo stabilumą didinančius metodus:
\begin{enumerate}
 \item Svoriais grįstas multikriterinis suliejimas;
 \item Reitingais grįstas multikriterinis suliejimas;
 \item Svoriais ir reitingais grįstas multikriterinis suliejimas;
 \item Multikriterinis rekursyvus dimensijų eliminavimas;
 \item Konsensuso grupėmis grįstas stabilių dimensijų atrinkimo metodas.
\end{enumerate}

\subsection{Svoriais grįstas multikriterinis suliejimas}

Svoriais grįsto multikriterinio dimensijų atrinkimo suliejimo pagal svorius algoritmo
pirmajame žingsnyje kiekvienas bazinis metodas priskiria duomenų rinkinio
dimensijoms svorius, tada tie svoriai yra kombinuojami į vieną sutarties
(angl. consensus) svorių vektorių, kurio pagrindu yra gaunami dimensijų 
reitingai. Algoritmas yra pavaizduotas ~\ref{fig:figure4} pav.
\begin{figure}
 \centering
 \includegraphics[width=1\textwidth]{images/score_based_fusion.pdf}
 \caption{Svoriais grįstas multikriterinis suliejimas.}
 \label{fig:figure4}
\end{figure}
\begin{figure}[ht]
\begin{minipage}[b]{0.5\linewidth}
\centering
\includegraphics[width=1\textwidth]{images/boxplot_colon_all.png}
\caption{Pavienių dimensijų atrinkimo metodų nenormalizuotas svorių
pasiskirstymas.}
\label{fig:figure1}
\end{minipage}
\hspace{0.5cm}
\begin{minipage}[b]{0.5\linewidth}
\centering
\includegraphics[width=1\textwidth]{images/boxplot_colon_all_normalized.png}
\caption{Pavienių dimensijų atrinkimo metodų normalizuotas svorių
pasiskirstymas.}
\label{fig:figure2}
\end{minipage}
\end{figure}

Suliejant svorius svarbu yra užtikrinti, kad svoriai, gauti naudojant
skirtingus bazinius kriterijus, būtų palyginami. Todėl svorių normalizavimas
turi būti atliekamas prieš svorių kombinavimą. Kitu
atveju dimensijų įvertinimo metodai bus nepalyginami. Paveikslėlyje ~\ref{fig:figure1} pav.
nenormalizuotų pavienių dimensijų vertinimo metodų skiriasi netgi suteiktų
svorių intervalai. Paveikslėlyje ~\ref{fig:figure2} pav. matome,
kad net ir normalizavus svorius gana stipriai skiriasi svorių kvartiliai - į 
tai reikia atkreipti dėmesį interpretuojant galutinius dimensijų vertinimo 
rezultatus. Šiame darbe svoriai yra 
normalizuoti intervale $[0, 1]$ pagal formulę:
\begin{equation}
 u_i'=\frac{u_i - u_{i_{min}}}{u_{i_{max}} - u_{i_{min}}}, 
\end{equation}
kur $u_i$ - dimensijų svorių vektorius pagal $i$ kriterijų, 
$u_{i_{min}}$ - minimali $u_i$ svorių vektoriaus reikšmė,
$u_{i_{max}}$ - maksimali $u_i$ svorių vektoriaus reikšmė,
$u_i'$ - normalizuotų svorių vektorius.

Sutarties svorių vektorius $u$ yra vidurkis normalizuotų svorių vektorių:
\begin{equation}
 u = \frac{1}{m}\sum_{i=1}^m u_i',
\end{equation}
kur $m$ yra bazinių kriterijų skaičius. Reikia paminėti, kad didesnė svorio
reikšmė reiškia, kad dimensija yra geresnė.

\subsection{Reitingais grįstas multikriterinis suliejimas}

Reitingais grįsto multikriterinio suliejimo pagal reitingus metodas gauna
duomenų rinkinio dimensijų reitingą,
pagal keletą bazinių dimensijų reitingavimo kriterijų. Algoritmo pirmajame žingsnyje
keletas dimensijų atrinkimo kriterijų grąžina dimensijų reitingu, paskui tie
reitingai yra kombinuojami į vieną bendra dimensijų reitingą.  Algoritmas yra
pavaizduotas ~\ref{fig:figure5} pav.
\begin{figure}
 \centering
 \includegraphics[width=1\textwidth]{images/ranking_based_fusion.pdf}
 \caption{Reitingais grįstas multikriterinis suliejimas.}
 \label{fig:figure5}
\end{figure}
Suliejimo pagal reitingus metodas nereikalauja dimensijų atrinkimo metodų 
rezultatų normalizavimo, nes tiesiog imame dimensijoms priskirtus reitingus ir 
juos kombinuojame. Skirtingai nei suliejimo pagal svorius algoritme, baziniai 
dimensijų atrinkimo kriterija dimensijų eliminavimas\cite{yang2011robust} susideda iš dviejų
dalių: keletos dimensijų atrinkimi turi gražinti dimensijų reitingus, o ne svorius.

Dimensijų reitingų kombinavimui yra keletas metodų\cite{dwork2001rank}, tačiau
paprastumo dėlei šiame darbe naudosiu Borda balsavimą\footnote{Dar žinomas kaip
,,Pažymių metodas``. Jis buvo pasiūlytas prancūzų matematiko ir fiziko 
Jean-Charles de Borda 1770 metais.} (angl. Borda count). Tarkime, kad turime
$m$ basuotojų ir $p$ kandidatų aibę. Tada Borda balsavimo metodas kiekvienam
$i$-ajam balsuotojui sukuria balsų vektorių $v_i$ tokiu būdu: geriausiai 
įvertintam kandidatui suteikiama $p$ taškų, antrajam kandidatui $p-1$, ir t.t.
Galutiniai taškai yra gaunami sudedant visų balsuotojų taškus
\begin{equation}
 v = \sum_{i=1}^m v_i,
\end{equation}
kur $v$ yra suminių taškų vektorius, o iš jo galime gauti ir dimensijų reitingus.

\subsection{Svoriais ir reitingais grįstas multikriterinis suliejimas}

Svoriais ir reitingais grįsto multikriterinio suliejimo metodas
nuo reitingais grįsto multikriterinio suliejimo metodo skiriasi tuo, kad kaip dar vienas 
reitingas yra panaudojamas svoriais grįsto multikriterinio dimensijų atrinkimo metu
gautas reitingas.
Multikriterinio dimensijų įverčių ir pagal svorius, ir pagal reitingus metodas vyksta trimis
žingsniais:
\begin{enumerate}
  \item Gauname dimensijų reitingus pagal $m$ pavienių dimensijų atrinkimo motodų;
  \item Suliejame dimensijų įverčius pagal svorius ir taip gauname vieną 
  dimensijų reitingą;
  \item Reitinguojame dimensijas pagal visus turimus $m+1$ pavienius reitingus.
\end{enumerate} 
Algoritmas yra pavaizduotas ~\ref{fig:figure3} pav.
\begin{figure}
 \centering
 \includegraphics[width=1\textwidth]{images/score_and_ranking_based_fusion.pdf}
 \caption{Svoriais ir reitingais grįstas multikriterinis suliejimas.}
 \label{fig:figure3}
\end{figure}

Kadangi yra suliejami keli mažai koreliuojantys dimensijų reitingavimo metodai,
yra pasiekiamas didesnis dimensijų atrinkimo stabilumas, kai varijuoja 
treniravimosi duomenų poaibis (angl. subsampling).

\subsection{Multikriterinis rekursyvus dimensijų eliminavimas}

Jei dimensijų atrinkimo tikslas yra pagerinti klasifikavimo rezultatus, tai taikymas
multikriterinio dimensijų atrinkimo metodų nebūtinai duos pageidaujamą rezultatą,
nes yra pastebėta, kad vien dimensijų reitingavimas nebūtinai suranda geriausią dimensijų 
poaibį. Tam, kad būtų surastas geriausias dimensijų poaibis reikia kombinuoti
multikriterinį dimensijų reitingavimą su paieškos strategija. Rekursyvus 
dimensijų eliminavimas yra dažnai naudojama paieškos strategija dimensijų
atrinkimui. Todėl yra kombinuojamas multikriterinis dimensijų reitingavimas ir
rekursyvus dimensijų eliminavimas.

Multikriterinis rekursyvus dimensijų eliminavimas\cite{yang2011robust} susideda iš dviejų
dalių: keletos dimensijų atrinkimo kriterijų suliejimo ir pagal svorius, ir 
pagal reitingus, ir rekursyvaus dimensijų eliminavimo aprašyto algoritme 
nr. \ref{RFE}. Algoritmas pavaizduotas ~\ref{fig:figure6} pav.
\begin{figure}
 \centering
 \includegraphics[width=0.7\textwidth]{images/mcf-rfe_procedure.pdf}
 \caption{Multikriterinio rekursyvaus dimensijų eliminavimo algoritmas.}
 \label{fig:figure6}
\end{figure}

Yra pastebėta, kad standartinis rekursyvus dimensijų eliminavimas, kai vienos
iteracijos metu yra eliminuojama viena dimensija, gali labai padidinti 
algoritmo sudėtingumą. Todėl genų ekspresijos duomenims yra rekomenduotina
eliminuoti keletą dimensijų vienu metu.

Nors SVM-RFE dimensijų atrinkimo algoritmas ir yra labai populiarus, tačiau yra
žinoma, kad jam trūksta stabilumo. Todėl kombinuodami didesnį stabilumą turintį
multikriterinį dimensijų atrinkimą su rekursyvaus dimensijų eliminavimo paieškos
strategija, turėtume gauti stabilesnį dimensijų atrinkimo algoritmą.

\subsection{Konsensuso grupėmis grįstas stabilių dimensijų atrinkimo metodas}

Konsensuso grupėmis grįstas stabilių dimensijų atrinkimo metodas\cite{loscalzo2009consensus}, pirma, identifikuoja panašių dimensijų grupes, antra, pagal surastas grupes transformuoja dimensijų erdvę, trečia, transformuotoje dimensijų erdvėje atlieka dimensijų atrinkimą. Schematiškai šis algoritmas pavaizduotas  ~\ref{fig:figure7} pav.

\begin{figure}
 \centering
 \includegraphics[width=\textwidth]{images/consensus_group_based_feature_selection_framework.pdf}
 \caption{Konsensuso grupėmis grįstas stabilių dimensijų atrinkimas.}
 \label{fig:figure7}
\end{figure}
Loscalzo pasiūlyto metodo pagrindinė dalis yra panašių dimensijų identifikavimas. Šio uždavinio sprendimui Loscalzo naudojo \textit{Dense Group Finder} (DGF) algoritmą. DGF aprašytas algoritme nr. \ref{DGF}.
\begin{algorithm}
\caption{DGF - \textit{Dense Group Finder}}
\label{DGF}
 \begin{algorithmic}
 \item \textbf{Įeitis:} duomenys $D=\{x_i\}_{i=1}^n$, branduolio plotis $h$
 \item \textbf{Išeitis:} tankios dimensijų grupės $G_1, G_1,..., G_L$
 \For{$i = 1$ \textbf{to} $n$ \do} 
  \State Inicializuojame $j=1, y_{i,j}=x_i$
  \Repeat
    \State Suskaičiuoti $y_{i, j+1}$ pagal (\ref{for_dgf})
  \Until{konverguoja}
  \State Nustatyti atskaitos tašką $y_{i,c} = y_{i,j+1}$ (Nustatyti piką $p_i$ kaip $y_{i,c}$)
  \State Sulieti piką $p_i$ su artimiausiais pikais jei atstumai tarp jų $ < h$
 \EndFor
 \item Iš kiekvieno unikalaus piko $p_r$, pridėkime $x_i$ į $G_r$ jei $||p_r - x_i|| < h$
 \end{algorithmic}
\end{algorithm}


\begin{equation}
\label{for_dgf}
  y_{i, j+1}=\frac{\sum_{i=1}^{n} x_i K(\frac{y_j - x_i}{h})}{\sum_{i=1}^{n} K(\frac{y_j - x_i}{h})} j=1,2,...
\end{equation}
kur

\begin{algorithm}
 \caption{Konsensuso grupėmis grįstas stabilių dimensijų atrinkimas}
 \label{CGS}
 \begin{algorithmic}
   \item \textbf{Įeitis:} mėginių aibė $D$, iteracijų skaičius $t$, dimensijų atrinkimo metodas $\Phi$\
   \item \textbf{Išeitis:} atrinktos konsensuso dimensijų grupės $CG_1, CG_1,..., CG_k$
   \item // Konsensuso grupių identifikavimas
   \For{$i = 1$ \textbf{to} $n$ \do}
    \State Parinkti mėginių  poaibį $D_i$ iš $D$
    \State Gauti tankių dimensijų grupes pagal $DGF(D_i, h)$
   \EndFor
   \For{kiekvienai dimensijų porai $X_i$ ir $X_j \in D$}
    \State Nustatyti $W_{i,j}=$ dažnis kai $X_i$ ir $X_j$ yra toje pačioje grupėje $/t$
   \EndFor
   \item Sudaryti konsensuso grupes $CG_1, CG_1,..., CG_L$ atliekant hierarchinį klasterizavimą visoms dimensijoms pagal $W_{i, j}$
   \item //Dimensijų atrinkimas grįstas konsensuso grupėmis
   \For{$i = 1$ \textbf{to} $l$ \do}
    \State Parinkti reprezentatyvią dimensiją $X_i$ iš $CG_i$
    \State Įvertinti dimensijos informatyvumą $\Phi(X_i)$
   \EndFor
   \item Reitinguoti konsensuso grupes $CG_1, CG_1,..., CG_L$ pagal $\Phi(X_i)$
   \item Pasirinkti $k$ dimensijų turinčių geriausią reitingą  
 \end{algorithmic}
\end{algorithm}


%\section{DIMENSIJŲ ATRINKIMO METODŲ SPARTA}

\section{DIMENSIJŲ ATRINKIMO STABILUMAS}
%Dimensijų atrinkimo technikų stabilumas gali būti apibrėžtas kaip dimensijų
atrinkimo rezultatų variacijos dėl mažų pakeitimo duomenų rinkinyje. Pekeitimai
duomenų rinkinyje gali būti duomenų objektų lygio (pvz. pridedami ar atimami
duomenų objektai), dimensijų lygio (pvz. pridedant dimensijoms triukšmo) ar
abiejų lygių kombinacija.

Dimensijų atrinkimo technikų stabilumas yra vis didesnę svarbą įgaunanti tyrimų
kryptis. Stabilumo aktualumas yra sąlygotas to, kad biologiniuose duomenyse
galima gana užtikrintai daryti prielaidą, kad konkrečiai problemai yra
aktualios tik tam tikros dimensijos. Todėl dalykinės srities ekspertams yra
aktualu naudoti tik tuos dimensijų atrinkimo metodus, kurie yra stabilūs ir 
relevantiški modeliuojamai problemai, nes tai atpigina tolimesnę duomenų
analizę. 

Šiame skyriuje apžvelgsime teorinį stabilumo matavimų modelį.
Taip pat įvertinsime pavienių dimensijų atrinkimo metodų stabilumą
taikant juos įvairiems duomenų rinkiniams. Taip pat išanalizuosime situaciją, kai
kombinuojami keletos dimensijų atrinkimo metodų rezultatai.

\subsection{Stabilumo matavimas}

Vertinant dimensijų atrinkimo metodų stabilumą yra svarbu kaip panašiai yra
atrenkamos dimensijos, kai yra atliekamas dimensijų atrinkimas su vis kitu 
duomenų poaibiu. Kuo panašesnius dimensijų atrinkimo rezultatus gauname, tuo stabilumas
yra didesnis. Vidutinis stabilumas gali būti apibrėžtas kaip vidurkis visų 
reitingavimo metu gautų sąrašų porų tarpusavio panašumo įverčių:

\begin{equation}
 S_{tot}=\frac{2\sum_{i=1}^{k-1}\sum_{j=i+1}^{k} S(f_i, f_j)}{k*(k-1)},
\end{equation} 
kur $k$ žymi kiek kartų buvo imtas skirtingas poaibis objektų dimensijų atrinkimui,
$f_i$, $f_j$ - dimensijų atrinkimo rezultatas - reitingai, $S(f_i, f_j)$ - yra kokia 
nors panašumo matavimo funkcija.

Kaip matome dimensijų atrinkimo stabilumas priklauso nuo to, kokią panašumo 
funkciją naudosime. Tradicinės panašumo funkcijos (persidengimo procentas, 
Pearson'o koreliacija, Spearman'o koreliacijoa, Jaccard indeksas) 
gali būti taikomos, bet jos yra linkusios priskirti didesnes panašumo
reikšmes, kai pasirenkamas didesnis dimensijų poaibis. Taip yra dėl padidėjusio 
sisteminio nuokrypio (ang. bias), nes imant didesnį poaibį padidėja tikimybė
tiesiog atsitiktinai pasirinkti dimensiją. 
Kad išventume šios problemos panašumui vertinti buvo pasirinktas Kunchevos
\cite{DBLP:conf/aia/Kuncheva07} indeksas:
\begin{equation}
 KI(f_i, f_j)=\frac{r*N - s^2}{s*(N-s)}=\frac{r - (s^2/N)}{s - (s^2/N)},
\end{equation}		
kur $s=|f_i|=|f_j|$ yra atrinktų dimensijų aibės dydis, $r=|f_i \bigcap f_j|$ -
abiems atrinktiems dimensijų poaibiams bendrų dimensijų skaičius, $N$ - bendras
 duomenų aibės
dimensijų skaičius. Pastebėtina, kad formulėje esantis atėminys $s^2/N$ ištaiso 
sisteminį nuokrypį atsirandantį dėl galimybės atsitiktinai pasirinkti dimensijas.
Kunchevos indeksas gali įgyti reikšmes iš intervalo
$[-1, 1]$, kur didesnė reikšmė reiškia didesnį panašumą, o artimos nuliui 
reikšmės reiškia, kad dimensijos atrenkamos daugiausia atsitiktinai. Kunchevos 
indekso ypatybė yra ta, kad jis atsižvelgia tik į
persidengiančias, tačiau visiškai nekreipia dėmesio į koreliuojančias dimensijas.

Vertinant stabilumą tarp skirtingų metodų gali iškilti problemų, nes ne visi 
dimensijų atrinkimo metodai gražina rezultatą tokiu pačiu formatu. Šiame darbe
dimensijų atrinkimo metodų rezultatas yra ne dimensijai priskirtas svoris, bet 
dimensijos reitingas. Todėl $f_i$ yra sąrašas, kurio ilgis yra $N$, kur pirmas
sąrašo elementas yra geriausią reitingą turinčios dimensijos numeris, o 
paskutinis sąrašo elementas yra blogiausią reitingą turinčios dimensijos numeris.

Galiausiai yra svarbu paminėti, kad dimensijų stabilumas nėra matuojamas
visiškai nepriklausomai - jis yra matuojamas atsižvelgiant į klasifikavimo
rezultatus. Dimensijų atrinkimo metodų stabilumas yra matuojamas tik tada
kai atrinktos dimensijos duoda gerus klasifikavimo rezultatus. Taip yra, nes 
kokios nors dalykinės srities ekspertui, nėra naudingos tos dimensijų atrinkimo
strategijos, kurios duoda labai stabilius rezultatus, bet nėra naudingos
klasifikavimo modelio kūrime.

\subsection{Pavienių dimensijų atrinkimo metodų stabilumas}

Šiame skyriuje apžvelgsime pavienių dimensijų atrinkimo motodų stabilumą. 
Stabilumas visiems metodams buvo matuojamas atsižvelgiant į klasifikavimo
rezultatus - buvo matuojamas stabilumas tikra

\subsubsection{Fisher'io dimensijų atrinkimo metodas}

\subsubsection{Atpalaidavimo dimensijų atrinkimo metodas}

Šis metodas yra vienas nestabiliausių, nes pasikeitimai duomenų rinkinyje 
stipriai įtakoja rezultatus.

\addcontentsline{toc}{section}{LITERATŪRA} 
\bibliographystyle{alpha}
\bibliography{literatura.bib}

\end{document}