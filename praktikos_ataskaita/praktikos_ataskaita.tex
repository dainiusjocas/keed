\documentclass[12pt, a4paper]{article}

\usepackage[utf8x]{inputenc}
\usepackage[lithuanian]{babel}
\usepackage[L7x]{fontenc}
\usepackage{lmodern}
\usepackage{verbatim}
\usepackage{tocloft} 
\usepackage{url}
\usepackage{setspace}
\usepackage{caption}
\captionsetup{margin=10pt,font=small,labelfont=bf}
\usepackage{parskip}
\usepackage{graphicx}
\usepackage{fixltx2e} % šitas dėl \textsubscript{}
\usepackage{color} % spalvinsim tekstą
\usepackage{mathtools} % pades rašyti matematinius intarpus
\usepackage{titlesec}
\usepackage{indentfirst} % atitraukta pirmoji eilute
\usepackage{longtable} % darysim didelias lenteles
\usepackage{algorithm}
\usepackage{algpseudocode}
\usepackage{lipsum}% http://ctan.org/pkg/lipsum
\usepackage[top=2cm, bottom=2cm, left=3cm, right=1.5cm]{geometry}
\usepackage{changepage}
\usepackage{pdfpages} % su šitua paketu galima įkelti pdf kaip puslapius
\linespread{1.5}
\parindent = 1cm
\titlelabel{\thetitle.\quad} % Padeda tašką po skyriaus numerio
\renewcommand{\cftsecleader}{\cftdotfill{\cftdotsep}}

\begin{document}
% Titulinis puslapis
\includepdf{images/praktikos_ataskaitos_titulinis.pdf}

\let \savenumberline \numberline
\def \numberline#1{\savenumberline{#1.}}
\tableofcontents 
\newpage

%% Įvade aprašomi darbo tikslai, nurodomas temos aktualumas, aptariamos teorinės darbo prielaidos bei metodika, apibrėžiamas tiriamasis objektas, apibūdinami su tema susiję literatūros ar kitokie šaltiniai, temos analizės tvarka, darbo atlikimo aplinkybės, pateikiama žinių apie naudojamus instrumentus (programas ir kt.). Darbo įvadas neturi būti dėstymo santrauka. Įvado apimtis 3–4 puslapiai.

\newpage
\section*{ĮVADAS}

%% JG: Idėja - naudok terminus "matai" ir "mėginiai". Kiekvienam mėginiui atliekama labai daug matavimų, iš to ir "daugiamatiškumas"

% JG: Paminėk, kad šiame darbe mes gilinamės į biomedicinoje kaupiamų genetinių duomenų analizės specifika. Šie duomenys ypatingi dėl daugiamatiškumo, mažo mėginių kiekio, triukšmingų matavimų.

Nuolat vystosi technologijos skirtos gauti biomedicininius duomenis, pvz. genomo sekvenavimas\cite{pettersson2009generations}, o tai reiškia, kad didėja gaunamų duomenų detalumas. Detalumas reiškia, kad daugėja biomedicininius duomenis abibūdinančių faktorių arba matų skaičius. Duomenys, kurių kiekvienas mėginys aprašomas dideliu skaičiumi matų, yra vadinami daugiamačiais duomenimis.

Šiame darbe yra nagrinėjama biomedicinoje kaupiamų genetinių daugiamačių duomenų analizės specifika. Šie duomenys yra ypatingi tuo, kad jie įprastai turi šimtus kartų daugiau matų nei mėginių. Santykinai mažas mėginių skaičius turimas, nes mėginio gavimo kaina yra aukšta. Biomedicininių duomenų analizę apsunkina ir tai, kad matavimai, kuriais tie duomenys gaunami, įneša atsitiktinių duomenų - triukšmo. Triukšmas matavimo metu gali atsirasti dėl įvairių priežasčių, pvz. netinkamai paruoštų cheminių preparatų. Kai duomenys yra triukšmingi, didėja tikimybė duomenyse rasti atsitiktinių priklausomybių. Tai yra viena priežasčių, kodėl biomedicininių duomenų analizės procesas yra sudėtingas.

%% JG: Būdai neatsiranda, o vystosi technologija. Jie nėra tikslesni, bet detalesni, t.y. kiekvienam mėginiui atliekama daugiau matavimų. 

%% JG: Nors matavimų kiekis didėja, mėginio kaina išlieka gana aukšta. Todėl biomedicinos eksperimentuose gaunami duomenys ypatingi tuo, kad matų visados ženkliai daugiau nei mėginių.

Klasifikavimu\cite{fisher1936use} yra vadinamas duomenų analizės procesas, kai duomenys suskirstomi į grupes pagal tam tikrus jų požymius. Algoritmai arba funkcijos, kurios turimus duomenis priskiria iš anksto žinomoms grupėms - atlieka klasifikavimą - yra vadinami klasifikatoriais. Klasifikatoriai paruošiami naudojant turimus mėginius - treniravimosi duomenis - ir informaciją apie jų būklę (sveikas ar sergantis). Klasifikatoriaus ruošimo procesas yra vadinamas apmokymu. Apmokyti klasifikatoriai paprastai naudojami nustatant naujų, dar nematytų, mėginių - testavimo duomenų - būklę. Pagal tai, kokią dalį visų testavimo mėginių klasifikatorius priskiria neteisingai klasei, yra nustatomas klasifikatoriaus tikslumas. 

Biomedicininių duomenų tipinė klasifikavimo užduotis yra atskirti sveikų pacientų mėginius nuo sergančiųjų. Klasifikavimu siekiama nustatyti, kurie matai veikdami drauge, geriausiai paaiškina skirtumą tarp ligos paveiktų ir sveikų mėginių. Labiausiai ligą paaiškinančių matų nustatymas galėtų palengvinti tiriamų ligų diagnozės ar gydymo metodų kūrimą.

Biomediciniuose duomenyse dažniausiai turime tik kelias dešimtis mėginių, todėl norint geriau įvertinti klasifikatoriaus tikslumą yra naudojami pakartotinio mėginių poaibio atrinkimo (angl. \textit{resampling}) metodai: kryžminio patikrinimo (angl. \textit{cross-validation}) arba įkelčių (angl. \textit{bootstrap\footnote{Terminas \textit{bootstrap} ,,įkelties`` prasme pradėtas naudoti dar Rudolfo Ericho Raspės knygoje ,,Barono Miunchauzeno nuotykiai``, kurioje Baronas Minchauzenas užkėlė save ant arklio tempdamas į viršų savo batų raištelius (angl. \textit{bootstraps}).}}). Šių metodų naudojimas naudojimas su duomenimis, kurių tikrasis pasiskirstymas nėra žinomas, padeda įvertinti klasifikavimo variabilumą (angl. \textit{variance}) ir sisteminį nuokrypį (angl. \textit{bias}).

Naudojant kryžminio patikrinimo metodą, daug kartų sudaromos skirtingos treniravimosi ir testinės mėginių imtys. Taikant atskirą šio metodo variantą, kryžminį patikrinimą išbraukiant po vieną mėginį (angl. \textit{leave-one-out cross-validation}), iš treniravimosi imties išbraukiamas vienas mėginys ir apmokomas klasifikatorius, kuris klasifikuoja išbrauktąjį mėginį. Procesas tęsiamas tol, kol suklasifikuojami visi objektai. Kitais kryžminio patikrinimo metodo variantais iš treniravimosi mėginių yra išmetama po keletą mėginių. Pagal tai, kiek testinių mėginių klasifikatorius priskyrė klaidingai kategorijai, yra nustatoma vidutinė klaidingo klasifikavimo tikimybė. Šiuo metodu gauti įverčiai pasižymi didele dispersija \cite{braga2004cross}.

Naudojant įkelčių metodą, iš $n$ dydžio mėginių aibės yra paimama tokio pačio dydžio atsitiktinių mėginių imtis su pasikartojimais, kuri vadinama įkelties treniravimosi imtimi. Į šią imtį nepaimti mėginiai yra priskiriami testavimo imčiai. Naudojant įkelties treniravimosi mėginių imtį yra apmokomas klasifikatorius, kuris klasifikuoja testavimo imtį. Procesą kartojant gaunama klaidingo klasifikavimo tikimybės įverčių imtis. Šios imties vidurkis yra klaidingo klasifikavimo tikimybės įvertis. Dažniausiai naudojamas ,“0.623 įkelčių`` (angl. \textit{0.623\footnote{0.623 yra tikimybė mėginiui būti įtrauktam į treniravimosi imtį.} bootstrap}) įverčiu. Šiuo metodu gautas klaidingo klasifikavimo tikimybės įverti pasižymi maža dispersija \cite{michie1994machine}.

Biomedicininių duomenų kontekste galima daryti prielaidą, kad ne visi matai yra susiję su tiriama problema, pvz. gaubtinės žarnos vėžiu, dėl tokių faktorių, kaip triukšmas duomenyse. Paprastai nagrinėjamai problemai svarbus yra mažas, palyginus su visu, matų kiekis. Ši biomedicininių duomenų ypatybė veda prie ,,daugiamatiškumo prakeiksmo`` (angl. \textit{the curse of dimentionality})\cite{bellman1966adaptive} - didėjant matų
kiekiui mėginiai pasidaro panašūs, o bandymas juos klasifikuoti tolygus spėliojimui. Todėl biomedicininių duomenų daugiamatiškumui sumažinti yra naudojami informatyviausių dimensijų atrinkimo metodai\cite{guyon2003introduction} (angl. \textit{feature selection}). Pagal tai, kaip susiję su klasifikatoriumi, matų atrinkimo metodai skirstomi į tris kategorijas\cite{saeys2008robust}: filtruojantys (angl. \textit{filter}), 
prisitaikantys (angl. \textit{wrapper}) ir įterptiniai (angl. \textit{embedded}) metodai. Dimensijų atrinkimas yra svarbi biomedicininių duomenų apdorojimo (angl. \textit{preprocessing}) etapo dalis. Naudojant dimensijų atrinkimo metodus galima kovoti su ,,daugiamatiškumo prakeiksmu`` dimensijų skaičių priartinant prie mėginių skaičiaus.

%% JG: Kodėl klasifikuojama? Norima nustatyti, kokie matai, veikdami drauge, geriausiai paaiškina skirtumą tarp ligos paveiktų ir sveikų mėginių.

%% JG: Kokios yra tradicinės klasifikavimo strategijos ir kodėl jos neveikia daugiamačių duomenų atveju? Skaičiavimo laikas nėra problema - Random forests veikia visai neblogai tokiais atvejais. Daugiamatiškumas veda prie the curse fo dimensionality.

% JG: sumažinus naudojamų matų kiekį, matų kiekis priartėja prie mėginių kiekio ir tokiu būdu apeinamas daugiamatiškumo prakeiksmo problema. Biomedicinos duomenų kontekste, galima daryti prielaidą, kad dauguma matų yra beprasmiai, pvz., tik kai kurie genai įtakoja ligą, todėl matų mažinimimas yra prasmingas. Taip pat, kuriant medicininius diagnostikos įrankius, naudojamų matų kiekis įtakoja įrankio kainą. Todėl pageidautina turėti kuo mažiau matų.

Kadangi biomedicininiuose duomenyse reikšmingų dimensijų kiekis tiriamai problemai yra nedidelis, todėl norima žinoti, kuris dimensijų poaibis yra svarbus tai problemai. Tokioje situacijoje tampa svarbu, kaip varijuoja atrenkamų dimensijų aibė, kai dimensijų atrinkimas vykdomas su vis kitu mėginių poaibiu. Dimensijos, kurios keičiant mėginių, kurie naudojami dimensijų atrinkime, poaibį, yra vėl ir vėl atrenkamos yra vadinamos stabiliomis dimensijomis \cite{needcitation}. Tačiau skirtingi dimensijų atrinkimo metodai tiems patiems mėginiams gali atrinkti skirtingas dimensijas. Taip pat, suskaidžius duomenis į persidengiančius poaibius ir atrinkus tą patį kiekį dimensijų tuo pačiu metodu, gaunami skirtingi dimensijų poaibis. Tačiau, norint geriau suprasti biomedicininius duomenis, itin svarbu fokusuoti dėmesį į sąlyginai nedidelį dimensijų poaibį. Dimensijų aibės sumažinimas paspartina biomedicininių duomenų tyrimus - tyrėjams reikia atlikinėti bandymus su mažesniu mėginių skaičiumi. Mažesnio skaičiaus mėginių tyrimas kainuoja mažiau, nes mažiau reikia žmonių darbo laiko, mažiau reikia ir cheminių reagentų. Todėl stabilių dimensijų atrinkimas dirbant su biomedicininiais duomenimis yra \textit{(angl. robustness)}.

% JG: Žiūrėkim į stabilumą kaip į šalutinį matų atrinkimo efektą, kurį svarbu pažaboti. Stabilumas svarbus, nes, analizuojant duomenis norima ne tik nustatyti, koks būtų vidutinis klasifikatoriaus tikslumas, bet ir sukurti tą vidutinį klasifikatorių. Norint pastarąjį sukurti, reikia žinoti, kuriuos konkrečius matus naudoti. 

% JG: Šitam paragrafe suplaki daug svarbių dalykų į krūvą ir juos turėtum būti paaiškinęs jau anksčiau. Mėginių trūkumas nėra tavo sprendžiama problema. Tačiau, kuo mažiau duomenų, tuo nestabilesni atrenkami matai. 

%Taigi, dirbant su daugiamačiais duomenimis, reikia atsižvelgti į keletą kriterijų:
%\begin{enumerate}
% \item Klasifikavimo tikslumą;
% \item Dimensijų atrinkimo stabilumą, atsižvelgiant į klasifikavimo rezultatus;
% \item Triukšmo lygį duomenyse;
% \item Skaičiavimo išteklių naudojimo racionalumą.
%\end{enumerate}
%Reikalavimas vienu metu atsižvelgti į keletą kriterijų užduotį daro sudėtinga. Klasifikuojant daugiamačius duomenis uždavinys yra surasti geriausius rezultatus duodančią strategiją, kuri geriausiai atsižvelgia į minėtus kriterijus.

%Darbo eksperimentinei daliai reikalingus skaičiavimo išteklius, suteikė VU MIF skaitmeninių tyrimų ir skaičiavimų centras \cite{mif2012stsc}. Eksperimentuose buvo naudojami laisvai internete prieinami biomedicininių duomenų rinkiniai (angl. \textit{datasets}). Biomedicininių duomenų apdorojimo algoritmų implementavimui buvo naudojama R \cite{r2012statistics} programavimo kalba. Eksperimentai atlikti profesinės praktikos MII metu.

Dimensijų atrinkimo stabilumo problemą Yang ir Mao \cite{yang2011robust} siūlė spręsti reitinguojant dimensijas remiantis keletos dimensijų atrinkimo metodų rezultatais. Galutinis dimensijų reitingų sąrašas gaunamas, kai po kiekvieno dimensijų atrinkimo yra išmetama viena žemiausią reitingą turinti dimensija iš dimensijų aibės, ir dimensijų atrinkimas yra kartojamas tol, kol nebelieka dimensijų. Tačiau dimensijų atrinkimo metodų kiekis yra ribotas ir skirtingų metodų dažnai negalima atlikti paraleliai. Tai riboja šio metodo pritaikomumą daugiamačių duomenų analizėje.

Dimensijų atrinkimo stabilumo problemą siūlyta spręsti surandant dimensijų grupių tankio centrus ir naudoti dimensijas, kurios artimiausios tiems centrams \cite{yu2008stable}. Pasiūlytas grupių tankių algoritmas užtrunka $O(\lambda n^2m)$ laiko, kur n yra dimensijų kiekis, o m - mėginių skaičius. Vėliau Loscalzo ir kt. pasiūlė mokymo duomenis skaidyti poaibiais ir kiekviename poaibyje ieškoti tankių grupių, o tada imti sprendimą balsavimo principu \cite{loscalzo2009consensus}. Nors šie metodai siūlo stabilų dimensijų atrinkimą, tačiau šių metodų panaudojamumą daugiamačiuose duomenyse riboja skaičiavimo sudėtingumas.

Šiame bakalauriniame darbe remiantis Yang, Mao bei Loscalzo darbuose pateiktomis įžvalgomis, bus stengiamasi pasiūlyti tyrimų kryptis, kurios galėtų padėtų sukurti metodus, skirtus spręsti stabilių dimensijų atrinkimo problemą. Idėja yra sugrupuoti dimensijas pagal greitą klasterizacijos algoritmą, išrinkti reprezentatyviausias dimensijas, transformuoti dimensijų erdvę ir joje vykdyti dimensijų atrinkimą remiantis keletu dimensijų atrinkimo metodų.

Šio darbo tikslas yra išanalizuoti darbo su daugiamačiais duomenis ypatybes. Šiam darbui yra keliamos tokios užduotys:
\begin{enumerate}
 \item Susipažinti su naujausiais klasifikavimo ir dimensijų atrinkimo metodais;
 \item Atlikti dimensijų atrinkimo metodų palyginimo eksperimentus;
 \item Pasiūlyti kryptis, kaip dabartiniai metodai gali būti patobulinti ir paruošti naujųjų metodų prototipus.
\end{enumerate}

Tolimesnė bakalaurinio darbo struktūra yra tokia: skyriuje


\newpage

% 2.	Įmonės/įstaigos apibūdinimas. Glaustai aprašoma įmonė/įstaiga, kurioje buvo
% atliekta praktika: jos veiklos sritis, organizacinė struktūra, teikiamos 
% paslaugos ir kt. Apibūdinamos praktikos vietoje sudarytos darbo sąlygos
% (1-2 psl.).

\section{ĮSTAIGOS APIBŪDINIMAS}
\label{imones_apibudinimas}

Vilniaus universiteto matematikos ir informatikos institutas (MII) nuo 2010 m. yra Vilniaus universiteto padalinys užsiimantis tyrimais matematikos ir informatikos srityse. Instituto įkūrimo data laikoma 1965 m. spalio 1d., kai buvo panaikintas Lietuvos mokslų akademijos Fizikos ir technikos institutas ir įkurti trys nauji institutai, tarp kurių buvo Fizikos ir matematikos institutas, kuris laikomas MII pirmtaku. 

Pagrindinė instituto veikla - moksliniai tyrimai ir eksperimentinė plėtra. Kitos veiklos sritys yra: mokslininkų ugdymas (doktorantūros studijos) (MII suteikta teisė ruošti matematikos, informatikos ir informatikos inžinerijos sričių mokslininkus); mokslo organizacinė veikla - konferencijos, seminarai, parodos, mokslinių knygų redagavimas; leidyba; mokymas, moksleivių ugdymas, švietimas. Mokslinė veikla sukoncentruota 12-oje mokslinių padalinių. Institute yra 5 matematikos krypties padaliniai, 7 informatikos bei informatikos inžinerijos padaliniai:
\begin{enumerate}
  \item Atpažinimo procesų skyrius;
  \item Atsitiktinių procesų skyrius;
  \item Informatikos metodologijos skyrius;
  \item Kompiuterinių tinklų laboratorija;
  \item Matematinės logikos sektorius;
  \item Programų sistemų inžinerijos skyrius;
  \item Sistemų analizės skyrius (SAS);
  \item SAS optimizavimo sektorius;
  \item SAS operacijų tyrimo sektorius;
  \item Skaičiavimo metodų skyrius (SMS);
  \item SMS diferencialinių lygčių sektorius;
  \item Tikimybių teorijos ir statistikos skyrius;
\end{enumerate}

MII organizuoja moksleivių ugdymą: veikia jaunųjų programuotojų neakivaizdinė mokykla, rengiamos lietuvos moksleivių informatikos ir matematikos olimpiados, rengiamas informacinių technologijų konkursas ,,Bebras``. MII yra vienas iš Lietuvos jaunųjų matematikų mokyklos steigėjų, jaunųjų matematikų konkurso ,,Kengūra`` rengėjas. Taip pat MII prisideda prie kompiuterijos naudotojų švietimo ir mokymo: dirba informatikos terminijos komisija, multimedijos centras humanitarams, palaikomas tinklalapis apie lietuviškų rašmenų naudojimą elektroninio pašto laiškuose.

MII leidybos skyrius atsakingas už visą eilę recenzuojamų periodinių leidinių: ,,Informatica``, ,,Informatics in Education``, ,,Lithuanian Mathematical Journal``, ,,Lietuvos matematikos rinkinys. LMD darbai``, ,,Mathematical Modelling and Analysis``, ,,Nonlinear Analysis. Modelling and Control``, ,,Olympiads in Informatics``. MII taip pat yra išleidusi mokslinių bei mokslo populiarinimo knygų lietuvių ir anglų kalbomis, mokymo priemonių, interaktyvių kompaktinių diskų bei sukūrusi įvairių internetinių informacinių sistemų (pvz. enciklopedinis kompiuterijos terminų žodynas).

MII man, kaip ir kiekvienam darbuotojui, parūpino: darbo vietą, prisijungimo prie vietinio tinklo, galimybe naudotis skaičiavimo ištekliais, galimybe su nuolaida pietauti vietinėje valgykloje. MII darbuotojai buvo kolegiški, todėl apsipratimas MII įvyko labai greitai. Todėl jau nuo pat pirmosios profesinės praktikos dienos galėjau pradėti spręsti užsibrėžtus uždavinius.

\newpage

%% Praktikos veiklos aprašymas (vienas arba keli skyriai). Aprašomas praktikos užduoties įgyvendinimas (pvz., atlikti projektavimo ir/ar programavimo darbai, sukurtas modelis, priimti sprendimai ir pan.).

\section{PROFESINĖS PRAKTIKOS VEIKLA}
\label{praktikos_veiklos_aprasymas}

Profesinę praktiką sudarė trys užduotys:
\begin{enumerate}
 \item Susipažinti su matų atrinkimo daugiamačiuose duomenyse problematika bei moksline literatūra;
 \item Suprogramuoti matų atrinkimo metodus;
 \item Ištirti matų atrinkimo metodų savybes.
\end{enumerate}
Toliau aprašau kiekvieną užduotį atskirai.

\subsection{Matų atrinkimas daugiamačių duomenų klasifikavimui}

Savo bakalauriniame darbe yra nagrinėju biomedicinoje kaupiamų genetinių daugiamačių duomenų analizės specifiką. Šie duomenys yra specifiški tuo, kad jie turi šimtus kartų daugiau matų nei mėginių. Kadangi mėginio gavimo kaina yra aukšta, turimas mažas mėginių skaičius turimas. Biomedicininių duomenų analizę apsunkina ir tai, kad matavimai, kuriais tie duomenys gaunami, yra triukšmingi. Triukšmas matavimo metu atsiranda dėl cheminių reakcijų netikslumo, tiriamo organizmo sudėtingumo. Kai duomenys yra triukšmingi ir didėja juos apibūdinančių matų skaičius, didėja tikimybė duomenyse rasti atsitiktinių priklausomybių. Tai yra pagrindinė priežastis, kodėl biomedicininių duomenų analizės procesas yra sudėtingas.

Biomedicininių duomenų klasifikavimo užduotis yra atskirti sveikų pacientų mėginius nuo sergančiųjų. Klasifikavimu siekiama nustatyti, kurie matai, veikdami drauge, geriausiai paaiškina skirtumą tarp ligos paveiktų ir sveikų mėginių. Labiausiai ligą paaiškinančių matų nustatymas galėtų palengvinti tiriamų ligų diagnozės ir gydymo metodų kūrimą. Klasifikavimu yra vadinamas duomenų analizės procesas, kurio metu yra sukonstruojama funkcija, atskirianti duomenis į grupes pagal jų matus \cite{fisher1936use}. Sukonstruotos funkcijos yra vadinamos klasifikatoriais, o jų konstravimo algoritmai -- klasifikavimo algoritmais. Klasifikatoriai paruošiami naudojant turimus mėginius -- treniravimosi duomenis -- ir informaciją apie jų būklę (sveikas ar sergantis). Klasifikatoriaus ruošimo procesas yra vadinamas apmokymu. Klasifikatoriai paprastai naudojami nustatant naujų, dar nematytų, mėginių būklę. 

Dėl ,,daugiamatiškumo prakeiksmo`` (angl. \textit{the curse of dimentionality}) -- didėjant matų kiekiui mėginiai pasidaro panašūs, todėl bandymas juos klasifikuoti tolygus spėliojimui \cite{bellman1966adaptive}. Biomedicininių duomenų kontekste galima daryti prielaidą, kad ne visi matai yra susiję su tiriama problema, pvz. gaubtinės žarnos vėžiu, dėl to, kad duomenys yra daugiamačiai. Paprastai nagrinėjamai problemai svarbus yra mažas, palyginus su visu, matų kiekis.  Todėl biomedicininių duomenų daugiamatiškumui sumažinti yra naudojami informatyviausių matų atrinkimo metodai \cite{guyon2003introduction} (angl. \textit{feature selection}). Pagal tai, kaip susiję su klasifikatoriumi, matų atrinkimo metodai skirstomi į tris kategorijas \cite{saeys2008robust}: filtruojantys (angl. \textit{filter}), prisitaikantys (angl. \textit{wrapper}) ir įterptiniai (angl. \textit{embedded}) metodai. Filtruojančiais metodais pirmiausia yra atrenkamos informatyviausi matai, o tada apmokomas klasifikatorius. Prisitaikančiųjų metodų atveju, pirma, apmokomas klasifikatorius su visais matais, antra, parenkamas matų poaibis ir apmokomas klasifikatorius, tada po daugkartinio matų aibių įvertinimo pagal klasifikavimo rezultatus yra nusprendžiama, kuris matų poaibis yra labiausiai tinkamas klasifikavimui. Įterptinių metodų atveju matų atrinkimo procesas yra neatsiejamas nuo klasifikavimo proceso -- pats klasifikatorius įvertina matus.

Matų atrinkimas yra svarbi biomedicininių duomenų apdorojimo (angl. \textit{preprocessing}) etapo dalis. Naudojant matų atrinkimo metodus, galima kovoti su daugiamatiškumo prakeiksmu matų skaičių priartinant prie mėginių skaičiaus. Todėl svarbu yra pasirinkti geriausiai tinkančią matų atrinkimo strategiją. Kadangi ir pačių matų atrinkimo metodų veikimas priklauso nuo konkrečių duomenų, tai metodo pasirinkimas yra sudėtinga užduotis.

Naudodami matų atrinkimo metodus, biomedicininius duomenis tiriantys mokslininkai susiduria su atrinktųjų matų aibės stabilumo problema -- atrenkant matus pagal kitą mėginių poaibį, gaunamas kitas matų poaibis. Matų atrinkimo nestabilumas yra sąlygotas šių veiksnių:
\begin{enumerate}
 \item Duomenys yra triukšmingi ir kai kurie matai gali būti palaikyti informatyviais vien dėl atsitiktinių priežasčių;
 \item Daugiamačiuose duomenyse tikėtina, kad dalis matų koreliuoja tarpusavyje, todėl, kuris iš koreliuojančių matų bus pasirinktas, priklauso nuo to, kuriuos mėginius pasirinksime klasifikatoriaus apmokymui;
 \item Kiekvienas matų atrinkimo algoritmas daro skirtingas prielaidas apie tai, kurie matai yra informatyvūs.
\end{enumerate}
Galime daryti išvadas, kad skirtingi metodai tiems patiems duomenims gali atrinkti skirtingus matus. Taip pat, suskaidžius turimus duomenis į atskiras persidengiančias aibes ir atrinkus tą patį kiekį matų tuo pačiu metodu, gaunamos skirtingos matų aibės. Be to, kuo triukšmingesni duomenys, kuo mažiau turima mėginių ir kuo daugiau yra matų, tuo ryškesnė yra ši problema \cite{loscalzo2009consensus}.

Matų atrinkimo stabilumo problemą pirma siūlyta spręsti surandant matų grupių tankio centrus ir naudoti matus, kurie artimiausi tiems centrams \cite{yu2008stable}. Pasiūlytas grupių tankių algoritmas užtrunka $O(\lambda n^2m)$ laiko, kur $n$ yra matų kiekis, o $m$ -- mėginių skaičius. Vėliau Loscalzo ir kt. pasiūlė mokymo duomenis skaidyti poaibiais ir kiekviename poaibyje ieškoti tankių matų grupių, o tada imti sprendimą balsavimo principu \cite{loscalzo2009consensus}. Nors šie metodai siūlo stabilų matų atrinkimą, tačiau jų panaudojamumą daugiamačiuose duomenyse riboja skaičiavimo sudėtingumas.

Yang ir Mao pasiūlė reitinguoti matus remiantis keletos matų atrinkimo metodų rezultatais \cite{yang2011robust}. Galutinis matų reitingų sąrašas gaunamas, kai po kiekvieno matų atrinkimo yra išmetama vienas žemiausią reitingą turintis matas iš matų aibės, ir matų atrinkimas yra kartojamas tol, kol nebelieka matų. Tačiau ši matų atrinkimo strategija yra ribota, nes matų atrinkimo metodų kiekis yra ribotas ir skirtingų metodų dažnai negalima atlikti išskirstytų skaičiavimų aplinkoje. Tai riboja šio metodo pritaikomumą daugiamačių duomenų analizėje.

Praktikos metų išstudijavau esamus stabilių matų atrinkimo metodus nustačiau, kad jie tik šiek tiek padidina matų atrinkimo stabilumą, bet problemos iš esmės neišsprendžia.

\subsection{Suprogramuoti matų atrinkimo algoritmai}

Profesinės praktikos metu suprogramavau populiariausius matų atrinkimo metodus. Taip pat programavau ir matų atrinkimo stabilumą didinančius metodus. Toliau šiame poskyryje aprašau šiuos metodus.

\subsubsection{\textit{Fisher} įvertis}

\textit{Fisher} įvertis vertina individualius matus pagal matų klasių atskiriamąją galią. Mato įvertis yra sudarytas iš tarpklasinio skirtumo santykio su vidiniu klasės pasiskirstymu:
\begin{equation}
 FR(j) = \frac{(\mu_{j1} - \mu_{j2})^2}{\sigma_{j1}^2 + \sigma_{j2}^2},
\end{equation}
kur, 
$j$ -- yra mato indeksas, 
$\mu_{jc}$ -- mato $j$ reikšmių vidurkis klasėje $c$, 
$\sigma_{jc}^2$ -- mato $j$ reikšmių standartinis nuokrypis klasėje $c$, kur $c={1,2}$. Kuo didesnis yra \textit{Fisher} įvertis, tuo geriau ts matas atskiria klases. Nors ir paprastas, šis metodas neįvertina matų tarpusavio sąveikų.

\subsubsection{\textit{Relief} metodas}

\textit{Relief} metodas iteratyviai skaičiuoja matų ,,susietumą``. Pradžioje ,,susietumas`` visiems matams yra lygus nuliui. Kiekvienoje
iteracijoje atsitiktinai pasirenkamas mėginys iš mėginių aibės, surandami artimiausi kaimynai iš tos pačios ir kitos klasės, ir atnaujinamos visų 
matų ,,susietumo`` reikšmės. Dėl atsitiktinumo faktoriaus klasifikavimo ir  matų atrinkimo stabilumo rezultatai naudojant šį metodą varijuoja. Mato įvertis yra vidurkis visų objektų atstumų skirtumų iki artimiausių kaimynų iš kitos ir tos pačios klasių:
\begin{equation}
 W(j)=W(j) - \frac{diff(j, x, x_H)}{n} + \frac{diff(i, x, x_M)}{n},
\end{equation}
kur 
$W(j)$ -- $j$-ojo mato ,,susietumo`` įvertis, 
$n$ -- mėginių aibės dydis, 
$x$ -- atsitiktinai pasirinktas mėginys, 
$x_H$ - artimiausias $x$ kaimynas iš tos pačios klasės (angl. \textit{nearest-Hit}), 
$x_M$ -- artimiausias $x$ kaimynas iš kitos klasės(angl. \textit{nearest-Miss}),
$diff(j, x, x')$ -- $j$-ojo mato reikšmių skirtumas tarp laisvai pasirinkto objekto $x$ ir atitinkamo kaimyno, kur skirtumą į intervalą $[0, 1]$ normalizuojanti funkcija yra:
\begin{equation}
 diff(j, x, x')=\frac{|x_j- x_j'|}{x_{j_{max}} - x_{i_{min}}},
\end{equation}
kur $x_{j_{max}}$ ir $x_{j_{min}}$ yra maximali ir minimali $j$-ojo matų reikšmės. ,,Susietumo`` reikšmių atnaujinimas yra vykdomas $n$ kartų ir kuo didesnė galutinė reikšmė, tuo svarbesnis matas. Aprašyta algoritmo versija yra skirta dviejų klasių atvejui, tačiau yra ir multiklasinis algoritmo variantas.

\subsubsection{Asimetrinis priklausomybės koeficientas}

Asimetrinis priklausomybės koeficientas (angl. \textit{asymetric dependency coefficient}, ADC) yra matų reitingavimo motodas, kuris matuoja mėginio grupės tikimybinę priklausomybę $j$-ajam matui, naudodamas informacijos prieaugį (angl. \textit{information gain}) \cite{kent1983information}:
\begin{equation}
 ADC(Y, j) = \frac{MI(Y, X_j)}{H(Y)},
\end{equation}
kur $H(Y)$ -- klasės $Y$ entropija (angl. \textit{entropy}), o $MI(Y, X_j)$ -- yra tarpusavio informacija \cite{Shannon:2001:MTC:584091.584093} (angl. \textit{mutual information}) tarp mėginio grupės $Y$ ir $j$-ojo mato.
\begin{equation}
 H(Y)=-\sum_y{p(Y=y)log{p(Y=y)}}, 
\end{equation}
\begin{equation}
 H(X_j)=-\sum_x{p(X_j=x) log{p(X_j=x)}},
\end{equation}
\begin{equation}
 MI(Y, X_j) = H(Y) + H(X_j) - H(Y, X_j),
\end{equation}
\begin{equation}
 H(Y, X_j) = -\sum_{y,x_j}{p(y, x_j)log{p(y, x_j)}},
\end{equation}
Kuo didesni ADC įverčiai, tuo matas yra svarbesnis, nes turi daugiau informacijos apie mėginio priklausomybę grupei.

\subsubsection{Absoliučių svorių SVM}

Atraminių vektorių metodas (SVM) yra vienas populiariausių klasifikavimo algortimų, nes jis gerai susidoroja su daugiamačiais duomenimis \cite{guyon2002gene}. Yra keletas bazinių SVM variantų \cite{vapnik2000nature}, bet šiame darbe naudosime tiesinį SVM, nes jis demonstruoja gerus rezultatus analizuojant genų ekspresijos duomenimis. Tiesinis SVM yra hiperplokštuma apibrėžta kaip:
\begin{equation}
 \sum_{j=1}^{p}{w_jx_j + b_0 = 0},
\end{equation}
kur $p$ -- dimensijų kiekis, $w_j$ -- j-osios dimensijos svoris, $x_j$ -- j-osios
dimensijos kintamasis, $b_0$ -- konstanta. Dimensijos absoliutus\footnote{Svorį
reikia imti absoliutaus dydžio, nes neigiamas svoris implikuoja priklausomybę 
vienai klasei, o teigiamas kitai klasei.} svoris $w_j$ gali būti panaudotas
dimensijų reitingavimui. Pastebėtina, kad svorių nustatymas yra atliekamas tik 
vieną kartą\footnote{SVM-RFE dimensijų atrinkimo metodas svorius nustato daug kartų.}.


\subsection{Suprogramuotų dimensijų atrinkimo algoritmų palyginimas}

\subsubsection{Dimensijų atrinkimo algoritmų skaičiavimo laikas}

Eksperimentai buvo atlikti su AltarA duomenų rinkiniu, kompiuteryje naudojant tik vieną procesoriaus branduolį, bet 2 GB RAM atminties. 

~\ref{fig:visu_laikas} pavaizduotas skaičiavimo laikas nuo mėginius apibūdinančių dimensijų skaičiaus. Pats sparčiausias dimensijų atrinkimo metodas yra \textit{Fisher} įvertis. Lėčiausias multikriterinis dimensijų atrinkimo \textit{Fusion} metodas.
\begin{figure}
 \centering
 \includegraphics[width=0.7\textwidth]{images/all_performance.png}
 \caption{Pagrindinių dimensijų atrinkimo metodų skaičiavimo laikas.}
 \label{fig:visu_laikas}
\end{figure}
~\ref{fig:cgs_laikas} pavaizduotas konsensuso grupėmis grįsto dimensijų atrinkimo algoritmo skaičiavimo laiko priklausomybė nuo mėginius apibūdinančių dimensijų kiekio. Algoritmo sudėtingumas laiko atžvilgiu yra kvadratinis. Jei lyginsime su dimensijų reitingavimo algoritmais, tai šis algoritmas yra daug kartų lėtesnis.
\begin{figure}
 \centering
 \includegraphics[width=0.7\textwidth]{images/cgs_performance.png}
 \caption{Konsensuso grupėmis grįtas dimensijų atrinkimo metodo skaičiavimo laikas.}
 \label{fig:cgs_laikas}
\end{figure}

Pagal gautus laiko priklausomybės nuo dimensijų kiekio grafikus galime daryti išvadą, kad CGS algoritmas daugiamačių duomenų dimensijų atrinkimui nėra tinkamas, nes darbo laikas yra per didelis.

\subsubsection{Klasifikavimo tikslumas}
\newpage

%% Rezultatų ir išvadų dalyje išdėstomi pagrindiniai darbo rezultatai (kažkas išanalizuota, kažkas sukurta, kažkas diegta), pateikiamos išvados (daromi nagrinėtų problemų sprendimo metodų palyginimai, siūlomos rekomendacijos, akcentuojamos naujovės).

\section*{REZULTATAI IR TOLIMESNIŲ TYRIMŲ KRYPTYS}

Šiame darbe analizuota vis didesnį susidomėjimą kelianti daugiamačių duomenų klasifikavimo problematika ypatingą dėmesį kreipiant į matų atrinkimo problematiką. 

Eksperimentai parodė, kad nėra vieno universaliai geriausio matų atrinkimo metodo. Reikia atsižvelgti į turimus duomenis bei į darbui keliamus tikslus.

Šiame darbe sukaupta patirtis gali būti panaudota kaip tolimesnių daugiamačių biomedicininių duomenų tyrimų pagrindas. Tolimesnių tyrimų kryptys galėtų būti stabilių matų, kurie maksimaliai padidina klasifikavimo tikslumą, atrinkimo metodo paiešką.

Čia bus rezultatai ir išvados. Ir nutarta, kad čia bus aprašytos tolimesnių tyrimų kryptys.
\newpage

\addcontentsline{toc}{section}{LITERATŪRA} 
\bibliographystyle{alpha}
\bibliography{../bachelor/literatura.bib}

\end{document}