% 2.	Įmonės/įstaigos apibūdinimas. Glaustai aprašoma įmonė/įstaiga, kurioje buvo
% atliekta praktika: jos veiklos sritis, organizacinė struktūra, teikiamos 
% paslaugos ir kt. Apibūdinamos praktikos vietoje sudarytos darbo sąlygos
% (1-2 psl.).

\section{Įstaigos apibūdinimas}

Vilniaus universiteto matematikos ir informatikos institutas (MII) nuo 2010 m. yra Vilniaus 
universiteto padalinys užsiimantis tyrimais matematikos ir informatikos srityse.
Instituto įkūrimo data laikoma 1965 m. spalio 1d., kai buvo panaikintas Lietuvos
mokslų akademijos Fizikos ir technikos institutas ir įkurti trys nauji
institutai, tarp kurių buvo Fizikos ir matematikos institutas, kuris laikomas 
MII pirmtaku. 

Pagrindinė instituto veikla - moksliniai tyrimai ir eksperimentinė plėtra. Kitos
veiklos sritys yra: mokslininkų ugdymas (doktorantūros studijos); mokslo 
organizacinė veikla; leidyba; mokymas, moksleivių ugdymas, švietimas.
Mokslinė veikla sukoncentruota 11-oje mokslinių padalinių. Institute yra 5
matematikos krypties padaliniai, 7 informatikos bei informatikos inžinerijos 
padaliniai:
\begin{enumerate}
  \item Atpažinimo procesų skyrius;
  \item Atsitiktinių procesų skyrius;
  \item Informatikos metodologijos skyrius;
  \item Kompiuterinių tinklų laboratorija;
  \item Matematinės logikos sektorius;
  \item Programų sistemų inžinerijos skyrius;
  \item Sistemų analizės skyrius (SAS);
  \item SAS optimizavimo sektorius;
  \item SAS operacijų tyrimo sektorius;
  \item Skaičiavimo metodų skyrius (SMS);
  \item SMS diferencialinių lygčių sektorius;
  \item Tikimybių teorijos ir statistikos skyrius;
\end{enumerate}

MII man, kaip ir kiekvienam darbuotojui, parūpino: darbo vietą, prisijungimo prie 
vietinio tinklo, galimybe naudotis skaičiavimo ištekliais, galimybe su nuolaida 
pietauti vietinėje valgykloje. MII darbuotojai buvo kolegiški, todėl apsipratimas
MII įvyko labai greitai. Todėl jau nuo pat pirmosios profesinės praktikos dienos
galėjau pradėti spręsti užsibrėžtus uždavinius.
