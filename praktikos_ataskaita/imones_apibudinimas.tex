% 2.	Įmonės/įstaigos apibūdinimas. Glaustai aprašoma įmonė/įstaiga, kurioje buvo
% atliekta praktika: jos veiklos sritis, organizacinė struktūra, teikiamos 
% paslaugos ir kt. Apibūdinamos praktikos vietoje sudarytos darbo sąlygos
% (1-2 psl.).

\section{ĮSTAIGOS APIBŪDINIMAS}
\label{imones_apibudinimas}

Vilniaus universiteto matematikos ir informatikos institutas (MII) nuo 2010 m. yra Vilniaus universiteto padalinys užsiimantis tyrimais matematikos ir informatikos srityse. Instituto įkūrimo data laikoma 1965 m. spalio 1d., kai buvo panaikintas Lietuvos mokslų akademijos Fizikos ir technikos institutas ir įkurti trys nauji institutai, tarp kurių buvo Fizikos ir matematikos institutas, kuris laikomas MII pirmtaku. 

Pagrindinė instituto veikla - moksliniai tyrimai ir eksperimentinė plėtra. Kitos veiklos sritys yra: mokslininkų ugdymas (doktorantūros studijos) (MII suteikta teisė ruošti matematikos, informatikos ir informatikos inžinerijos sričių mokslininkus); mokslo organizacinė veikla - konferencijos, seminarai, parodos, mokslinių knygų redagavimas; leidyba; mokymas, moksleivių ugdymas, švietimas. Mokslinė veikla sukoncentruota 12-oje mokslinių padalinių. Institute yra 5 matematikos krypties padaliniai, 7 informatikos bei informatikos inžinerijos padaliniai:
\begin{enumerate}
  \item Atpažinimo procesų skyrius;
  \item Atsitiktinių procesų skyrius;
  \item Informatikos metodologijos skyrius;
  \item Kompiuterinių tinklų laboratorija;
  \item Matematinės logikos sektorius;
  \item Programų sistemų inžinerijos skyrius;
  \item Sistemų analizės skyrius (SAS);
  \item SAS optimizavimo sektorius;
  \item SAS operacijų tyrimo sektorius;
  \item Skaičiavimo metodų skyrius (SMS);
  \item SMS diferencialinių lygčių sektorius;
  \item Tikimybių teorijos ir statistikos skyrius;
\end{enumerate}

MII organizuoja moksleivių ugdymą: veikia jaunųjų programuotojų neakivaizdinė mokykla, rengiamos lietuvos moksleivių informatikos ir matematikos olimpiados, rengiamas informacinių technologijų konkursas ,,Bebras``. MII yra vienas iš Lietuvos jaunųjų matematikų mokyklos steigėjų, jaunųjų matematikų konkurso ,,Kengūra`` rengėjas. Taip pat MII prisideda prie kompiuterijos naudotojų švietimo ir mokymo: dirba informatikos terminijos komisija, multimedijos centras humanitarams, palaikomas tinklalapis apie lietuviškų rašmenų naudojimą elektroninio pašto laiškuose.

MII leidybos skyrius atsakingas už visą eilę recenzuojamų periodinių leidinių: ,,Informatica``, ,,Informatics in Education``, ,,Lithuanian Mathematical Journal``, ,,Lietuvos matematikos rinkinys. LMD darbai``, ,,Mathematical Modelling and Analysis``, ,,Nonlinear Analysis. Modelling and Control``, ,,Olympiads in Informatics``. MII taip pat yra išleidusi mokslinių bei mokslo populiarinimo knygų lietuvių ir anglų kalbomis, mokymo priemonių, interaktyvių kompaktinių diskų bei sukūrusi įvairių internetinių informacinių sistemų (pvz. enciklopedinis kompiuterijos terminų žodynas).

MII man, kaip ir kiekvienam darbuotojui, parūpino: darbo vietą, prisijungimo prie vietinio tinklo, galimybe naudotis skaičiavimo ištekliais, galimybe su nuolaida pietauti vietinėje valgykloje. MII darbuotojai buvo kolegiški, todėl apsipratimas MII įvyko labai greitai. Todėl jau nuo pat pirmosios profesinės praktikos dienos galėjau pradėti spręsti užsibrėžtus uždavinius.
