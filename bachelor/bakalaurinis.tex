\documentclass[12pt, a4paper]{article}

\usepackage[utf8x]{inputenc}
\usepackage[lithuanian]{babel}
\usepackage[L7x]{fontenc}
\usepackage{lmodern}
\usepackage{verbatim}
\usepackage{tocloft} 
\usepackage{url}
\usepackage{setspace}
\usepackage{caption}
\renewcommand{\cftsecleader}{\cftdotfill{\cftdotsep}}
\usepackage{parskip}
\usepackage{graphicx}
\usepackage{fixltx2e} % šitas dėl \textsubscript{}
\usepackage{color} % spalvinsim tekstą
\parindent = 1cm

\usepackage{mathtools} % pades rašyti matematinius intarpus
\usepackage{titlesec}
\titlelabel{\thetitle.\quad} % Padeda tašką po skyriaus numerio
\usepackage{indentfirst} % atitraukta pirmoji eilute
\usepackage{longtable} % darysim didelias lenteles
\usepackage{algorithm}
\usepackage{algpseudocode}
\usepackage[top=2cm, bottom=2cm, left=3cm, right=1.5cm]{geometry}
\usepackage{changepage}
\usepackage{pdfpages} % su šitua paketu galima įkelti pdf kaip puslapius
\linespread{1.5}

\begin{document}

%\begin{titlepage}

\begin{center}
VILNIAUS UNIVERSITETAS\\
MATEMATIKOS IR INORMATIKOS FAKULTETAS\\
PROGRAMŲ SISTEMŲ KATEDRA\\
\vspace{150pt}

\huge \textbf{Daugiamačių duomenų klasifikavimo analizė\\}
\vspace{20pt}
\large\textbf{Classification Analysis of High-Dimensional Data\\}
\vspace{20pt}
\small Bakalauro baigiamasis darbas\\
\vspace{40pt}
\end{center} 


\begin{flushleft}
Atliko: \hspace{57pt} Dainius Jocas \hspace{125pt}\textsubscript{(para\v{s}as)}

\vspace{10pt}
Darbo vadovas: \hspace{14pt} dr. Juozas Gordevičius \hspace{76pt}\textsubscript{(para\v{s}as)}

\vspace{10pt}
Darbo recenzentas: prof. dr. Romas Baronas \hspace{66pt}\textsubscript{(para\v{s}as)}
\\
\vspace{130pt}
\end{flushleft}

\begin{center}
VILNIUS - 2012
\end{center}

\end{titlepage} % kol kas neaktualu

%\section*{SANTRAUKA}

Čia bus santrauka lietuvių kalba.
%\newpage

%\section*{SUMMARY}
\label{summary}

This bachelor thesis is dedicated to classification analysis of noisy, high-dimensional, small-sample biomedical data, which is often found in genetical, gene expression, epigenetics research. Next, we study feature selection methods as they are necessary for the analysis of high-dimensional data. We put special emphasis on robustness of feature selection because this is of paramount importance in the biomedical domain. We evaluate different feature selection methods with respect to computation complexity, accuracy of resulting classifier, and robustness of selected features. Experimental analysis revealed that there is no single best method for the given data and application domain. Therefore, in the summary of our work we provide directions for future research.

\textbf{Keywords:} high-dimensional data, machine learning, classification, feature selection, robustness of feature selection.

%\newpage

\let \savenumberline \numberline
\def \numberline#1{\savenumberline{#1.}}
\setcounter{tocdepth}{5}
\setcounter{secnumdepth}{5}
\tableofcontents  
\newpage

\addcontentsline{toc}{section}{{\c I}VADAS}
%%% Įvade aprašomi darbo tikslai, nurodomas temos aktualumas, aptariamos teorinės darbo prielaidos bei metodika, apibrėžiamas tiriamasis objektas, apibūdinami su tema susiję literatūros ar kitokie šaltiniai, temos analizės tvarka, darbo atlikimo aplinkybės, pateikiama žinių apie naudojamus instrumentus (programas ir kt.). Darbo įvadas neturi būti dėstymo santrauka. Įvado apimtis 3–4 puslapiai.

\newpage
\section*{ĮVADAS}

%% JG: Idėja - naudok terminus "matai" ir "mėginiai". Kiekvienam mėginiui atliekama labai daug matavimų, iš to ir "daugiamatiškumas"

% JG: Paminėk, kad šiame darbe mes gilinamės į biomedicinoje kaupiamų genetinių duomenų analizės specifika. Šie duomenys ypatingi dėl daugiamatiškumo, mažo mėginių kiekio, triukšmingų matavimų.

Nuolat vystosi technologijos skirtos gauti biomedicininius duomenis, pvz. genomo sekvenavimas\cite{pettersson2009generations}, o tai reiškia, kad didėja gaunamų duomenų detalumas. Detalumas reiškia, kad daugėja biomedicininius duomenis abibūdinančių faktorių arba matų skaičius. Duomenys, kurių kiekvienas mėginys aprašomas dideliu skaičiumi matų, yra vadinami daugiamačiais duomenimis.

Šiame darbe yra nagrinėjama biomedicinoje kaupiamų genetinių daugiamačių duomenų analizės specifika. Šie duomenys yra ypatingi tuo, kad jie įprastai turi šimtus kartų daugiau matų nei mėginių. Santykinai mažas mėginių skaičius turimas, nes mėginio gavimo kaina yra aukšta. Biomedicininių duomenų analizę apsunkina ir tai, kad matavimai, kuriais tie duomenys gaunami, įneša atsitiktinių duomenų - triukšmo. Triukšmas matavimo metu gali atsirasti dėl įvairių priežasčių, pvz. netinkamai paruoštų cheminių preparatų. Kai duomenys yra triukšmingi, didėja tikimybė duomenyse rasti atsitiktinių priklausomybių. Tai yra viena priežasčių, kodėl biomedicininių duomenų analizės procesas yra sudėtingas.

%% JG: Būdai neatsiranda, o vystosi technologija. Jie nėra tikslesni, bet detalesni, t.y. kiekvienam mėginiui atliekama daugiau matavimų. 

%% JG: Nors matavimų kiekis didėja, mėginio kaina išlieka gana aukšta. Todėl biomedicinos eksperimentuose gaunami duomenys ypatingi tuo, kad matų visados ženkliai daugiau nei mėginių.

Klasifikavimu\cite{fisher1936use} yra vadinamas duomenų analizės procesas, kai duomenys suskirstomi į grupes pagal tam tikrus jų požymius. Algoritmai arba funkcijos, kurios turimus duomenis priskiria iš anksto žinomoms grupėms - atlieka klasifikavimą - yra vadinami klasifikatoriais. Klasifikatoriai paruošiami naudojant turimus mėginius - treniravimosi duomenis - ir informaciją apie jų būklę (sveikas ar sergantis). Klasifikatoriaus ruošimo procesas yra vadinamas apmokymu. Apmokyti klasifikatoriai paprastai naudojami nustatant naujų, dar nematytų, mėginių - testavimo duomenų - būklę. Pagal tai, kokią dalį visų testavimo mėginių klasifikatorius priskiria neteisingai klasei, yra nustatomas klasifikatoriaus tikslumas. 

Biomedicininių duomenų tipinė klasifikavimo užduotis yra atskirti sveikų pacientų mėginius nuo sergančiųjų. Klasifikavimu siekiama nustatyti, kurie matai veikdami drauge, geriausiai paaiškina skirtumą tarp ligos paveiktų ir sveikų mėginių. Labiausiai ligą paaiškinančių matų nustatymas galėtų palengvinti tiriamų ligų diagnozės ar gydymo metodų kūrimą.

Biomediciniuose duomenyse dažniausiai turime tik kelias dešimtis mėginių, todėl norint geriau įvertinti klasifikatoriaus tikslumą yra naudojami pakartotinio mėginių poaibio atrinkimo (angl. \textit{resampling}) metodai: kryžminio patikrinimo (angl. \textit{cross-validation}) arba įkelčių (angl. \textit{bootstrap\footnote{Terminas \textit{bootstrap} ,,įkelties`` prasme pradėtas naudoti dar Rudolfo Ericho Raspės knygoje ,,Barono Miunchauzeno nuotykiai``, kurioje Baronas Minchauzenas užkėlė save ant arklio tempdamas į viršų savo batų raištelius (angl. \textit{bootstraps}).}}). Šių metodų naudojimas naudojimas su duomenimis, kurių tikrasis pasiskirstymas nėra žinomas, padeda įvertinti klasifikavimo variabilumą (angl. \textit{variance}) ir sisteminį nuokrypį (angl. \textit{bias}).

Naudojant kryžminio patikrinimo metodą, daug kartų sudaromos skirtingos treniravimosi ir testinės mėginių imtys. Taikant atskirą šio metodo variantą, kryžminį patikrinimą išbraukiant po vieną mėginį (angl. \textit{leave-one-out cross-validation}), iš treniravimosi imties išbraukiamas vienas mėginys ir apmokomas klasifikatorius, kuris klasifikuoja išbrauktąjį mėginį. Procesas tęsiamas tol, kol suklasifikuojami visi objektai. Kitais kryžminio patikrinimo metodo variantais iš treniravimosi mėginių yra išmetama po keletą mėginių. Pagal tai, kiek testinių mėginių klasifikatorius priskyrė klaidingai kategorijai, yra nustatoma vidutinė klaidingo klasifikavimo tikimybė. Šiuo metodu gauti įverčiai pasižymi didele dispersija \cite{braga2004cross}.

Naudojant įkelčių metodą, iš $n$ dydžio mėginių aibės yra paimama tokio pačio dydžio atsitiktinių mėginių imtis su pasikartojimais, kuri vadinama įkelties treniravimosi imtimi. Į šią imtį nepaimti mėginiai yra priskiriami testavimo imčiai. Naudojant įkelties treniravimosi mėginių imtį yra apmokomas klasifikatorius, kuris klasifikuoja testavimo imtį. Procesą kartojant gaunama klaidingo klasifikavimo tikimybės įverčių imtis. Šios imties vidurkis yra klaidingo klasifikavimo tikimybės įvertis. Dažniausiai naudojamas ,“0.623 įkelčių`` (angl. \textit{0.623\footnote{0.623 yra tikimybė mėginiui būti įtrauktam į treniravimosi imtį.} bootstrap}) įverčiu. Šiuo metodu gautas klaidingo klasifikavimo tikimybės įverti pasižymi maža dispersija \cite{michie1994machine}.

Biomedicininių duomenų kontekste galima daryti prielaidą, kad ne visi matai yra susiję su tiriama problema, pvz. gaubtinės žarnos vėžiu, dėl tokių faktorių, kaip triukšmas duomenyse. Paprastai nagrinėjamai problemai svarbus yra mažas, palyginus su visu, matų kiekis. Ši biomedicininių duomenų ypatybė veda prie ,,daugiamatiškumo prakeiksmo`` (angl. \textit{the curse of dimentionality})\cite{bellman1966adaptive} - didėjant matų
kiekiui mėginiai pasidaro panašūs, o bandymas juos klasifikuoti tolygus spėliojimui. Todėl biomedicininių duomenų daugiamatiškumui sumažinti yra naudojami informatyviausių dimensijų atrinkimo metodai\cite{guyon2003introduction} (angl. \textit{feature selection}). Pagal tai, kaip susiję su klasifikatoriumi, matų atrinkimo metodai skirstomi į tris kategorijas\cite{saeys2008robust}: filtruojantys (angl. \textit{filter}), 
prisitaikantys (angl. \textit{wrapper}) ir įterptiniai (angl. \textit{embedded}) metodai. Dimensijų atrinkimas yra svarbi biomedicininių duomenų apdorojimo (angl. \textit{preprocessing}) etapo dalis. Naudojant dimensijų atrinkimo metodus galima kovoti su ,,daugiamatiškumo prakeiksmu`` dimensijų skaičių priartinant prie mėginių skaičiaus.

%% JG: Kodėl klasifikuojama? Norima nustatyti, kokie matai, veikdami drauge, geriausiai paaiškina skirtumą tarp ligos paveiktų ir sveikų mėginių.

%% JG: Kokios yra tradicinės klasifikavimo strategijos ir kodėl jos neveikia daugiamačių duomenų atveju? Skaičiavimo laikas nėra problema - Random forests veikia visai neblogai tokiais atvejais. Daugiamatiškumas veda prie the curse fo dimensionality.

% JG: sumažinus naudojamų matų kiekį, matų kiekis priartėja prie mėginių kiekio ir tokiu būdu apeinamas daugiamatiškumo prakeiksmo problema. Biomedicinos duomenų kontekste, galima daryti prielaidą, kad dauguma matų yra beprasmiai, pvz., tik kai kurie genai įtakoja ligą, todėl matų mažinimimas yra prasmingas. Taip pat, kuriant medicininius diagnostikos įrankius, naudojamų matų kiekis įtakoja įrankio kainą. Todėl pageidautina turėti kuo mažiau matų.

Kadangi biomedicininiuose duomenyse reikšmingų dimensijų kiekis tiriamai problemai yra nedidelis, todėl norima žinoti, kuris dimensijų poaibis yra svarbus tai problemai. Tokioje situacijoje tampa svarbu, kaip varijuoja atrenkamų dimensijų aibė, kai dimensijų atrinkimas vykdomas su vis kitu mėginių poaibiu. Dimensijos, kurios keičiant mėginių, kurie naudojami dimensijų atrinkime, poaibį, yra vėl ir vėl atrenkamos yra vadinamos stabiliomis dimensijomis \cite{needcitation}. Tačiau skirtingi dimensijų atrinkimo metodai tiems patiems mėginiams gali atrinkti skirtingas dimensijas. Taip pat, suskaidžius duomenis į persidengiančius poaibius ir atrinkus tą patį kiekį dimensijų tuo pačiu metodu, gaunami skirtingi dimensijų poaibis. Tačiau, norint geriau suprasti biomedicininius duomenis, itin svarbu fokusuoti dėmesį į sąlyginai nedidelį dimensijų poaibį. Dimensijų aibės sumažinimas paspartina biomedicininių duomenų tyrimus - tyrėjams reikia atlikinėti bandymus su mažesniu mėginių skaičiumi. Mažesnio skaičiaus mėginių tyrimas kainuoja mažiau, nes mažiau reikia žmonių darbo laiko, mažiau reikia ir cheminių reagentų. Todėl stabilių dimensijų atrinkimas dirbant su biomedicininiais duomenimis yra \textit{(angl. robustness)}.

% JG: Žiūrėkim į stabilumą kaip į šalutinį matų atrinkimo efektą, kurį svarbu pažaboti. Stabilumas svarbus, nes, analizuojant duomenis norima ne tik nustatyti, koks būtų vidutinis klasifikatoriaus tikslumas, bet ir sukurti tą vidutinį klasifikatorių. Norint pastarąjį sukurti, reikia žinoti, kuriuos konkrečius matus naudoti. 

% JG: Šitam paragrafe suplaki daug svarbių dalykų į krūvą ir juos turėtum būti paaiškinęs jau anksčiau. Mėginių trūkumas nėra tavo sprendžiama problema. Tačiau, kuo mažiau duomenų, tuo nestabilesni atrenkami matai. 

%Taigi, dirbant su daugiamačiais duomenimis, reikia atsižvelgti į keletą kriterijų:
%\begin{enumerate}
% \item Klasifikavimo tikslumą;
% \item Dimensijų atrinkimo stabilumą, atsižvelgiant į klasifikavimo rezultatus;
% \item Triukšmo lygį duomenyse;
% \item Skaičiavimo išteklių naudojimo racionalumą.
%\end{enumerate}
%Reikalavimas vienu metu atsižvelgti į keletą kriterijų užduotį daro sudėtinga. Klasifikuojant daugiamačius duomenis uždavinys yra surasti geriausius rezultatus duodančią strategiją, kuri geriausiai atsižvelgia į minėtus kriterijus.

%Darbo eksperimentinei daliai reikalingus skaičiavimo išteklius, suteikė VU MIF skaitmeninių tyrimų ir skaičiavimų centras \cite{mif2012stsc}. Eksperimentuose buvo naudojami laisvai internete prieinami biomedicininių duomenų rinkiniai (angl. \textit{datasets}). Biomedicininių duomenų apdorojimo algoritmų implementavimui buvo naudojama R \cite{r2012statistics} programavimo kalba. Eksperimentai atlikti profesinės praktikos MII metu.

Dimensijų atrinkimo stabilumo problemą Yang ir Mao \cite{yang2011robust} siūlė spręsti reitinguojant dimensijas remiantis keletos dimensijų atrinkimo metodų rezultatais. Galutinis dimensijų reitingų sąrašas gaunamas, kai po kiekvieno dimensijų atrinkimo yra išmetama viena žemiausią reitingą turinti dimensija iš dimensijų aibės, ir dimensijų atrinkimas yra kartojamas tol, kol nebelieka dimensijų. Tačiau dimensijų atrinkimo metodų kiekis yra ribotas ir skirtingų metodų dažnai negalima atlikti paraleliai. Tai riboja šio metodo pritaikomumą daugiamačių duomenų analizėje.

Dimensijų atrinkimo stabilumo problemą siūlyta spręsti surandant dimensijų grupių tankio centrus ir naudoti dimensijas, kurios artimiausios tiems centrams \cite{yu2008stable}. Pasiūlytas grupių tankių algoritmas užtrunka $O(\lambda n^2m)$ laiko, kur n yra dimensijų kiekis, o m - mėginių skaičius. Vėliau Loscalzo ir kt. pasiūlė mokymo duomenis skaidyti poaibiais ir kiekviename poaibyje ieškoti tankių grupių, o tada imti sprendimą balsavimo principu \cite{loscalzo2009consensus}. Nors šie metodai siūlo stabilų dimensijų atrinkimą, tačiau šių metodų panaudojamumą daugiamačiuose duomenyse riboja skaičiavimo sudėtingumas.

Šiame bakalauriniame darbe remiantis Yang, Mao bei Loscalzo darbuose pateiktomis įžvalgomis, bus stengiamasi pasiūlyti tyrimų kryptis, kurios galėtų padėtų sukurti metodus, skirtus spręsti stabilių dimensijų atrinkimo problemą. Idėja yra sugrupuoti dimensijas pagal greitą klasterizacijos algoritmą, išrinkti reprezentatyviausias dimensijas, transformuoti dimensijų erdvę ir joje vykdyti dimensijų atrinkimą remiantis keletu dimensijų atrinkimo metodų.

Šio darbo tikslas yra išanalizuoti darbo su daugiamačiais duomenis ypatybes. Šiam darbui yra keliamos tokios užduotys:
\begin{enumerate}
 \item Susipažinti su naujausiais klasifikavimo ir dimensijų atrinkimo metodais;
 \item Atlikti dimensijų atrinkimo metodų palyginimo eksperimentus;
 \item Pasiūlyti kryptis, kaip dabartiniai metodai gali būti patobulinti ir paruošti naujųjų metodų prototipus.
\end{enumerate}

Tolimesnė bakalaurinio darbo struktūra yra tokia: skyriuje


\newpage

\section{SUSIJUSIŲ DARBŲ APŽVALGA}
\label{darbu_apzvalga}

Šiame skyriuje aprašysiu teoriją reikalingą bakalauriniame darbe atliekamam tyrimui: mašininis mokymasis, SVM klasifikatorius, random forest, dimensijų atrinkimo stabilumas.

\subsection{Mokymasis su mokytoju ir mokymasis be mokytojo}

Šiame skyriuje atsakysiu į klausimą kuo skiriasi mokymasis su
mokytoju (angl. supervised learning) nuo mokymosi be mokytojo (angl.
unsupervised learning). Mokymasis, duomenų klasifikavimo kontekste, reiškia modelių(pvz. klasifikatorių) kūrimo metodus (algoritmus), kurie naudoja
mokymosi duomenis\footnote{Mokymosi duomenys (angl. sample data)- duomenys,
kurie yra paruošti darbui programų, kurios kurs modelius (pvz.
klasifikatorius).}, kitaip tariant, tai mokymasis iš pavyzdžių.

\subsubsection{Mokymasis su mokytoju}

Mokymasis su mokytoju tai toks mokymasis, kai turime mokymosi duomenis, kuriems jau
yra priskirtas tam tikras teisingas atsakymas. Kitaip tariant, mes sprendžiame
uždavinį, kuriam atsakymą galime pasitikrinti. Mokymasis su mokytoju yra
skirstomas į dvi rūšis:
\begin{enumerate}
  \item Klasifikavimas (angl. classification) - pagal nepriklausomus
  kintamuosius bandome nuspÄ—ti kokybinius (kategorinius) priklausomus kintamuosius. 
  \item Regresija (angl. regression) - pagal nepriklausomus kintamuosius bandome
  nuspÄ—ti kiekybinius priklausomus kintamuosius.
\end{enumerate} 

%% JG: Pateik vizualų klasifikavimo pavyzdį iliustruojanti visus 3 etapus.
%% DJ: Vizualų, ta prasme su paveiksliukais ar ir tas pavyzdys su paštu
% pakankamai vaizdingas?

%% JG: Reikia kitaip struktūrizuoti šitą skyrių: 
% +Pradžioj pasakyk, kad yra klasifikavimas ir regresija ir po
%  sakinį kiekvienam.
% +Tada aptark klasifikavimą ir pateik pavyzdį. 
% +Tada pateik regresijos pavyzdį.
% +Tada parašyk, kad šiame darbe studijuojama klasifikavimo problema.

\paragraph{Klasifikavimo uždavinio pavyzdys}

Klasifikavimo tikslas - identifikuoti parametrus, kurie nusakytų grupę (klasę),
kuriai priklauso objektas. Klasifikavimo sąvoka gali būti naudojama tiek esamų
duomenų suvokimui, tiek naujų objektų charakteristikų prognozavimui.
Klasifikavimo uždavinių aktualumą galima parodyti tokiu pavyzdžiu.

\begin{figure}[htb]
\begin{center}
\leavevmode
\includegraphics[width=0.5\textwidth]{images/ranka_rasyti_skaiciai.png}
\end{center}
\caption{Ranka rašytas tekstas, kurį reikia atpažinti.}
\label{fig:flash}
\end{figure}

Uždavinys: Pašto skyriuose laiškai siun�iami įvairiomis kryptimis pagal gavėjo
adresą ir (arba) pašto kodą. Norima automatizuoti laiškų rūšiavimą pagal
siuntimo kryptį. Tam, kad būtų galima laiškų rūšiavimą pagal kryptį
automatizuoti, mums reikia priemonės atpažinti ant voko užrašytą
pašto kodą.

Sprendimas: Šią problemą mums padėtų išspręsti skeneris ir programine įranga,
kuri sugebėtų ranka rašytus skaitmenis atpažinti ir konvertuoti į skaitmeninį
formatą. Tų skaitmenų atpažinimui ir konvertavimui į skaitmeninį formatą
mes naudosime klasifikavimo algoritmus, nes uždavinys pasižymi
visomis klasifikavimui būdingomis savybėmis: turime aibę duomenų (vaizdinė
informacija su ranka rašytais skaitmenimis), turime teisingus atsakymus (žmogus
pažiūrėjęs į ranka rašytą skaitmenį gali pasakyti programai, koks ten yra
skaitmuo), bei galimų sprendimai yra kategorinio tipo (dešimt skaitmenų nuo
0 iki 9).

Klasifikatorių kursime trimis etapais:
\begin{enumerate}
  \item diskriminavimo (atskirian�iųjų) kintamųjų parinkimas - nuskenuotų
  pašto kodų skaitmenų dažniausiai pasitaikan�ių, charakteringiausių linijų
  radimas,
  \item klasifikavimo taisyklių sudarymas - pagal tam tikrą charakteringiausių
  linijų grupę objektui priskiriama klasė,
  \item klasifikavimo kokybės įvertinimas - kokybei įvertinti naudojami įvairūs
  metodai, tokie kaip kryžminis patikrinimas (angl. cross-validation) ir
  įkel�ių metodas (angl. bootstrap).
\end{enumerate}

Įgyvendinę aukš�iau aprašyto uždavinio sprendimą, pašto skyriaus vadybininkai
galėtų atlaisvinti žmones nuo iš esmės mechaninio darbo - rūšiuoti laiškus.
Tokiu būdu būtų optimizuotas pašto skyrių efektyvumas.

\paragraph{RegresinÄ—s analizÄ—s payzdys}

Regresija prognozuojant naujų duomenų reikšmes naudojasi žinomais, jau turimais
duomenimis. Ji naudoja standartinius statistinius metodus, tokius kaip mažiausių
kvadratų metodas (angl. least squares). Regresinė analizė dažniausiai naudojama
įvertinti (ang. forecast) ateities duomenų vertes bei interpoliacijai -
funkcijos tikėtinos reikšmės tarp dviejų taškų įvertinimui.

Tipinio uždavinio, kuriam naudojama regresinė analizė pavyzdys: Aktuarinėje
(draudimo) matematikoje reikia turėti įver�ius, pasakan�ius kokia tikimybė, kad
žmogus vienokio ar kitokio amžiaus mirs. Tam yra naudojamos taip vadinamos 
mirtingumo lentelės. Jose duomenys aprašo kiek ir kokio amžiaus žmonių
kažkuriais metais mirė, pvz. 2010 metais Lietuvoje mirė 1000 20 metų amžiaus 
žmonių. Detalesni duomenys nėra naudojami, nes per daug sudėtinga juos apdoroti.
Kadangi aktuarai nori apskai�iuoti draudimo kainą, jiems reikia įvertinti
riziką, kada žmogus mirs, tai jie naudodamiesi regresinės analizės metodais 
paskai�iuoja tikėtiniausią reikšmę, kad pvz. yra 3\% tikimybė, kad žmogus  mirs
dvidešimtaisiais savo gyvenimo metais. Kitais žodžiais tariant, iš turimų
duomenų mes sukursime tolydžią funkcija, kuri mums pasakys reikšmes taškuose,
kurių mes neturime.

\paragraph{Klasifikavimas ir regresija}

Abiejų mokymosi su mokytoju rūšių tikslas yra pagal mokymosi duomenis sukurti
modelį, kuriuo remiantis būtų galima identifikuoti naujų objektų
savybes.\cite{markhall99} Å iame darbe negrinÄ—sime klasifikavimo problemÄ….

\subsubsection{Mokymasis be mokytojo}

Mokymasis be mokytojo tai toks mokymasis, kai turime mokymosi duomenis, kuriems
nėra priskirtas teisingas atsakymas. Kitaip tariant, mes sprendžiame
uždavinį, kuriam atsakymo galime pasitikrinti. Mokymosi be mokytojo principas - 
maksimizuoti objektų, esan�ių vienoje grupėje, tarpusavio panašumą ir 
minimizuoti tarpgrupinį objektų panašumą.

Mokymosi su mokytoju metu galima išmatuoti gauto modelio tikslumą įvairiais metodais, pvz.
kryžminiu patikrinimu. Mokymesi be mokytojo mes tokių tiesioginio patikrinimo
procedūrų neturime. Todėl yra sunkiau išsiaiškinti patikimumą išvadų, gautų pagal
daugumos mokymosi be mokytojo algoritmų darbo rezultatus. 

Yra mažiausiai penkios pagrindinės priežastys, kodėl mums gali būti įdomūs
mokymosi be mokytojo algoritmai:

\begin{enumerate}
	\item Turime labai daug nesužymėtų (angl. unlabelled) duomenų, o jų
	sužymėjimas rankomis būtų labai brangus. 
	\item Norime apsimokyti su dideliu kiekiu sąlyginai ,,pigių`` duomenų tam,
	kad paskui galÄ—tume	pasitelkti mokymosi su mokytoju algoritmus, ir tada
	detaliau ištirti duomenis.
	\item Duomenų struktūros šablonas yra nuolat kintantis, ir jei tą kitimą
	galėtume sekti mokymosi be mokytojo režimu, tai būtų galima padidinti 
	mūsų programos našumą.
	\item Galima panaudoti mokymosi be mokytojo algoritmus, kad surastume
	duomenų savybes, kurias vėliau panaudosime duomenų kategorizavimui.
	\item Pradinėje duomenų analizės stadijoje pasinaudoję mokymosi be mokytojo
	metodais galime geriau pažinti turimus duomenis.
\end{enumerate}

Mokymosi be mokytojo algoritmų pagrindinis privalumas – gebėjimas atpažinti grupavimo
struktūrą be jokios išankstinės informacijos.

%% JG: neprižiūrimų mokymosi metodų yra visokių: association rule mining,
% clustering, ir t.t. Zr ESL knygos 14 skyrių.
%% DJ: Nurašinėjau nuo Duda knygos tą vietą, kur mokymas be mokytojo ir 
% klasterizavimas yra sinonimai.

%% Kartais šiokia tokia informacija žinoma. Pvz., klasterių kiekis nurodomas
% k-means algoritme. Arba galima daryti prielaidas apie klasterių struktūrą:
% k-means ieško apvalių klasterių. Esminis dalykas yra tas, kad teisingas
% atsakymas nėra žinomas.

%% JG: algoritmas turi atrasti grupes duomenyse, jos nėra iš anksto žinomos.

\subsubsection{Mokymosi su mokytoju ir mokymosi be mokytojo skirtumai}

Pagrindiniai skirtumai tarp mokymosi su mokytoju ir mokymosi be mokytojo yra:
\begin{itemize}
	\item mokymosi duomenys - mokymosi su mokytoju algoritmų įeities duomenyse
	yra	išreikštinai pasakyta, kokio rezultato mes laukiame, o mokymosi be
	mokytojo įeities duomenyse tokios papildomos informacijos nėra.
	\item  naudojimo tikslai - mokymasis su mokytoju siekia iš pavyzdžių
	išmokti vertinti naujus duomenis, o mokymasis be mokytojo siekia atrasti
	vidinę duomenų struktūrą.
\end{itemize}

Aptarkime pavyzdį: nuotraukų apdorojimas.

Mokymosi su mokytoju programai kaip įeities duomenis paduotume keletą 
nuotraukų su žymėmis pasakan�iomis, ar nuotraukoje yra žmogaus veidas ar jo ten
nėra, kitaip tariant, duotume keletą pavyzdžių su teisingais atsakymais.
Programa peržvelgs visas nuotraukas ir susikurs klasifikatorių (modelį), kuris
kažkokiu tikslumu galės atskirti nuotraukas su žmogaus veidu. Tokiu būdu mūsų
mokymosi programa ,,išmoks`` nuotraukose atpažinti veidus.

Mokymosi be mokytojo programai kaip įeities duomenis paduotume keletą
nuotraukų be jokių papildomų žymių. Žinoma, mūsų programa pati nesugebės
,,išrasti``, kas yra žmogaus veidas, ta�iau ji tikriausiai sugrupuos nuotraukas
su žmonių veidais ir tarkim peizažais į skirtingas grupes. Kitaip tariant,
nuotraukų su žmonių veidais vidinė struktūra mūsų mokymosi be mokytojo programai
bus nepanaši į nuotraukų su peizažais vidinę struktūrą, todėl ji į vieną grupę
sudės nuotraukas, kurios jai atrodo tarpusavyje panašiausios: vienoje
grupėje nuotraukos su žmonių veidais, o kitoje su gamtos peizažais.

Abu mokymo procesai yra panašūs savo esme (siekia išgauti žinias apie turimus
duomenis), bet jų panaudojimas skiriasi iš esmės (mokymo su mokytoju atveju mes
kuriame modelį apibūdinantį kaip buvo sukurti mokymo duomenys, kad galėtume
spėti naujų objektų savybes, o mokymo be mokytojo atveju siekiame susipažinti
su vidine mokymo duomenų struktūra, kai nėra kaip pamatuoti ar geri ar blogi
klasteriai buvo rasti).

%% JG: aš nesutinku, kad abiem procesais siekiama tų pa�ių tikslų. Vienu atveju 
% siekiama išmokti iš pavyzdžių. Kitu atveju siekiama atrasti nežinomas
% struktūras turimuose duomenyse. Procesai yra panašūs savo esme, bet jų 
% panaudojimas skiriasi iš esmės.

%% JG: iš vikipedijos: In machine learning, unsupervised learning refers to the 
% problem of trying to find hidden structure in unlabeled data. Since the
% examples given to the learner are unlabeled, there is no error or reward
% signal to evaluate a potential solution. This distinguishes unsupervised 
% learning from supervised learning and reinforcement learning.

%% JG: visą šitą skyrių reikia pateikti koncentruotai. Esminiai teiginiai ir grafiniai pavyzdžiai. 

%% DJ: Turiu pripažint, kad šitam pavyzdyje prigrybavau stipriai. Nurašinėjau
% pavyzdį kur prastai paaiškino skirtumą, bet užtat man pavyzdys patiko. Dabar
% labiau į temą surašyta.


\subsection{Atraminių vektorių klasifikatoriai}

Atraminių vektorių klasifikatoriai (angl. \textit{support vector machines}, SVM) - tai mašininio mokymosi  algoritmas, kuris gali būti taikomas tiek klasifikavime, tiek regresinėje analizėje. Jis priskiriamas prie mokymosi su mokytoju algoritmų \cite{vapnik2000nature}.

Atraminių vektorių klasifikatorių algoritmo idėja yra duomenų vektorių erdvėje surasti hiperplokštumą (sprendimo ribą (angl. \textit{decision boundary})), kurios atstumas nuo skirtingoms klasėms priklausančių objektų būtų didžiausias, galimai pašalinant triukšmą bei išimtis. Kitaip tariant, yra ieškoma hiperplokštuma, kuri geriausiai atskiria objektus priklausančius skirtingoms klasėms. 

Tarkime, kad turime $L$ mokymosi objektų, kurių kiekvienas objektas $x_i$ turi $D$ matų ir priklauso vienai iš dviejų klasių $y_i=-1$ arba $y_i=+1$. Taigi turime mokymosi duomenis, kurių pavidalas yra:
\begin{equation}
 \{x_i, y_i\}, kur\; i=1..L, y_i \in \{-1,1\}, x \in \Re^D
\end{equation}
Tarkime, kad duomenys yra tiesiškai atskiriami. Tai reiškia, kad galima nupiešti tiesę grafe $x_1$ ir $x_2$, kuri atskiria dvi klases, kai $D=2$ ir hiperplokštumą grafuose $x_1, x_2,...x_D$, kai $D > 2$. Hiperplokštuma apibrėžta $w\cdot x + b = 0$, kur $w$ -- hiperplokštumos normalės vektorius, $\frac{b}{||w||}$ -- statmens einančio nuo hiperplokštumos iki koordinačių pradžios taško ilgis.

Atraminiai vektoriai (angl. \textit{support vectors}) yra duomenų objektai esantys arčiausiai atskiriančiosios hiperplokštumos. Atraminių vektorių klasifikatorių algoritmo tikslas yra orientuoti hiperplokštumą tokiu, būdu, kad atstumas tarp jos ir artimiausių objektų iš abiejų klasių \cite{cortes1995support}. Tai pavaizduota ~\ref{fig:support_vector_machines} pav. Taigi, atraminių vektorių klasifikatorių sukūrimas yra parametrų $w$ ir $b$ tenkinančių minėtas sąlygas radimas. Tai galima užrašyti tokia nelygybe:
\begin{equation}
 \label{svm_separable}
 y_i(x_i \cdot w + b) - 1 > 0
\end{equation}
Jei abiejų klasių objektai nėra tiesiškai atskiriami, reikia ,,atpalaiduoti'' (\ref{svm_separable}) salygą:
\begin{equation}
 \label{svm_non_separable}
 y_i(x_i \cdot w + b) - 1 + \xi_i > 0, kur\; \xi_i \geq 0, \;  \forall_i,
\end{equation}
kur $\xi_i$ yra baudos už neteisingai klasei priskirtą objektą dydis.
\begin{figure}
 \centering
 \includegraphics[width=.7\textwidth]{images/support_vector_machines.jpg}
 \caption{Hiperplokštuma nubrėžta per dvi tiesiškai atskiriamas klases.}
 \label{fig:support_vector_machines}
\end{figure}

Atraminių vektorių klasifikatoriai gerai tinka uždaviniams, kai turima labai maža mokymosi duomenų aibė. Biomedicininiai duomenys ir pasižymi tuo, kad mokymosi duomenų aibė yra maža palyginus su turimų matų skaičiumi. Todėl atraminių vektorių klasifikatorių naudojimas dirbui su biomedicininiais duomenimis yra tapęs standartiniu pasirinkimu. 


%% JG: cituoti turi originalų darbą:
%% JG: C. Cortes and V. Vapnik, Support-Vector Networks, Machine Learning, 20(3):273-297, September 1995
%% JG: Vladimir N. Vapnik. The Nature of Statistical Learning Theory. Springer, New York, 1995


%SVM is a type of machine learning algorithm derived from statistical learning
%[theory](http://download.oracle.com/docs/cd/B14117_01/text.101/b10729/classify.htm).

%% JG: nepamiršksio daugiamatiškumo erdvę, o ten juos galima atskirti tiesiškai.

\subsection{\textit{Random Forest} klasifikatorius}

\textit{Random Forrest} klasifikatorius yra įrankis, kuris kuris sukuria eilę klasifikavimo medžių (angl. \textit{decision tree}), kurie visi nepriklausomai klasifikuoja mėginius, ir daugumos balsavimo (angl. \textit{majority voting}) būdų yra skelbiamas galutinis klasifikavimo rezultatas \cite{breiman1984classification}. Toks daugelio klasifikatorių panaudojimas yra vadinamas kombinuotoju mokymusi (angl. \textit{ensemble learning}). 

Kiekvienas klasifikavimo medis yra konstruojamas pagal algoritmą:

\begin{algorithm}
 \caption{\textit{Random Forest} klasifikavimo medžių konstravimas}
 \label{random_forest_algorithm}
 \begin{enumerate}
  \item Turimas $N$ mėginių, kurie turi $M$ matų;
  \item Pasirenkamas $m$ matų, kurie bus naudojami klasifikavimo medžių kūrimui; $m << M$;
  \item Sudaroma treniravimosi mėginių aibė $n$ kartų pasirenkant mėginius su pasikartojimais iš visų $N$ mėginių. Visi nepasirinkti mėginiai paliekami klasifikatoriaus testavimui; 
  \item Kiekvienam medžio mazgui atsitiktinai pasirinkama $m$ matų, kuries sudarys sąlygą tam mazgui. Randamas geriausia atskyrimo sąlyga treniravimos duomenims pagal tuos $m$ matų;
  \item Pilnai užauginti medžiai nėra genėjami (angl. \textit{pruning}).
 \end{enumerate}
\end{algorithm}

\textit{Random forest} algoritmo tikslumas priklauso nuo: koreliacijos tarp sukurtų klasifikavimo medžių (didesnė koreliacija lemia mažesnį klasifikavimo tikslumą.); atskiro klasifikavimo medžio skiriamoji galia (kuo didesnė atskiro klasisifikavimo medžio skiriamoji galia, tuo geresnis klasifikavimo tikslumas).

\textit{Randon forest} klasifikatoriai yra tikslūs, greiti, bei sugeba išvengti persimokymo (angl. \textit{overfitting}). Šios trys klasifikavimo algoritmo savybės yra labai svarbios dirbant su biomedicininiais duomenimis. 

\subsection{Matų atrinkimo stabilumas}

Matų atrinkimo metodų stabilumas gali būti apibrėžtas kaip matų atrinkimo rezultatų variacijos dėl mažų pakeitimo duomenų rinkinyje. Pekeitimai duomenų rinkinyje gali būti duomenų objektų lygio (pvz. pridedami ar atimami duomenų objektai), matų lygio (pvz. pridedant matams triukšmo) ar abiejų lygių kombinacija.

Matų atrinkimo metodų stabilumas yra vis daugiau dėmesio gaunanti tyrimų kryptis. Stabilumo aktualumas yra sąlygotas to, kad biologiniuose duomenyse galima daryti prielaidą, kad konkrečiai problemai yra aktualūs tik tam tikri matai. Todėl dalykinės srities ekspertams yra aktualu naudoti tuos matų atrinkimo metodus, kurie yra stabilūs ir susiję su modeliuojama problema, nes tai atpigina tolimesnę duomenų analizę. 

Svarbu paminėti, kad matų stabilumas nėra matuojamas visiškai nepriklausomai -- jis yra matuojamas atsižvelgiant į klasifikavimo rezultatus. Matuoti stabilumą verta tada, kai atrinkti matai duoda gerus klasifikavimo rezultatus. Kitaip tariant, nėra naudingi tie matų atrinkimo metodai, kurie duoda labai stabilius rezultatus, bet jais remiantis atrinktais matais pavyksta sukurti tik atsitiktinius rezultatus duodančius klasifikatorius.

\subsubsection{Stabilumo matavimas}

Vertinant matų atrinkimo metodų stabilumą yra svarbu kaip panašiai yra atrenkami matai, kai yra atliekamas matų atrinkimas su vis kitu mėginių ar matų poaibiu. Kuo mažiau skiriasi atrinktoji matų aibė darant pakeitimus duomenyse, tuo matų atrinkimo stabilumas yra didesnis. Vidutinis matų atrinkimo stabilumas gali būti apibrėžtas kaip vidurkis visų reitingavimo metu gautų sąrašų porų tarpusavio panašumo įverčių \cite{kalousis2007stability}:
\begin{equation}
 S_{tot}=\frac{2\sum_{i=1}^{k-1}\sum_{j=i+1}^{k} S(f_i, f_j)}{k*(k-1)},
\end{equation} 
kur $k$ žymi kiek kartų buvo imtas skirtingas poaibis objektų matų atrinkimui,
$f_i$, $f_j$ -- matų atrinkimo rezultatas -- reitingai, 
$S(f_i, f_j)$ -- yra aibių panašumo įvertinimo funkcija.

Matų atrinkimo stabilumo įvertis priklauso nuo to, kokią aibių panašumo funkciją naudosime. Tradicinės panašumo funkcijos (persidengimo procentas, Pearson'o koreliacija, Spearman'o koreliacijoa) gali būti taikomos, bet jos yra linkusios priskirti didesnes panašumo reikšmes, kai pasirenkamas didesnis matų poaibis. Taip yra dėl padidėjusio sisteminio nuokrypio (ang. bias), nes imant didesnį poaibį padidėja tikimybė tiesiog atsitiktinai pasirinkti matą.

\subsubsection{\textit{Kuncheva} indexas}

\textit{Kuncheva} indexas \cite{DBLP:conf/aia/Kuncheva07} yra funkcija skirta matuoti aibių panašumui. Ši funkcija gerai tinka matuoti matų atrinkimo atabilumą, nes atsižvelgia į paimto matų poaibio dydį. \textit{Kuncheva} indeksas:
\begin{equation}
\label{kuncheva_index}
 KI(f_i, f_j)=\frac{r*N - s^2}{s*(N-s)}=\frac{r - (s^2/N)}{s - (s^2/N)},
\end{equation}		
kur $s=|f_i|=|f_j|$ yra atrinktų matų aibės dydis, $r=|f_i \bigcap f_j|$ - abiems atrinktiems matų poaibiams bendrų matų skaičius, $N$ - bendras  duomenų aibės matų skaičius. Pastebėtina, kad formulėje esantis atėminys $s^2/N$ ištaiso sisteminį nuokrypį atsirandantį dėl galimybės atsitiktinai pasirinkti matus. 

Kunchevos indeksas gali įgyti reikšmes iš intervalo $[-1, 1]$, kur didesnė reikšmė reiškia didesnį panašumą, o artimos nuliui reikšmės reiškia, kad matai atrenkami daugiausia atsitiktinai. Kunchevos indekso ypatybė yra ta, kad jis atsižvelgia tik į persidengiančias, tačiau visiškai nekreipia dėmesio į koreliuojančius matus.

\subsubsection{\textit{Jaccard} indeksas}

Vienas paprasčiausi aibių panašumo įverčių yra \textit{Jaccard} indeksas \cite{jaccard1901etude}. \textit{Jaccard} indexas yra santykis tarp aibių sankirtos ir aibių sąjungos:
\begin{equation}
\label{jaccard_index}
 JI(f_i, f_j)=\frac{|f_i \bigcap f_j|}{|f_i \cup f_j|}=\frac{\sum_{l}I(f_i^l=f_j^l=1)}{\sum_{l}I{f_i^l+f_j^l > 0)}}, 
\end{equation}
kur $f_i$ ir $f_j$ yra dimensijų reitingai, $I(x)$ - funkcija grąžinanti 1, jei $x=TRUE$, ir 0 kitu atveju.

\subsubsection{\textit{Hamming} atstumas}

Informacijos teorijoje \textit{Hamming} atstumas \cite{hamming1950error} tarp dviejų vienodo ilgio vektorių yra apibrėžtas kaip pozicijų skaičius, kuriose esantys simboliai nesutampa. Kitaip tariant, \textit{Hamming} atstumas yra minimalus skaičius pakeitimų, kad vieną vektorių padarytume lygų kitam. 
\begin{equation}
\label{hamming_distance}
 Hamming Distance(X, Y)= \sum_{i=1}^{n} (x_i \oplus y_i),
\end{equation}
kur $\oplus$ - sumos moduliu 2 arba XOR operacija.

Šiuo metodu matuojant dimensijų atrinkimo stabilumą, prieš atstumo matavimą reikia atlikti pertvarkymus. Pirma, iš atrinktų dimensijų vektorių padaryti bendro dimensijų skaičiaus ilgio binarinius vektorius. Antra, vienetukus sudėti tose vektoriaus elementuose, kurių indeksus gavome dimensijų atrinkimo metodu. Tada jau galima matuoti atstumą tarp dviejų dimensijų atrinkimo rezultatų.

\newpage

\section{PAGRINDINIAI DIMENSIJŲ ATRINKIMO METODAI}

Biomedicininių duomenų kontekste galima daryti prielaidą, kad ne visi matai yra susiję su tiriama problema, pvz. gaubtinės žarnos vėžiu, dėl tokių faktorių, kaip triukšmas duomenyse. Paprastai nagrinėjamai problemai svarbus yra mažas, palyginus su visu, matų kiekis. Ši biomedicininių duomenų ypatybė veda prie ,,daugiamatiškumo prakeiksmo`` (angl. \textit{the curse of dimentionality}) \cite{bellman1966adaptive} - didėjant matų kiekiui mėginiai pasidaro panašūs, todėl bandymas juos klasifikuoti tolygus spėliojimui. Vienas iš būdų kovoti su ,,daugiamatiškumo prakeiksmu`` yra naudoti dimensijų atrinkimo metodus. Dimensijų atrinkimas yra svarbus etapas biomedicininių duomenų pirminiam apdorojimui (angl. \textit{preprocessing}). Dimensijų atrinkimas dažniausiai yra naudojamas surasti mažiausią dimensijų poaibį, kuris maksimaliai pagerina klasifikatoriaus tikslumą.

Pagal tai, kaip dimensijų atrinkimo metodai yra susiję su klasifikatoriumi, dimensijų atrinkimo metodus galima skirstyti į tris kategorijas \cite{saeys2008robust}:
\begin{enumerate}
 \item Filtruojantys metodai (angl. \textit{filter methods}), pvz. \textit{Fisher} įvertis. Jie dirba tiesiogiai su duomenimis, o jų rezultatas gali būti dimensijų įvertinimas svoriais, dimensijų reitingavimas ar tiesiog geriausių dimensijų poaibis, kuriuo remiantis vėliau apmokomas klasifikatorius. Tokių metodų pagrindinis privalumas yra tai, kad jie yra greiti, tinka paskirstytų skaičiavimų aplinkoms ir nepriklausomi nuo klasifikavimo  metodo, tačiau remiantis atrinktosiomis dimensijomis nebūtinai bus sukurtas geriausias klasifikatorius.
 \item Prisitaikantieji metodai (angl. \textit{wrapper methods}). Pirma, apmokomas klasifikatorius su visomis dimensijomis, antra, parenkamas dimensijų poaibis ir apmokomas klasifikatorius, tada po daugkartinio dimensijų aibių įvertinimo pagal klasifikavimo rezultatus yra nusprendžiama, kuris dimensijų poaibis yra labiausiai tinkamas klasifikavimui. Įterptinių metodų atveju dimensijų atrinkimo procesas yra neatsiejamas nuo klasifikavimo proceso - pats klasifikatorius įvertina dimensijas. Jie dažnai duoda geresnius rezultatus negu filtravimo metodai, bet yra reiklūs resursams.
 \item Įterptiniai metodai (angl. \textit{embedded methods}), pvz. AW-SVM\cite{vapnik2000nature}. Jie dimensijų atrinkimui naudoja vidinius klasifikatoriaus duomenis (pvz. dimensijų svoriai gauti pagal SVM). Šie metodai dažnai siūlo gerą santykį tarp klasifikavimo tikslumo ir skaičiavimų sudėtingumo.
\end{enumerate}

Šiame skyriuje nagrinėsiu pagrindinius dimensijų atsirinkimo metodus:
\begin{enumerate}
 \item \textit{Fisher} įvertis (angl. \textit{Fisher ratio})\cite{Pavlidis:2001:GFC:369133.369228};
 \item \textit{Relief} metodas\cite{DBLP:journals/ml/Robnik-SikonjaK03};
 \item Asimetrinis priklausomybės koeficientas\cite{Shannon:2001:MTC:584091.584093} (angl. \textit{Asymmetric Dependency Coefficient, ADC});
 \item Absoliučių svorių SVM\cite{vapnik2000nature} (AW-SVM) (angl. \textit{Absolute Weight SVM})
 \item Rekursyvus dimensijų eliminavimas pagal SVM\cite{Guyon:2002:GSC:599613.599671} (SVM-RFE) (angl. \textit{Recursive Feature Elimination by SVM})
\end{enumerate}

\subsection{\textit{Fisher} įvertis}

\textit{Fisher} įvertis vertina individualias dimensijas pagal dimensijos klasių atskiriamąją galią. Dimensijos įvertis yra sudarytas iš tarpklasinio skirtumo santykio su vidiniu klasės pasiskirstymu:
\begin{equation}
 FR(j) = \frac{(\mu_{j1} - \mu_{j2})^2}{\sigma_{j1}^2 + \sigma_{j2}^2},
\end{equation}
kur, 
$j$ - yra dimensijos indeksas, 
$\mu_{jc}$ - dimensijos $j$ reikšmių vidurkis klasėje $c$, 
$\sigma_{jc}^2$ - dimensijos $j$ reikšmių standartinis nuokrypis klasėje $c$, kur $c={1,2}$. Kuo didesnis yra \textit{Fisher} įvertis, tuo geriau ta dimensija atskiria klases.

\subsection{\textit{Relief} metodas}

\textit{Relief} metodas iteratyviai skaičiuoja dimensijų ,,susietumą``. Pradžioje
,,susietumas`` visoms dimensijoms yra lygus nuliui. Kiekvienoje
iteracijoje atsitiktinai\footnote{Pastebėtina, kad dėl atsitiktinumo faktoriaus klasifikavimo ir  dimensijų atrinkimo stabilumo
rezultatai varijuoja.} pasirenkamas objektas iš mėginių aibės, surandami
artimiausi kaimynai iš tos pačios ir kitos klasės, ir atnaujinamos visų 
dimensijų ,,susietumo`` reikšmės. Dimensijos įvertis yra vidurkis visų objektų
atstumų iki artimiausių kaimynų iš tos pačios ir kitos klasės:
\begin{equation}
 W(j)=W(j) - \frac{diff(j, x, x_H)}{n} + \frac{diff(i, x, x_M)}{n},
\end{equation}
kur 
$W(j)$ - $j$-osios dimensijos ,,susietumo`` įvertis, 
$n$ - mėginių aibės dydis, 
$x$ - atsitiktinai pasirinktas mėginys, 
$x_H$ - artimiausias $x$ kaimynas iš tos pačios klasės (angl. \textit{nearest-Hit}), 
$x_M$ - artimiausias $x$ kaimynas iš kitos klasės(angl. \textit{nearest-Miss}),
$diff(j, x, x')$ - $j$-osios dimensijos reikšmių skirtumas tarp laisvai pasirinkto objekto $x$ ir atitinkamo kaimyno, kur skirtumą į intervalą $[0, 1]$ normalizuojanti funkcija yra:
\begin{equation}
 diff(j, x, x')=\frac{|x_j- x_j'|}{x_{j_{max}} - x_{i_{min}}},
\end{equation}
kur $x_{j_{max}}$ ir $x_{j_{min}}$ yra maximali ir minimali $j$-osios dimensijos reikšmės. ,,Susietumo`` reikšmių atnaujinimas yra vykdomas $n$ kartų ir kuo didesnė galutinė reikšmė, tuo svarbesnė dimensija. Pastebėtina, kad aprašyta algoritma versija yra skirta dirbti su dviejų klasių atveju, tačiau yra ir multiklasinis algoritmo variantas.

\subsection{Asimetrinis priklausomybės koeficientas}

Asimetrinis priklausomybės koeficientas (ADC) yra dimensijų reitingavimo motodas, kuris matuoja klasės $Y$ etiketės (angl. \textit{label}) tikimybinę priklausomybę $j$-ąjai dimensijai, naudodamas informacijos prieaugį \cite{kent1983information} (angl. information gain):
\begin{equation}
 ADC(Y, j) = \frac{MI(Y, X_j)}{H(Y)},
\end{equation}
kur $H(Y)$ - klasės $Y$ entropija \cite{Shannon:2001:MTC:584091.584093}, o $MI(Y, X_j)$ - yra bendrumo informacija \cite{Shannon:2001:MTC:584091.584093} (angl. mutual information) tarp klasės etiketės $Y$ ir $j$-osios dimensijos
\begin{equation}
 H(Y)=-\sum_y{p(Y=y)log{p(Y=y)}}, 
\end{equation}
\begin{equation}
 H(X_j)=-\sum_x{p(X_j=x) log{p(X_j=x)}},
\end{equation}
\begin{equation}
 MI(Y, X_j) = H(Y) + H(X_j) - H(Y, X_j),
\end{equation}
\begin{equation}
 H(Y, X_j) = -\sum_{y,x_j}{p(y, x_j)log{p(y, x_j)}},
\end{equation}
Kuo didesni ADC įverčiai, tuo dimensija yra svarbesnė, nes turi daugiau informacijos apie mėginių klases.

\subsection{Absoliučių svorių SVM}

Atraminių vektorių metodas (SVM) yra vienas populiariausių klasifikavimo algortimų, nes jis gerai susidoroja su daugiamačiais duomenimis \cite{guyon2002gene}. Yra keletas bazinių SVM variantų \cite{vapnik2000nature}, bet šiame darbe naudosime tiesinį SVM, nes jis demonstruoja gerus rezultatus analizuojant genų ekspresijos duomenimis. Tiesinis SVM yra hiperplokštuma apibrėžta kaip:
\begin{equation}
 \sum_{j=1}^{p}{w_jx_j + b_0 = 0},
\end{equation}
kur $p$ - dimensijų kiekis, $w_j$ - j-osios dimensijos svoris, $x_j$ - j-osios
dimensijos kintamasis, $b_0$ - konstanta. Dimensijos absoliutus\footnote{Svorį
reikia imti absoliutaus dydžio, nes neigiamas svoris implikuoja priklausomybę 
vienai klasei, o teigiamas kitai klasei.} svoris $w_j$ gali būti panaudotas
dimensijų reitingavimui. Pastebėtina, kad svorių nustatymas yra atliekamas tik 
vieną kartą\footnote{SVM-RFE dimensijų atrinkimo metodas svorius nustato daug kartų.}.

\subsection{Rekursyvus dimensijų eliminavimas pagal SVM}

Rekursyvus dimensijų eliminavimas pagal SVM \cite{guyon2002gene} yra vienas populiariausių dimensijų
atrinkimo algoritmų. Todėl, jis yra naudojamas, kaip atskaitos taškas (angl. \textit{benchmark})
vertinant kitus dimensijų atrankos metodus. Iš esmės šis metodas yra daugkartinis 
absoliučių svorių SVM metodo taikymas nuolat išmetinėjant dimensijas su 
mažiausiais svoriais. Rekursyvus dimensijų eliminavimas mums padeda surasti 
tinkamą dimensijų poaibį, kas nevisada pavyksta su dimensijų reitingavimo 
metodais. Bendroji rekursyvaus dimensijų eliminavimo procedūra:
\begin{algorithm}
\caption{Rekursyvus dimensijų eliminavimas}
\label{RFE}
 \begin{enumerate}
 \item Turime pilną dimensijų rinkinį $F_0$, nustatome $i=0$;
 \item Įvertiname kiekvienos dimensijos kokybę dimensijų aibėje $F_i$;
 \item Išmetame mažiausiai kokybišką dimensiją iš $F_I$ tam, kad gautume
 dimensijų rinkinį $F_{i+1}$;
 \item Nustatome $i=i+1$ ir grįžtame į antrąjį žingsnį kol nėra patenkinta 
 algoritmo pabaigos sąlyga.
\end{enumerate}
\end{algorithm}
Jei trečiajame algoritmo žingsnyje iš dimensijų aibės yra pašalinama tik viena dimensija, tai gauname dimensijų reitingavimą, o jei pašalinamas fiksuotas skaičius ar dalis (pvz. 50\%) dimensijų, tai dimensijų reitingavimo negauname. Pastebėtina, kad rekursyvus dimensijų eliminavimas labai padidina algoritmo sudėtingumą. Algoritmo pabaigos sąlyga gali būti koks nors konkretus dimensijų skaičius arba tiesiog dimensijų aibę mažinti tol, kol dimensijų visai nebeliks.

\section{STABILIŲ DIMENSIJŲ ATRINKIMO METODAI}
\label{stabiliu_dimensiju_atrinkimo_metodai}

Naudodami dimensijų atrinkimo metodus, biomedicininius duomenis tiriantys mokslininkai susiduria su atrinktųjų dimensijų aibės stabilumo problema - atrenkant dimensijas pagal kitą mėginių poaibį, gaunamas kitas dimensijų poaibis. Dimensijų atrinkimo nestabilumas yra sąlygotas šių veiksnių:
\begin{enumerate}
 \item Duomenys yra triukšmingi ir kai kurios dimensijos gali būti palaikytos informatyviomis grynai dėl atsitiktinių priežasčių;
 \item Daugiamačiuose duomenyse tikėtina, kad dalis dimensijų koreliuoja, todėl, kuri iš koreliuojančių dimensijų bus pasirinkta, priklauso nuo to, kuriuos mėginius pasirinksime klasifikatoriaus apmokymui;
 \item Kiekvienas dimensijų atrinkimo algoritmas daro skirtingas prielaidas apie tai, kurios dimensijos yra informatyvios.
\end{enumerate}
Galime teigti, kad skirtingi metodai tiems patiems duomenims gali atrinkti skirtingas dimensijas. Taip pat, suskaidžius turimus duomenis į atskiras persidengiančias aibes ir atrinkus tą patį kiekį dimensijų tuo pačiu metodu, gaunamos skirtingos dimensijų aibės. Be to, kuo triukšmingesni duomenys, kuo mažiau turima mėginių ir kuo daugiau yra dimensijų, tuo ryškesnė yra ši problema \cite{loscalzo2009consensus}. 

Vienas iš būdų didinti dimensijų atrinkimo stabilumą yra naudoti multikriterinius dimensijų atrinkimo metodus. Jų esmė yra panaudoti kelis dimensijų atrinkimo metodus suliejant jų rezultatus į vieną bendrą rezultatą. Yra skiriamos trys priežastys, kodėl keletas agreguotų silpnų ir nestabilių dimensijų atrinkimo metodų gali duoti stabilesnius dimensijų atrinkimo rezultatus \cite{dietterich2000ensemble}:
\begin{enumerate}
 \item Keletas skirtingų, bet vienodai optimalių hipotezių gali būti teisingos, ir kriterijų agregavimas sumažiną tikimybę, kad bus pasirinkta neteisinga  hipotezė;
 \item Atskiri dimensijų metodai gali dirbti skirtinguose lokaliuose optimumuose, tuo tarpu agregavimas gali geriau reprezentuoti tikrąją  duomenis generuojančią funkciją;
 \item Tikroji duomenų funkcija negali būti reprezentuojama jokia hipoteze paskiro algoritmo hipotezių erdvėje ir agreguojant pavienių metodų rezultatus galima praplėsti hipotezių erdvę.
\end{enumerate}
Apibendrinant galima sakyti, kad suliejant keletą skirtingų dimensijų atrinkimo metodų rezultatų suliejamos gerosios pavienių dimensijų atrinkimo metodų savybės, taip bandant kompensuoti tų algoritmų silpnybes.

Šiame skyriuje aptarsiu dimensijų atrinkimo stabilumą didinančius metodus:
\begin{enumerate}
 \item Svoriais grįstas multikriterinis suliejimas;
 \item Reitingais grįstas multikriterinis suliejimas;
 \item Svoriais ir reitingais grįstas multikriterinis suliejimas;
 \item Multikriterinis rekursyvus dimensijų eliminavimas;
 \item Konsensuso grupėmis grįstas stabilių dimensijų atrinkimo metodas.
\end{enumerate}

\subsection{Svoriais grįstas multikriterinis suliejimas}

Svoriais grįsto multikriterinio dimensijų atrinkimo suliejimo pagal svorius algoritmo
pirmajame žingsnyje kiekvienas bazinis metodas priskiria duomenų rinkinio
dimensijoms svorius, tada tie svoriai yra kombinuojami į vieną sutarties
(angl. consensus) svorių vektorių, kurio pagrindu yra gaunami dimensijų 
reitingai. Algoritmas yra pavaizduotas ~\ref{fig:figure4} pav.
\begin{figure}
 \centering
 \includegraphics[width=1\textwidth]{images/score_based_fusion.pdf}
 \caption{Svoriais grįstas multikriterinis suliejimas.}
 \label{fig:figure4}
\end{figure}
\begin{figure}[ht]
\begin{minipage}[b]{0.5\linewidth}
\centering
\includegraphics[width=1\textwidth]{images/boxplot_colon_all.png}
\caption{Pavienių dimensijų atrinkimo metodų nenormalizuotas svorių
pasiskirstymas.}
\label{fig:figure1}
\end{minipage}
\hspace{0.5cm}
\begin{minipage}[b]{0.5\linewidth}
\centering
\includegraphics[width=1\textwidth]{images/boxplot_colon_all_normalized.png}
\caption{Pavienių dimensijų atrinkimo metodų normalizuotas svorių
pasiskirstymas.}
\label{fig:figure2}
\end{minipage}
\end{figure}

Suliejant svorius svarbu yra užtikrinti, kad svoriai, gauti naudojant
skirtingus bazinius kriterijus, būtų palyginami. Todėl svorių normalizavimas
turi būti atliekamas prieš svorių kombinavimą. Kitu
atveju dimensijų įvertinimo metodai bus nepalyginami. Paveikslėlyje ~\ref{fig:figure1} pav.
nenormalizuotų pavienių dimensijų vertinimo metodų skiriasi netgi suteiktų
svorių intervalai. Paveikslėlyje ~\ref{fig:figure2} pav. matome,
kad net ir normalizavus svorius gana stipriai skiriasi svorių kvartiliai - į 
tai reikia atkreipti dėmesį interpretuojant galutinius dimensijų vertinimo 
rezultatus. Šiame darbe svoriai yra 
normalizuoti intervale $[0, 1]$ pagal formulę:
\begin{equation}
 u_i'=\frac{u_i - u_{i_{min}}}{u_{i_{max}} - u_{i_{min}}}, 
\end{equation}
kur $u_i$ - dimensijų svorių vektorius pagal $i$ kriterijų, 
$u_{i_{min}}$ - minimali $u_i$ svorių vektoriaus reikšmė,
$u_{i_{max}}$ - maksimali $u_i$ svorių vektoriaus reikšmė,
$u_i'$ - normalizuotų svorių vektorius.

Sutarties svorių vektorius $u$ yra vidurkis normalizuotų svorių vektorių:
\begin{equation}
 u = \frac{1}{m}\sum_{i=1}^m u_i',
\end{equation}
kur $m$ yra bazinių kriterijų skaičius. Reikia paminėti, kad didesnė svorio
reikšmė reiškia, kad dimensija yra geresnė.

\subsection{Reitingais grįstas multikriterinis suliejimas}

Reitingais grįsto multikriterinio suliejimo pagal reitingus metodas gauna
duomenų rinkinio dimensijų reitingą,
pagal keletą bazinių dimensijų reitingavimo kriterijų. Algoritmo pirmajame žingsnyje
keletas dimensijų atrinkimo kriterijų grąžina dimensijų reitingu, paskui tie
reitingai yra kombinuojami į vieną bendra dimensijų reitingą.  Algoritmas yra
pavaizduotas ~\ref{fig:figure5} pav.
\begin{figure}
 \centering
 \includegraphics[width=1\textwidth]{images/ranking_based_fusion.pdf}
 \caption{Reitingais grįstas multikriterinis suliejimas.}
 \label{fig:figure5}
\end{figure}
Suliejimo pagal reitingus metodas nereikalauja dimensijų atrinkimo metodų 
rezultatų normalizavimo, nes tiesiog imame dimensijoms priskirtus reitingus ir 
juos kombinuojame. Skirtingai nei suliejimo pagal svorius algoritme, baziniai 
dimensijų atrinkimo kriterija dimensijų eliminavimas\cite{yang2011robust} susideda iš dviejų
dalių: keletos dimensijų atrinkimi turi gražinti dimensijų reitingus, o ne svorius.

Dimensijų reitingų kombinavimui yra keletas metodų\cite{dwork2001rank}, tačiau
paprastumo dėlei šiame darbe naudosiu Borda balsavimą\footnote{Dar žinomas kaip
,,Pažymių metodas``. Jis buvo pasiūlytas prancūzų matematiko ir fiziko 
Jean-Charles de Borda 1770 metais.} (angl. Borda count). Tarkime, kad turime
$m$ basuotojų ir $p$ kandidatų aibę. Tada Borda balsavimo metodas kiekvienam
$i$-ajam balsuotojui sukuria balsų vektorių $v_i$ tokiu būdu: geriausiai 
įvertintam kandidatui suteikiama $p$ taškų, antrajam kandidatui $p-1$, ir t.t.
Galutiniai taškai yra gaunami sudedant visų balsuotojų taškus
\begin{equation}
 v = \sum_{i=1}^m v_i,
\end{equation}
kur $v$ yra suminių taškų vektorius, o iš jo galime gauti ir dimensijų reitingus.

\subsection{Svoriais ir reitingais grįstas multikriterinis suliejimas}

Svoriais ir reitingais grįsto multikriterinio suliejimo metodas
nuo reitingais grįsto multikriterinio suliejimo metodo skiriasi tuo, kad kaip dar vienas 
reitingas yra panaudojamas svoriais grįsto multikriterinio dimensijų atrinkimo metu
gautas reitingas.
Multikriterinio dimensijų įverčių ir pagal svorius, ir pagal reitingus metodas vyksta trimis
žingsniais:
\begin{enumerate}
  \item Gauname dimensijų reitingus pagal $m$ pavienių dimensijų atrinkimo motodų;
  \item Suliejame dimensijų įverčius pagal svorius ir taip gauname vieną 
  dimensijų reitingą;
  \item Reitinguojame dimensijas pagal visus turimus $m+1$ pavienius reitingus.
\end{enumerate} 
Algoritmas yra pavaizduotas ~\ref{fig:figure3} pav.
\begin{figure}
 \centering
 \includegraphics[width=1\textwidth]{images/score_and_ranking_based_fusion.pdf}
 \caption{Svoriais ir reitingais grįstas multikriterinis suliejimas.}
 \label{fig:figure3}
\end{figure}

Kadangi yra suliejami keli mažai koreliuojantys dimensijų reitingavimo metodai,
yra pasiekiamas didesnis dimensijų atrinkimo stabilumas, kai varijuoja 
treniravimosi duomenų poaibis (angl. subsampling).

\subsection{Multikriterinis rekursyvus dimensijų eliminavimas}

Jei dimensijų atrinkimo tikslas yra pagerinti klasifikavimo rezultatus, tai taikymas
multikriterinio dimensijų atrinkimo metodų nebūtinai duos pageidaujamą rezultatą,
nes yra pastebėta, kad vien dimensijų reitingavimas nebūtinai suranda geriausią dimensijų 
poaibį. Tam, kad būtų surastas geriausias dimensijų poaibis reikia kombinuoti
multikriterinį dimensijų reitingavimą su paieškos strategija. Rekursyvus 
dimensijų eliminavimas yra dažnai naudojama paieškos strategija dimensijų
atrinkimui. Todėl yra kombinuojamas multikriterinis dimensijų reitingavimas ir
rekursyvus dimensijų eliminavimas.

Multikriterinis rekursyvus dimensijų eliminavimas\cite{yang2011robust} susideda iš dviejų
dalių: keletos dimensijų atrinkimo kriterijų suliejimo ir pagal svorius, ir 
pagal reitingus, ir rekursyvaus dimensijų eliminavimo aprašyto algoritme 
nr. \ref{RFE}. Algoritmas pavaizduotas ~\ref{fig:figure6} pav.
\begin{figure}
 \centering
 \includegraphics[width=0.7\textwidth]{images/mcf-rfe_procedure.pdf}
 \caption{Multikriterinio rekursyvaus dimensijų eliminavimo algoritmas.}
 \label{fig:figure6}
\end{figure}

Yra pastebėta, kad standartinis rekursyvus dimensijų eliminavimas, kai vienos
iteracijos metu yra eliminuojama viena dimensija, gali labai padidinti 
algoritmo sudėtingumą. Todėl genų ekspresijos duomenims yra rekomenduotina
eliminuoti keletą dimensijų vienu metu.

Nors SVM-RFE dimensijų atrinkimo algoritmas ir yra labai populiarus, tačiau yra
žinoma, kad jam trūksta stabilumo. Todėl kombinuodami didesnį stabilumą turintį
multikriterinį dimensijų atrinkimą su rekursyvaus dimensijų eliminavimo paieškos
strategija, turėtume gauti stabilesnį dimensijų atrinkimo algoritmą.

\subsection{Konsensuso grupėmis grįstas stabilių dimensijų atrinkimo metodas}

Konsensuso grupėmis grįstas stabilių dimensijų atrinkimo metodas\cite{loscalzo2009consensus}, pirma, identifikuoja panašių dimensijų grupes, antra, pagal surastas grupes transformuoja dimensijų erdvę, trečia, transformuotoje dimensijų erdvėje atlieka dimensijų atrinkimą. Schematiškai šis algoritmas pavaizduotas  ~\ref{fig:figure7} pav.

\begin{figure}
 \centering
 \includegraphics[width=\textwidth]{images/consensus_group_based_feature_selection_framework.pdf}
 \caption{Konsensuso grupėmis grįstas stabilių dimensijų atrinkimas.}
 \label{fig:figure7}
\end{figure}
Loscalzo pasiūlyto metodo pagrindinė dalis yra panašių dimensijų identifikavimas. Šio uždavinio sprendimui Loscalzo naudojo \textit{Dense Group Finder} (DGF) algoritmą. DGF aprašytas algoritme nr. \ref{DGF}.
\begin{algorithm}
\caption{DGF - \textit{Dense Group Finder}}
\label{DGF}
 \begin{algorithmic}
 \item \textbf{Įeitis:} duomenys $D=\{x_i\}_{i=1}^n$, branduolio plotis $h$
 \item \textbf{Išeitis:} tankios dimensijų grupės $G_1, G_1,..., G_L$
 \For{$i = 1$ \textbf{to} $n$ \do} 
  \State Inicializuojame $j=1, y_{i,j}=x_i$
  \Repeat
    \State Suskaičiuoti $y_{i, j+1}$ pagal (\ref{for_dgf})
  \Until{konverguoja}
  \State Nustatyti atskaitos tašką $y_{i,c} = y_{i,j+1}$ (Nustatyti piką $p_i$ kaip $y_{i,c}$)
  \State Sulieti piką $p_i$ su artimiausiais pikais jei atstumai tarp jų $ < h$
 \EndFor
 \item Iš kiekvieno unikalaus piko $p_r$, pridėkime $x_i$ į $G_r$ jei $||p_r - x_i|| < h$
 \end{algorithmic}
\end{algorithm}


\begin{equation}
\label{for_dgf}
  y_{i, j+1}=\frac{\sum_{i=1}^{n} x_i K(\frac{y_j - x_i}{h})}{\sum_{i=1}^{n} K(\frac{y_j - x_i}{h})} j=1,2,...
\end{equation}
kur

\begin{algorithm}
 \caption{Konsensuso grupėmis grįstas stabilių dimensijų atrinkimas}
 \label{CGS}
 \begin{algorithmic}
   \item \textbf{Įeitis:} mėginių aibė $D$, iteracijų skaičius $t$, dimensijų atrinkimo metodas $\Phi$\
   \item \textbf{Išeitis:} atrinktos konsensuso dimensijų grupės $CG_1, CG_1,..., CG_k$
   \item // Konsensuso grupių identifikavimas
   \For{$i = 1$ \textbf{to} $n$ \do}
    \State Parinkti mėginių  poaibį $D_i$ iš $D$
    \State Gauti tankių dimensijų grupes pagal $DGF(D_i, h)$
   \EndFor
   \For{kiekvienai dimensijų porai $X_i$ ir $X_j \in D$}
    \State Nustatyti $W_{i,j}=$ dažnis kai $X_i$ ir $X_j$ yra toje pačioje grupėje $/t$
   \EndFor
   \item Sudaryti konsensuso grupes $CG_1, CG_1,..., CG_L$ atliekant hierarchinį klasterizavimą visoms dimensijoms pagal $W_{i, j}$
   \item //Dimensijų atrinkimas grįstas konsensuso grupėmis
   \For{$i = 1$ \textbf{to} $l$ \do}
    \State Parinkti reprezentatyvią dimensiją $X_i$ iš $CG_i$
    \State Įvertinti dimensijos informatyvumą $\Phi(X_i)$
   \EndFor
   \item Reitinguoti konsensuso grupes $CG_1, CG_1,..., CG_L$ pagal $\Phi(X_i)$
   \item Pasirinkti $k$ dimensijų turinčių geriausią reitingą  
 \end{algorithmic}
\end{algorithm}

\newpage

\section{PASIŪLYTAS SPRENDIMAS}

Šiame skyriuje aprašysiu siūlomą sprendimą dimensijų atrinkimo stabilumui padidinti.
\newpage

\section{EKSPERIMENTAI}
\label{eksperimentai}

Šiame skyriuje yra aprašyti daugiamačių duomenų klasifikavimo analizės metu atliktų eksperimentų rezultatai. Skyrius susideda iš eksperimentuose naudotų biomedicininių duomenų rinkinių aprašymo, eksperimentų nustatymų įvardinimo, matų atrinkimo metodų spartos matavimų įvertinimo, klasifikavimo tikslumo matavimų rezultatų pristatymo, bei stabilių matų atrinkimo rezultatų pristatymo. 

\subsection{Eksperimentuose naudoti duomenys}
\label{eksperimentuose_naudoti_duomenys}

Šiame darbe eksperimentai buvo atliekami su biomedicininiais viešai prieinamais genų ekspresijos mėginių rinkiniais. Informacija apie mėginių rinkinius pateikta \ref{table:datasets} lentelėje.
\begin{longtable}{|p{4.5cm}|p{2cm}|p{3.5cm}|p{2.3cm}|p{2cm}|}
\captionsetup{labelsep=period}
\caption{Darbe naudoti mėginių rinkiniai\label{table:datasets}}\\
%This is the header for the first page of the table...
\hline \hline
{\textbf{Pavadinimas}} &
{\textbf{Šaltinis}} &
{\textbf{Mėginių skaičius (+/-)}} &
{\textbf{Matų \newline skaičius}} &
{\textbf{OMS}}\\
\hline
\endfirsthead
%This is the header for the remaining page(s) of the table...
\multicolumn{3}{c}{{\tablename} \thetable{} -- Tęsinys} \\[0.5ex]
\hline \hline
{\textbf{Pavadinimas}} &
{\textbf{Šaltinis}} &
{\textbf{Mėginių skaičius (+/-)}}&
{\textbf{Matų \newline skaičius}} &
{\textbf{OMS}}\\
\hline
\endhead
%This is the footer for all pages except the last page of the table...
\multicolumn{3}{l}{{Lentelės tęsinys kitame puslapyje\ldots}} \\
\endfoot
%This is the footer for the last page of the table...
\hline \hline
\endlastfoot
\hline 
Gaubtinės žarnos auglys (angl. Colon) 
& 
\cite{alon1999broad} 
& 
62 (40/22) 
& 
2000 
& 
0.031 \\
\hline
Centrinės nervų sistemos auglys (CNS) 
& 
\cite{pomeroy2002prediction} 
& 
60 (39 / 21) 
& 
7129 
& 
0.0084 \\
\hline
Prostatos auglys 
& 
\cite{singh2002gene} 
& 
102 (52/50) 
& 
6033 
& 
0.0169 \\
\hline
Šizofrenija ir maniakinė depresija
&
\cite{altara}
&
90 \newline (bp\footnote{bp (angl. \textit{Bipolar disorder}) - maniakine depresija sergantys pacientai.}:
sz\footnote{sz (angl. \textit{Schizophrenia}) - šizofrenija sergantys pacientai.}:
cc\footnote{cc (angl. \textit{Control Crowd}) - kontrolinė grupė.} \newline =30:31:29)
&
22283
&
0.00404 \\
\hline
\end{longtable}

Mėginių rinkinius apibūdinantis dydis OMS (Objektų-Matų Santykis), kuris turimiems mėginių rinkiniams yra nuo 0,403\% iki 3,01\% procento, reiškia, kad turimi mėginiai turi šimtus kartų daugiau matų nei mėginių. Tai apsunkina duomenų tyrimo procesą ir gali sukelti persimokymo (angl. \textit{overfitting}) problemą.

Šizofrenijos ir maniakinės depresijos mėginių rinkinys ypatingas tuo, kad jis turi tris klases. Šiame darbe nagrinėjamas tik dviejų klasių atvejis, todėl šizofrenija sergančių pacientų mėginiai nebuvo naudojami.

\subsection{Metodologija}

Eksperimentuose buvo naudojami skyrelyje nr. \ref{eksperimentuose_naudoti_duomenys} aprašyti mėginių rinkiniai. Mėginių rinkiniai nebuvo atskirai normalizuojami, nes daryta prielaida, jog duomenys jau yra apdoroti. 

Klasifikavimui naudota atraminių vektorių klasifikatorių algoritmo R programavimo kalbos paketo ,,e1071`` implementacija. Naudotas tiesinis atraminių vektorių klasifikavimo algoritmas su parametro $C$ reikšme 0,01, kuri buvo nustatyta empiriškai.

Matų atrinkimo metodai buvo suprogramuoti šio darbo autoriaus, nes nėra standartinių R kalbos paketų, kuriuose matų atrinkimo metodai jau būtų implementuoti.

Dėl to, kad turima mažai mėginių ir daug matų, klasifikavimas buvo kartojamas 300 kartų, kai treniravimosi duomenų aibę sudarė kaskart atsitiktinai parenkami 90\% mėginių. Kiekvienoje iteracijoje su vis kitu mėginių poaibiu buvo atliekamas matų atrinkimas, po to būdavo atliekamas klasifikavimas su 10, 20, .., 500 aukščiausią reitingą turinčių matų.

\subsection{Matų atrinkimo metodų sparta}

Matų atrinkimo metodų darbo laikas buvo palygintas naudojant vieną biomedicininių duomenų rinkinį - AltarA \cite{altara}. Skaičiavimai buvo atlikti kompiuteryje naudojant vieną procesoriaus branduolį veikiantį 2.66 GHz, bei 2 GB RAM atminties. ~\ref{fig:visu_laikas} pav. ir ~\ref{fig:cgs_laikas} pav. pavaizduota matų atrinkimo metodo darbo laiko priklausomybė nuo mėginius apibūdinančių matų skaičiaus. 
\begin{figure}[hq]
\begin{minipage}[b]{0.45\linewidth}
\centering
\includegraphics[width=1\textwidth]{images/all_performance.png}
 \caption{Pagrindini matų atrinkimo metodų darbo laikas.}
 \label{fig:visu_laikas}
\end{minipage}
\hspace{0.2cm}
\begin{minipage}[b]{0.45\linewidth}
\centering
\includegraphics[width=1\textwidth]{images/cgs_performance.png}
 \caption{Konsensuso grupėmis grįsto matų atrinkimo metodo darbo laikas.}
 \label{fig:cgs_laikas}
\end{minipage}
\end{figure}

Matų atrinkimo metodų darbo laikas yra atvaizduotas dviem grafikais, nes pagal atliktų eksperimentų rezultatus buvo pastebėta, kad CGS matų atrinkimo metodas yra apie 1000 kartų lėtesnis už kitus suprogramuotus matų atrinkimo metodus, todėl viename grafike neįmanoma atvaizduoti visų turimų matų atrinkimo metodų. Pagal ~\ref{fig:visu_laikas} pav. galime daryti išvadą, kad sparčiausias matų atrinkimo metodas yra \textit{Fisher} įvertis. Pagal gautus matų darbo laiko priklausomybės nuo matų kiekio grafikus galime daryti išvadą, kad CGS algoritmas daugiamačių duomenų matų atrinkimui nėra tinkamas, nes jo skaičiavimų laikas yra per ilgas.

\subsection{Klasifikavimo pagal atrinktus matus tikslumas}

Matų atrinkimo metodų įtaką klasifikavimo klasifikavimo tikslumui buvo matuojama naudojant tris biomedicininių duomenų rinkinius: Gaubtinės žarnos auglio (angl. Colon) \cite{alon1999broad}, Centrinės nervų sistemos (CNS) \cite{pomeroy2002prediction}, prostatos \cite{singh2002gene}. Klasifikavimui buvo naudojami tiesiniai atraminių vektorių klasifikatoriai (SVM) \cite{vapnik2000nature}, su parametru $C=0.01$, kurį nustačiau  empiriškai. Keičiant parametrus keičiasi ir klasifikavimo tikslumas. Klasifikatoriui apmokyti buvo naudojama 90\% atsitiktinai parinktų mėginių iš duomenų rinkinio. Likusiais 10\% mėginių buvo testuojamas klasifikatorius. Klasifikatorius buvo testuojamas po 300 kartų su įvairiu matų skaičiumi: nuo 10 iki 500. Klasifikavimo tikslumas pavaizduotas dviejų tipų grafikais: vidutinio klaidų procento priklausomybės nuo atrinktų matų skaičiaus, bei ROC kreivėmis, kurios buvo gautos pagal duomenis gautus klasifikuojant su tiek atrinktų matų su kiek klasifikavimo tikslumas buvo pats geriausias \cite{green1966signal}.
\begin{figure}[H]
\begin{minipage}[b]{0.45\linewidth}
\centering
\includegraphics[width=.9\textwidth]{../bachelor/images/nncolon_classification.pdf}
\caption{Gaubtinės žarnos auglio meginių klasifikatorių tikslumas.}
\label{fig:class_colon}
\end{minipage}
\hspace{0.2cm}
\begin{minipage}[b]{0.45\linewidth}
\centering
\includegraphics[width=.85\textwidth]{../bachelor/images/nncolon_roc.pdf}
\caption{Gaubtinės žarnos auglio mėginių klasifikatorių ROC kreivės.}
\label{fig:roc_colon}
\end{minipage}
\hspace{0.2cm}
\begin{minipage}[b]{0.47\linewidth}
\centering
\includegraphics[width=.85\textwidth]{../bachelor/images/nncns_classification.pdf}
\caption{Centrinės nervų sistemos meginių klasifikatorių tikslumas.}
\label{fig:class_cns}
\end{minipage}
\hspace{0.2cm}
\begin{minipage}[b]{0.5\linewidth}
\centering
\includegraphics[width=.85\textwidth]{../bachelor/images/nncns_roc.pdf}
\caption{Centrinės nervų sistemos mėginių klasifikatorių ROC kreivės.}
\label{fig:roc_cns}
\end{minipage}
\hspace{0.2cm}
\begin{minipage}[b]{0.5\linewidth}
\centering
\includegraphics[width=.85\textwidth]{../bachelor/images/prostate_classification.pdf}
\caption{Prostatos meginių klasifikatorių tikslumas.}
\label{fig:class_prostate}
\end{minipage}
\hspace{0.2cm}
\begin{minipage}[b]{0.5\linewidth}
\centering
\includegraphics[width=.85\textwidth]{../bachelor/images/prostate_roc.pdf}
\caption{Prostatos mėginių klasifikatorių ROC kreivės.}
\label{fig:roc_prostate}
\end{minipage}
\end{figure}
~\ref{fig:class_colon} pav. matome, kad gaubtinės žarnos auglio duomenų rinkinio matus geriausiai atrenka ADC metodas. Tik šiek tiek prasčiau pasirodo \textit{Fisher} įvertis. Blogiausiai su gaubtinės žarnos auglio mėginiais susidoroja absoliučių svorių SVM matų atrinkimo metodas.

Centrinės nervų sistemos duomenų rinkinys yra sunkiai klasifikuojamas, nes vidutinis klaidų skaičius yra apie 35\%, kai, pvz. gautinės žarnos auglio duomenų rinkinio vidutinis klaidų skaičius yra tik 15\%. ~\ref{fig:class_cns} pav. matome, kad šiam duomenų rinkiniui vidutiniškai geriausiai matus atrenka ADC ir multikriterinio rekursyvaus matų eliminavimo metodai. Prasčiausiai pasirodo \textit{Relief} metodas.
~\ref{fig:class_prostate} pav. matome, kad prostatos duomenų rinkinio matus klasifikavimui geriausiai atrenka ADC absoliučių svorių SVM metodas. Prasčiausiai matus atrenka \textit{Relief}.

Apibendrindamas gautus klasifikavimo tikslumo matavimo rezultatus, galiu teigti, kad nėra vieno absoliučiai geriausio matų atrinkimo metodo. Reikia eksperimentuoti, kad būtų rastas konkrečiai problemai geriausiai tinkantis matų atrinkimo metodas. Tačiau rezultatai parodė, kad matų atrinkimas svariai prisideda prie geresnio klasifikatoriaus sukūrimo.

\subsection{Matų atrinkimo stabilumas}

Matų atrinkimo stabilumas buvo tiriamas naudojant tuos pačius biomedicininių duomenų rinkinius kaip ir tiriant klasifikavimo pagal atrinktus matus tikslumą. Matų atrinkimo stabilumas buvo matuojamas pagal \textit{Kuncheva} ir \textit{Jaccard} indeksus. Stabilumas pats savaime nėra svarbus, jis turi būti matuojamas atsižvelgiant į klasifikavimo tikslumą. Todėl šio skyrelio grafikus reikia nagrinėti atsižvelgiant į skyrelio, kuriame buvo nagrinėtas klasifikavimo pagal atrinktus matus tikslumas.

\begin{figure}[H]
\begin{minipage}[b]{0.47\linewidth}
\centering
\includegraphics[width=.85\textwidth]{../bachelor/images/nncolon_robustness_kuncheva.pdf}
\caption{Matų atrinkimo gaubtinės žarnos auglio mėginiams stabilumo grafikas pagal Kuncheva indeksą.}
\label{fig:robk_colon}
\end{minipage}
\hspace{0.2cm}
\begin{minipage}[b]{0.47\linewidth}
\centering
\includegraphics[width=.85\textwidth]{../bachelor/images/nncolon_robustness_jaccard.pdf}
\caption{Matų atrinkimo gaubtinės žarnos auglio mėginiams stabilumo grafikas pagal Jaccard indeksą.}
\label{fig:robj_colon}
\end{minipage}
\hspace{0.2cm}
\begin{minipage}[b]{0.47\linewidth}
\centering
\includegraphics[width=.85\textwidth]{../bachelor/images/nncns_robustness_kuncheva.pdf}
\caption{Matų atrinkimo CNS mėginiams stabilumo grafikas pagal Kuncheva indeksą.}
\label{fig:robk_cns}
\end{minipage}
\hspace{0.2cm}
\begin{minipage}[b]{0.47\linewidth}
\centering
\includegraphics[width=.85\textwidth]{../bachelor/images/nncns_robustness_jaccard.pdf}
\caption{Matų atrinkimo CNS mėginiams stabilumo grafikas pagal Jaccard indeksą.}
\label{fig:robj_cns}
\end{minipage}
\hspace{0.2cm}
\begin{minipage}[b]{0.47\linewidth}
\centering
\includegraphics[width=.85\textwidth]{../bachelor/images/prostate_robustness_kuncheva.pdf}
\caption{Matų atrinkimo prostatos mėginiams stabilumo grafikas pagal Kuncheva indeksą.}
\label{fig:robk_prostate}
\end{minipage}
\hspace{0.2cm}
\begin{minipage}[b]{0.47\linewidth}
\centering
\includegraphics[width=.85\textwidth]{../bachelor/images/prostate_robustness_jaccard.pdf}
\caption{Matų atrinkimo prostatos mėginiams stabilumo grafikas pagal Jaccard indeksą.}
\label{fig:robj_prostate}
\end{minipage}
\end{figure}
Pagal ~\ref{fig:robk_colon} pav. ir ~\ref{fig:robj_colon} pav. matome, kad gaubtinės žarnos auglio duomenų rinkinio matus stabiliausiai atrenka \textit{Fisher} įvertis. Mažiausiai stabiliai matus atrenka \textit{Relief} metodas.

Pagal ~\ref{fig:robk_prostate} pav. ir ~\ref{fig:robj_prostate} pav. matome, kad prostatos duomenų rinkinio matus stabiliausiai atrenka \textit{Fisher} įvertis. Mažiausiai stabiliai matus atrenka \textit{Relief} metodas.

Apibendrindamas matų atrinkimo stabilumo matavimus galiu sakyti, kad matų atrinkimo stabilumas priklauso ne tik nuo matų atrinkimo metodo, bet ir nuo duomenų rinkinio, kurio matai yra atrinkinėjami. Lengvai klasifikuojamo prostatos duomenų rinkinio matų atrinkimo stabilumas vidutiniškai yra didesnis nei sunkiai klasifikuojamo CNS duomenų rinkinio. Eksperimentų rezultatai rodo, kad \textit{Relief} matų atrinkimo metodas yra nestabiliaus iš tirtųjų. Gana geru stabilumu pasižymi ADC metodas bei \textit{Fisher} įvertis.

\newpage

\addcontentsline{toc}{section}{REZULTATAI IR I{\v S}VADOS}
%% Rezultatų ir išvadų dalyje išdėstomi pagrindiniai darbo rezultatai (kažkas išanalizuota, kažkas sukurta, kažkas diegta), pateikiamos išvados (daromi nagrinėtų problemų sprendimo metodų palyginimai, siūlomos rekomendacijos, akcentuojamos naujovės).

\section*{REZULTATAI IR TOLIMESNIŲ TYRIMŲ KRYPTYS}

Šiame darbe analizuota vis didesnį susidomėjimą kelianti daugiamačių duomenų klasifikavimo problematika ypatingą dėmesį kreipiant į matų atrinkimo problematiką. 

Eksperimentai parodė, kad nėra vieno universaliai geriausio matų atrinkimo metodo. Reikia atsižvelgti į turimus duomenis bei į darbui keliamus tikslus.

Šiame darbe sukaupta patirtis gali būti panaudota kaip tolimesnių daugiamačių biomedicininių duomenų tyrimų pagrindas. Tolimesnių tyrimų kryptys galėtų būti stabilių matų, kurie maksimaliai padidina klasifikavimo tikslumą, atrinkimo metodo paiešką.

Čia bus rezultatai ir išvados. Ir nutarta, kad čia bus aprašytos tolimesnių tyrimų kryptys.


\addcontentsline{toc}{section}{LITERATŪRA} 
\bibliographystyle{alpha}
\bibliography{literatura.bib}
\newpage

\end{document}
