\subsection{Bajeso teorija}

Kuris terminas geresnis: ,,Bajeso teorija`` ar ,,Bajeso išvadų teorija``?
Arba kaip reikėtų versti ``Bayesian Decision Theory''?
Kaip reikėtų suprasti P(error|x)?
% Sensitivity: measures difficulty of task
% Bias: measures strategy of subject

Bajeso teorija\cite{duda2001pattern} yra statistinis požiūris į modelių (angl.
pattern) klasifikavimo problemą. Bajeso teorija paremta klasifikavimo sprendimų kompromisų (angl.
tradeoff) matavimų tikimybėmis ir tų sprendimų svoriais. Bajeso teorija remiasi
prielaida, kad sprendimą galima aprašyti tikimybiniais teminais ir kad
susijusios tikimybės yra iš anksto žinomos.

Kitaip tariant, Bajeso teorija padeda atsakyti į klausimą, kokią tikimybė, kad
objektas priklauso kažkokiai klasei $\omega_j$ ir jis turi kažkokią savybę $x$.
Tai galime užrašyti taip: 

\begin{equation}
p(\omega_j, x) = P(\omega_j | x) \times p(x) = p(x|\omega_j) \times P(\omega_j)
\end{equation}

O jei pergrupuosime pirmąją formulę, tai gausime taip vadinamąją Bajeso formulę:

\begin{equation}
P(\omega_j | x) = \frac{p(x | \omega_j) \times P(\omega_j)}{p(x)}, kur
\end{equation}

$P(\omega_j | x)$ (angl. posterior probability) - sąlyginė tikimybė,
kad objektas priklauso $\omega_j$ klasei, kai kažkuri jo savybė yra $x$;

$p(x | \omega_j)$ (angl. likelihood) - sąlyginė tikimybė, kad $\omega_j$ klasei
priklausantis objektas turės kažkokią savybę x;

$P(\omega_j)$ (angl. prior probability) - tikimybė, kad objektas priklauso
$\omega_j$ klasei (nusatoma iš istorinių duomenų);

$p(x) = \sum_{j=1}^{n} p(x | \omega_j) \times P(\omega_j)$ (angl. evidence) -
perskaičiavimo faktorius (angl. scale factor), kuris parodo tai, kaip dažnai mes
įvertinsime modelį su savybe $x$. Jis užtikrina, kad visų $P(\omega_j | x)$
sąlyginių tikimybių suma bus lygi 1.

Remdamiesi Bajeso teorija laikysime, kad objektas priklauso klasei $\omega_i$,
kai $P(\omega_i | x) > P(\omega_j | x)$ ir atvikščiai. Tai yra taip vadinama
Bajeso sprendimo taisyklė (angl. Bayes Decision Rule).
