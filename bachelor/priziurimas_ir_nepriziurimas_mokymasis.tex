\subsubsection{Prižiūrimas mokymasis}

Prižiūrimojo mokymosi esmė: turime aibę duomenų, kurios objektai yra vadinami
įėjimo duomenimis (angl. Input) arba nepriklausomais kintamaisiais (angl.
Independent variables), jų reikšmės yra išmatuotos arba nustatytos. Darome prielaidą, 
kad nepriklausomi kintamieji turi įtakos vienam ar daugiau rezultato kintamųjų 
(angl. Output) arba priklausomų kintamųjų (angl. Dependent variables). Paėmus dar vieną 
duomenų objektą (angl. Tuple), tikslas yra pagal nepriklausomus kintamuosius nuspėti priklausomus kintamuosius.

Prižiūrėtų mokymosi metodų pagrindinė prielaida yra ta, kad kontekstas suteikia pakankamai 
informacijos. Kitaip tariant - jei žinai pakankamai daug objektų priklausančių kažkokioms tai 
klasėms, tai naujiems objektams pakankamai tiksliai gali priskirti tas klases.

Prižiūrėtasis mokymasis turi du pagrindinius būdus:
\begin{enumerate}
  \item Klasifikavimas (angl. classification) - pagal nepriklausomus
  kintamuosius bandome nuspėti kokybinius (kategorinius) priklausomus kintamuosius. 
  \item Regresija (angl. regression) - pagal nepriklausomus kintamuosius bandome
  nuspėti kiekybinius priklausomus kintamuosius.
\end{enumerate}

Išskiriami trys pagrindiniai klasifikavimo etapai:
\begin{enumerate}
  \item diskriminavimo (atskiriančiųjų) kintamųjų parinkimas,
  \item klasifikavimo taisyklių sudarymas,
  \item klasifikavimo kokybės įvertinimas.
\end{enumerate}

\subsubsection{Neprižiūrimas mokymasis}

Neprižiūrimas mokymasis dar vadinamas klasterizavimu (angl. clustering) arba
mokymusi be mokytojo. Patogumo dėlei, toliau naudosiu klasterizavimo sąvoką kaip
ekvivalentą neprižiūrimojo makymosi sąvokai.

Klasterizavimas (angl.  clustering) - tai viena iš duomenų gavybos sričių. Klasterizavimo 
algoritmo užduotis – objektų suskirstymas  į prasmingas\cite{martisiute08} 
grupes – klasterius, kai jokia papildoma informacija apie tas grupes (jų dydį, kiekį, grupavimo požymius) nėra iš anksto žinoma. 
Klasterizavimo algoritmas pats, pagal pasirinktus algoritmo parametrus, turi nurodyti, kokioms 
grupėms priklauso atitinkami įvesties duomenys. 

Klasterizavimo algoritmų pagrindinis privalumas – gebėjimas atpažinti grupavimo
struktūrą be jokios išankstinės informacijos.  

Klasterizavimo principas - maksimizuoti objektų, esančių vienoje grupėje,
tarpusavio panašumą ir minimizuoti tarpgrupinį objektų panašumą.
