\subsection{Prižiūrimas ir neprižiūrimas mokymasis}
Šiame skyriuje stengsiuosi atsakyti į klausimą kuo skiriasi prižiūrimas mokymasis
(angl. supervised learning) nuo neprižiūrimo mokymosi (angl. unsupervised
learning). Mokymasis, duomenų klasifikavimo kontekste, reiškia modelių
(klasifikatorių) kūrimo metodus (algoritmus), kurie naudoja
mokymosi duomenis\footnote{Mokymosi duomenys (angl. sample data)- duomenys,
kurie yra paruošti darbui programų, kurios kurs klasifikatorius.}, kitaip
tariant, tai mokymasis iš pavyzdžių.

\subsubsection{Prižiūrimas mokymasis}

Prižiūrimas mokymasis tai toks mokymasis, kai turime iš anksto nustatytas
klases bei mokymosi duomenis, kuriems jau yra priskirtos tam tikros
teisingos klasės. Tikslas yra pagal mokymosi duomenis sukurti klasifikatorių,
kuriuo remiantis būtų galima identifikuoti naujų objektų priklausomybę vienai iš
žinomų klasių.\cite{markhall99}

\begin{comment}
Prižiūrimojo mokymosi esmė: turime aibę duomenų, kurios objektai yra vadinami
įėjimo duomenimis (angl. Input) arba nepriklausomais kintamaisiais (angl.
Independent variables), jų reikšmės yra išmatuotos arba nustatytos. Darome prielaidą, 
kad nepriklausomi kintamieji turi įtakos vienam ar daugiau rezultato kintamųjų 
(angl. Output) arba priklausomų kintamųjų (angl. Dependent variables). Paėmus dar vieną 
duomenų objektą (angl. Tuple), tikslas yra pagal nepriklausomus kintamuosius nuspėti priklausomus kintamuosius.
\end{comment}

Prižiūrimo mokymosi metodų pagrindinė prielaida yra ta, kad kontekstas suteikia
pakankamai informacijos. Kitaip tariant - jei žinai pakankamai daug objektų priklausančių kažkokioms tai 
klasėms, tai naujiems objektams pakankamai tiksliai gali priskirti tas klases.

Prižiūrėtasis mokymasis turi du pagrindinius būdus:
\begin{enumerate}
  \item Klasifikavimas (angl. classification) - pagal nepriklausomus
  kintamuosius bandome nuspėti kokybinius (kategorinius) priklausomus kintamuosius. 
  \item Regresija (angl. regression) - pagal nepriklausomus kintamuosius bandome
  nuspėti kiekybinius priklausomus kintamuosius.
\end{enumerate}

Išskiriami trys pagrindiniai klasifikavimo etapai:
\begin{enumerate}
  \item diskriminavimo (atskiriančiųjų) kintamųjų parinkimas,
  \item klasifikavimo taisyklių sudarymas,
  \item klasifikavimo kokybės įvertinimas.
\end{enumerate}

\subsubsection{Neprižiūrimas mokymasis}

Neprižiūrimas mokymasis dar vadinamas klasterizavimu (angl. clustering) arba
mokymusi be mokytojo. Patogumo dėlei, toliau naudosiu klasterizavimo sąvoką kaip
ekvivalentą neprižiūrimojo makymosi sąvokai.

Klasterizavimas (angl.  clustering) - tai viena iš duomenų gavybos sričių. Klasterizavimo 
algoritmo užduotis – objektų suskirstymas  į prasmingas 
grupes – klasterius, kai jokia papildoma informacija apie tas grupes (jų dydį, kiekį, grupavimo požymius) nėra iš anksto žinoma. 
Klasterizavimo algoritmas pats, pagal pasirinktus algoritmo parametrus, turi nurodyti, kokioms 
grupėms priklauso atitinkami įvesties duomenys.\cite{martisiute08}

Klasterizavimo algoritmų pagrindinis privalumas – gebėjimas atpažinti grupavimo
struktūrą be jokios išankstinės informacijos.  

Klasterizavimo principas - maksimizuoti objektų, esančių vienoje grupėje,
tarpusavio panašumą ir minimizuoti tarpgrupinį objektų panašumą.

\subsubsection{Prižiūrimojo ir neprižiūrimojo mokymosi skirtumai}
Ppagrindinis skirtumas tarp prižiūrimojo ir neprižiūrimojo mokymosi slypi
mokymosi duomenyse: prižiūrimojo mokymosi algoritmų įeities duomenyse yra
išreikštinai pasakyta, kokio rezultato mes laukiame, o neprižiūrimojo mokymosi duomenyse tokios
papildomos informacijos nėra. Aptarkime pavyzdį: mums reikia sukurti
klasifikatorių, kuris pasakytų, ar nuotraukoje yra žmogaus veidas. 

Prižiūrimojo mokymosi programai kaip įeities duomenis paduotume keletą 
nuotraukų su žymėmis pasakančiomis, ar nuotraukoje yra žmogaus veidas ar jo ten
nėra, kitaip tariant, duotume keletą pavyzdžių su teisingais atsakymais.
Programa peržvelgs visas nuotraukas ir susikurs klasifikatorių (modelį), kuris
kažkokiu tikslumu galės atskirti nuotraukas su žmogaus veidu. Tokiu būdu mūsų
prižiūrimojo mokymosi programa ``išmoks'', kas yra veidas.

Neprižiūrimojo mokymosi programai kaip įeities duomenis paduotume keletą
nuotraukų be jokių papildomų žymių. Žinoma, mūsų programa pati nesugebės
``išrasti'', kas yra žmogaus veidas, tačiau ji tikriausiai sugrupuos nuotraukas
su žmonių veidais ir tarkim peizažais į skirtingas grupes. Kitaip tariant,
nuotraukos su žmonių veidais mūsų neprižiūrimo mokymosi programai bus nepanašios
į nuotraukas su peizažais, todėl ji į vieną klasterį susidės nuotraukas, kurios
jai atrodo tarpusavyje panašiausios: viename klasteryje nuotraukos su žmonių
veidais, o kitoje su gamtos peizažais.

Apibendrinant galime pasakyti, kad abi mokymosi rūšys siekia to paties tikslo,
tik skitingomis priemonėmis. Pvz. atskirti nuotraukas su žmonių
veidais nuo kitų nuotraukų su ar be teisingos žymės apie konkrečią nuotrauką.
Bendras bruožas yra tai, kad jos mokymosi procese naudoja pavyzdžius, tik tie pavyzdžiai 
skiriasi programai suteikiama informacija. % TODO Kuris daugiamatei analizei
% svarbiau

