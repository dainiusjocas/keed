\section{Teorinis darbo pagrindas}

Šiame skyriuje aprašysiu teorinį darbo pagrindą.

\subsection{Mokymasis su ir be mokytojo}

Šiame skyriuje stengsiuosi atsakyti į klausimą kuo skiriasi mokymas su
mokytoju (angl. supervised learning) nuo mokymo be mokytojo (angl.
unsupervised learning). Mokymasis, duomenų klasifikavimo kontekste, reiškia modelių
(pvz. klasifikatorių) kūrimo metodus (algoritmus), kurie naudoja
mokymosi duomenis\footnote{Mokymosi duomenys (angl. sample data)- duomenys,
kurie yra paruošti darbui programų, kurios kurs modelius (pvz.
klasifikatorius).}, kitaip tariant, tai mokymasis iš pavyzdžių.

\subsubsection{Mokymas su mokytoju}

Mokymas su mokytoju tai toks mokymas, kai turime mokymo duomenis, kuriems jau
yra priskirtos tam tikras teisingas atsakymas. Kitaip tariant, mes sprendžiame
uždavinį, kuriam atsakymą galime pasitikrinti. Mokymas su mokytoju yra
skirstomas į dvi rūšis:
\begin{enumerate}
  \item Klasifikavimas (angl. classification) - pagal nepriklausomus
  kintamuosius bandome nuspėti kokybinius (kategorinius) priklausomus kintamuosius. 
  \item Regresija (angl. regression) - pagal nepriklausomus kintamuosius bandome
  nuspėti kiekybinius priklausomus kintamuosius.
\end{enumerate} 

\subsubsection{Klasifikavimo uždavinio pavyzdys}

Klasifikavimo uždavinių aktualumą galima pagrįsti paprastu pavyzdžiu. 
%% surasti paveiksliukų. 

Uždavinys: Pašto skyriuose laiškai siunčiami įvairiomis kryptimis pagal gavėjo
adresą ir (arba) pašto kodą. Mes norime automatizuoti laiškų rūšiavimą pagal
siuntimo kryptį. Tam, kad galėtume laiškų rūšiavimą pagal kryptį automatizuoti,
mums reikėtų galimybės nuo voko nuskaityti pašto kodą.

Sprendimas: Šią problemą mums padėtų išspręsti skeneris ir programine įranga,
kuri sugebėtų ranka rašytus skaitmenis atpažinti ir konvertuoti į skaitmeninį
formatą. Tų skaitmenų atpažinimui ir konvertavimui į skaitmeninį formatą,
tikėtina, kad mes naudosime klasifikavimo algortimus, nes uždavinys pasižymi
visomis klasifikavimui būdingomis savybėmis: turime aibę duomenų (vaizdinė
informacija su ranka rašytais skaitmenimis), turime teisingus atsakymus (žmogus
pažiūrėjęs į ranka rašytą skaitmenį gali pasakyti programai, koks ten yra
skaitmuo), bei galimų sprendimai yra kategorinio tipo (dešimt skaitmenų nuo
0 iki 9).

Įgyvendinę aukščiau aprašyto uždavinio sprendimą, pašto skyrių vadybininkai
galėtume atlaisvinti žmones nuo iš esmės mechaninio darbo - rūšiuoti laiškus.
Tokiu būdu būtų optimizuotas pašto skyrių efektyvumas.

\subsubsection{Regresijos uždavinio payzdys}



Abiejų mokymo su mokytoju rūšių tikslas yra pagal mokymosi duomenis sukurti
modelį, kuriuo remiantis būtų galima identifikuoti naujų objektų
savybes.\cite{markhall99} Šiame darbe negrinėsime klasifikavimo problemą.

