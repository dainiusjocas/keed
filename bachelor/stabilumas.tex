Dimensijų atrinkimo technikų stabilumas yra vis didesnę svarbą įgaunanti tyrimų
kryptis. Stabilumo aktualumas yra sąlygotas to, kad biologiniuose duomenyse
galima gana užtikrintai daryti prielaidą, kad konkrečiai problemai yra
aktualios tik tam tikros dimensijos. Todėl dalykinės srities ekspertams yra
aktualu naudoti tik tuos dimensijų atrinkimo metodus, kurie yra stabilūs ir 
relevantiški modeliuojamai problemai.

Šiame skyriuje apžvelgsime teorinį stabilumo matavimų modelį.
Taip pat įvertinsime pavienių dimensijų atrinkimo metodų stabilumą
taikant juos įvairioms duomenų bazėms. Taip pat išanalizuosime situaciją, kai
kombinuojami keletos dimensijų atrinkimo metodų rezultatai.

\subsection{Stabilumo matavimas}

Vertinant dimensijų atrinkimo metodų stabilumą yra svarbu kaip panašiai yra
atrenkamos dimensijos, kai yra atliekamas dimensijų atrinkimas su vis kitu 
duomenų poaibiu. Kuo panašesnius dimensijų atrinkimo rezultatus gauname, tuo stabilumas
yra didesnis. Vidutinis stabilumas gali būti apibrėžtas kaip vidurkis visų 
reitingavimo metu gautų sąrašų porų tarpusavio panašumo įverčių:

\begin{equation}
 S_{tot}=\frac{2\sum_{i=1}^{k-1}\sum_{j=i+1}^{k} S(f_i, f_j)}{k*(k-1)},
\end{equation} 
kur $k$ žymi kiek kartų buvo imtas skirtingas poaibis objektų dimensijų atrinkimui,
$f_i$, $f_j$ - dimensijų atrinkimo rezultatas - reitingai, $S(f_i, f_j)$ - yra kokia 
nors panašumo matavimo funkcija.

Kaip matome dimensijų atrinkimo stabilumas priklauso nuo to, kokią panašumo 
funkciją naudosime. Tradicinės panašumo funkcijos (persidengimo procentas, 
Pearson'o koreliacija, Spearman'o koreliacijoa, Jaccard indeksas) 
gali būti taikomos, bet jos yra linkusios priskirti didesnes panašumo
reikšmes, kai pasirenkamas didesnis dimensijų poaibis. Taip yra dėl padidėjusio 
sisteminio nuokrypio (ang. bias), nes imant didesnį poaibį padidėja tikimybė
tiesiog atsitiktinai pasirinkti dimensiją. 
Kad išventume šios problemos panašumui vertinti buvo pasirinktas Kunchevos
\cite{DBLP:conf/aia/Kuncheva07} indeksas:
\begin{equation}
 KI(f_i, f_j)=\frac{r*N - s^2}{s*(N-s)}=\frac{r - (s^2/N)}{s - (s^2/N)},
\end{equation}		
kur $s=|f_i|=|f_j|$ yra atrinktų dimensijų aibės dydis, $r=|f_i \bigcap f_j|$ -
abiems atrinktiems dimensijų poaibiams bendrų dimensijų skaičius, $N$ - bendras
 duomenų aibės
dimensijų skaičius. Pastebėtina, kad formulėje esantis atėminys $s^2/N$ ištaiso 
sisteminį nuokrypį atsirandantį dėl galimybės atsitiktinai pasirinkti dimensijas.
Kunchevos indeksas gali įgyti reikšmes iš intervalo
$[-1, 1]$, kur didesnė reikšmė reiškia didesnį panašumą, o artimos nuliui 
reikšmės reiškia, kad dimensijos atrenkamos daugiausia atsitiktinai. Kunchevos 
indekso ypatybė yra ta, kad jis atsižvelgia tik į
persidengiančias, tačiau visiškai nekreipia dėmesio į koreliuojančias dimensijas.

Vertinant stabilumą tarp skirtingų metodų gali iškilti problemų, nes ne visi 
dimensijų atrinkimo metodai gražina rezultatą tokiu pačiu formatu. Šiame darbe
dimensijų atrinkimo metodų rezultatas yra ne dimensijai priskirtas svoris, bet 
dimensijos reitingas. Todėl $f_i$ yra sąrašas, kurio ilgis yra $N$, kur pirmas
sąrašo elementas yra geriausią reitingą turinčios dimensijos numeris, o 
paskutinis sąrašo elementas yra blogiausią reitingą turinčios dimensijos numeris.

Galiausiai yra svarbu paminėti, kad dimensijų stabilumas nėra matuojamas
visiškai nepriklausomai - jis yra matuojamas atsižvelgiant į klasifikavimo
rezultatus. Dimensijų atrinkimo metodų stabilumas yra matuojamas tik tada
kai atrinktos dimensijos duoda gerus klasifikavimo rezultatus. Taip yra, nes 
kokios nors dalykinės srities ekspertui, nėra naudingos tos dimensijų atrinkimo
strategijos, kurios duoda labai stabilius rezultatus, bet nėra naudingos
klasifikavimo modelio kūrime.

\subsection{Pavienių dimensijų atrinkimo metodų stabilumas}

Šiame skyriuje apžvelgsime pavienių dimensijų atrinkimo motodų stabilumą. 
Stabilumas visiems metodams buvo matuojamas atsižvelgiant į klasifikavimo
rezultatus - buvo matuojamas stabilumas tikra

\subsubsection{Fisher'io dimensijų atrinkimo metodas}