\documentclass[a4paper,10pt]{article}
\usepackage[utf8x]{inputenc}
\usepackage[lithuanian]{babel}
\usepackage[L7x]{fontenc}
\usepackage{lmodern}
\usepackage{verbatim}
\usepackage{tocloft} 
\usepackage{url}
\renewcommand{\cftsecleader}{\cftdotfill{\cftdotsep}}
\usepackage{parskip}
\usepackage{graphicx} 
\parindent = 1cm
\usepackage{mathtools} % pades rašyti matematinius intarpus
\usepackage{longtable} % kad galeciau ilgas lenteles tureti

\title{moo}
\author{You}
\begin{document}
\maketitle
%\begin{titlepage}

\begin{center}
VILNIAUS UNIVERSITETAS\\
MATEMATIKOS IR INORMATIKOS FAKULTETAS\\
PROGRAMŲ SISTEMŲ KATEDRA\\
\vspace{150pt}

\huge \textbf{Daugiamačių duomenų klasifikavimo analizė\\}
\vspace{20pt}
\large\textbf{Classification Analysis of High-Dimensional Data\\}
\vspace{20pt}
\small Bakalauro baigiamasis darbas\\
\vspace{40pt}
\end{center} 


\begin{flushleft}
Atliko: \hspace{57pt} Dainius Jocas \hspace{125pt}\textsubscript{(para\v{s}as)}

\vspace{10pt}
Darbo vadovas: \hspace{14pt} dr. Juozas Gordevičius \hspace{76pt}\textsubscript{(para\v{s}as)}

\vspace{10pt}
Darbo recenzentas: prof. dr. Romas Baronas \hspace{66pt}\textsubscript{(para\v{s}as)}
\\
\vspace{130pt}
\end{flushleft}

\begin{center}
VILNIUS - 2012
\end{center}

\end{titlepage}


%\begin{titlepage}

\begin{center}
VILNIAUS UNIVERSITETAS\\
MATEMATIKOS IR INORMATIKOS FAKULTETAS\\
PROGRAMŲ SISTEMŲ KATEDRA\\
\vspace{150pt}

\huge \textbf{Daugiamačių duomenų klasifikavimo analizė\\}
\vspace{20pt}
\large\textbf{Classification Analysis of High-Dimensional Data\\}
\vspace{20pt}
\small Bakalauro baigiamasis darbas\\
\vspace{40pt}
\end{center} 


\begin{flushleft}
Atliko: \hspace{57pt} Dainius Jocas \hspace{125pt}\textsubscript{(para\v{s}as)}

\vspace{10pt}
Darbo vadovas: \hspace{14pt} dr. Juozas Gordevičius \hspace{76pt}\textsubscript{(para\v{s}as)}

\vspace{10pt}
Darbo recenzentas: prof. dr. Romas Baronas \hspace{66pt}\textsubscript{(para\v{s}as)}
\\
\vspace{130pt}
\end{flushleft}

\begin{center}
VILNIUS - 2012
\end{center}

\end{titlepage} % kol kas neaktualu
\let \savenumberline \numberline
\def \numberline#1{\savenumberline{#1.}}
%\tableofcontents 
\begin{longtable}{|p{3cm}|p{12.5cm}|}
\captionsetup{labelsep=period}
\caption{OLTP sistemų savybės}\\
%This is the header for the first page of the table...

\hline \hline
{\textbf{Savybė}} &
{\textbf{Paaiškinimas}}\\
\hline
\endfirsthead

%This is the header for the remaining page(s) of the table...

\multicolumn{2}{c}{{\tablename} \thetable{} -- Tęsinys} \\[0.5ex]
\hline \hline
{\textbf{Savybė}} &
{\textbf{Paaiškinimas}}\\
\hline
\endhead

%This is the footer for all pages except the last page of the table...

\multicolumn{2}{l}{{Lentelės tęsinys kitame puslapyje\ldots}} \\
\endfoot

%This is the footer for the last page of the table...

\hline \hline
\endlastfoot
\hline 
Greitas atsako laikas 
&
Visos OLTP sistemos yra kuriamos tam, kad jos būtų interaktyvios bei su jomis būtų galima dirbti tada, kada reikia. OLTP sistemoms reikia trumpų atsako laikų, kad sistemos naudotojai išliktų produktyvūs. \\ 
\hline
Smulkios transakcijos 
&
OLTP sistemos paprastai dirba tik su tam tikrais nedideliais duomenų kiekiais; duomenų apdorojimas dažniausiai yra paprastas, o sudėtingi DB lentelių sujungimai (angl. join) yra naudojami palyginti retai. 
\\ 
\hline
Duomenų palaikymo (angl. maintenance) operacijos 
&
Nėra neįprasta turėti ataskaitų rengimo ir duomenų atnaujinimo programas, kurios turi dirbti arba reguliariai, arba pagal poreikį. Tos programos, kurios veikia foniniu (angl. Background) režimu, kol kiti sistemos klientai normaliai dirba su sistema, gali bandyti dirbti su dideliais kiekiais duomenų, kurie yra toje pačioje DB.
\\
\hline
Daugelio naudotojų sistema
&
OLTP sistemos gali turėti labai daug naudotojų, kuriems gali reikėti tuo pačiu metu dirbti su tais pačiais duomenimis.
\\
\hline
Intensyvus vienalaikiškumas
&
Dėl to, kad OLTP sistemos gali turėti daug klientų; joms reikia apdoroti didelius operacijų kiekius, kuriuos reikia apdoroti per trumpą laiką, intensyvaus vienalaikiškumo palaikymas yra labai svarbus.
\\
\hline
Dideli duomenų kiekiai
&
Priklausomai nuo programos tipo, klientų kiekio ir duomenų saugojimo laiko, OLTP sistemų dydis gali labai išaugti. Pavyzdžiui, elektroninės bankininkystės, klientas nori pamatyti visas savo paskutinių metų sąskaitas.
\\
\hline
Pasiekiamumas
&
Pasiekiamumas OLTP sistemose turi būti nuolatinis. Laikinai nepasiekiama sistema gali paveikti didelį klientų skaičių, dėl kurio gali stipriai nukentėti visa kompanija.
\\
\hline
Periodiniai naudojimo ypatumai
&
Gali nutikti taip, kad OLTP sistemų naudojamumas turi kažkokį cikliškumą. Pavyzdžiui, kiekvieno mėnesio gale yra apskaičiuojama visų banko sąskaitų suvestinė.
\\
\hline
\end{longtable}
\begin{comment}
%% JG: Sekcija apie darbe nagrinėjamą problemą. Problem definition.

\section{Teorija}
Šiame skyriuje aptarsiu teorinį darbo pagrindą ir pagrindus.
\subsection{Mokymasis su mokytoju ir mokymasis be mokytojo}

Šiame skyriuje atsakysiu į klausimą kuo skiriasi mokymasis su
mokytoju (angl. supervised learning) nuo mokymosi be mokytojo (angl.
unsupervised learning). Mokymasis, duomenų klasifikavimo kontekste, reiškia modelių(pvz. klasifikatorių) kūrimo metodus (algoritmus), kurie naudoja
mokymosi duomenis\footnote{Mokymosi duomenys (angl. sample data)- duomenys,
kurie yra paruošti darbui programų, kurios kurs modelius (pvz.
klasifikatorius).}, kitaip tariant, tai mokymasis iš pavyzdžių.

\subsubsection{Mokymasis su mokytoju}

Mokymasis su mokytoju tai toks mokymasis, kai turime mokymosi duomenis, kuriems jau
yra priskirtas tam tikras teisingas atsakymas. Kitaip tariant, mes sprendžiame
uždavinį, kuriam atsakymą galime pasitikrinti. Mokymasis su mokytoju yra
skirstomas į dvi rūšis:
\begin{enumerate}
  \item Klasifikavimas (angl. classification) - pagal nepriklausomus
  kintamuosius bandome nuspÄ—ti kokybinius (kategorinius) priklausomus kintamuosius. 
  \item Regresija (angl. regression) - pagal nepriklausomus kintamuosius bandome
  nuspÄ—ti kiekybinius priklausomus kintamuosius.
\end{enumerate} 

%% JG: Pateik vizualų klasifikavimo pavyzdį iliustruojanti visus 3 etapus.
%% DJ: Vizualų, ta prasme su paveiksliukais ar ir tas pavyzdys su paštu
% pakankamai vaizdingas?

%% JG: Reikia kitaip struktūrizuoti šitą skyrių: 
% +Pradžioj pasakyk, kad yra klasifikavimas ir regresija ir po
%  sakinį kiekvienam.
% +Tada aptark klasifikavimą ir pateik pavyzdį. 
% +Tada pateik regresijos pavyzdį.
% +Tada parašyk, kad šiame darbe studijuojama klasifikavimo problema.

\paragraph{Klasifikavimo uždavinio pavyzdys}

Klasifikavimo tikslas - identifikuoti parametrus, kurie nusakytų grupę (klasę),
kuriai priklauso objektas. Klasifikavimo sąvoka gali būti naudojama tiek esamų
duomenų suvokimui, tiek naujų objektų charakteristikų prognozavimui.
Klasifikavimo uždavinių aktualumą galima parodyti tokiu pavyzdžiu.

\begin{figure}[htb]
\begin{center}
\leavevmode
\includegraphics[width=0.5\textwidth]{images/ranka_rasyti_skaiciai.png}
\end{center}
\caption{Ranka rašytas tekstas, kurį reikia atpažinti.}
\label{fig:flash}
\end{figure}

Uždavinys: Pašto skyriuose laiškai siun�iami įvairiomis kryptimis pagal gavėjo
adresą ir (arba) pašto kodą. Norima automatizuoti laiškų rūšiavimą pagal
siuntimo kryptį. Tam, kad būtų galima laiškų rūšiavimą pagal kryptį
automatizuoti, mums reikia priemonės atpažinti ant voko užrašytą
pašto kodą.

Sprendimas: Šią problemą mums padėtų išspręsti skeneris ir programine įranga,
kuri sugebėtų ranka rašytus skaitmenis atpažinti ir konvertuoti į skaitmeninį
formatą. Tų skaitmenų atpažinimui ir konvertavimui į skaitmeninį formatą
mes naudosime klasifikavimo algoritmus, nes uždavinys pasižymi
visomis klasifikavimui būdingomis savybėmis: turime aibę duomenų (vaizdinė
informacija su ranka rašytais skaitmenimis), turime teisingus atsakymus (žmogus
pažiūrėjęs į ranka rašytą skaitmenį gali pasakyti programai, koks ten yra
skaitmuo), bei galimų sprendimai yra kategorinio tipo (dešimt skaitmenų nuo
0 iki 9).

Klasifikatorių kursime trimis etapais:
\begin{enumerate}
  \item diskriminavimo (atskirian�iųjų) kintamųjų parinkimas - nuskenuotų
  pašto kodų skaitmenų dažniausiai pasitaikan�ių, charakteringiausių linijų
  radimas,
  \item klasifikavimo taisyklių sudarymas - pagal tam tikrą charakteringiausių
  linijų grupę objektui priskiriama klasė,
  \item klasifikavimo kokybės įvertinimas - kokybei įvertinti naudojami įvairūs
  metodai, tokie kaip kryžminis patikrinimas (angl. cross-validation) ir
  įkel�ių metodas (angl. bootstrap).
\end{enumerate}

Įgyvendinę aukš�iau aprašyto uždavinio sprendimą, pašto skyriaus vadybininkai
galėtų atlaisvinti žmones nuo iš esmės mechaninio darbo - rūšiuoti laiškus.
Tokiu būdu būtų optimizuotas pašto skyrių efektyvumas.

\paragraph{RegresinÄ—s analizÄ—s payzdys}

Regresija prognozuojant naujų duomenų reikšmes naudojasi žinomais, jau turimais
duomenimis. Ji naudoja standartinius statistinius metodus, tokius kaip mažiausių
kvadratų metodas (angl. least squares). Regresinė analizė dažniausiai naudojama
įvertinti (ang. forecast) ateities duomenų vertes bei interpoliacijai -
funkcijos tikėtinos reikšmės tarp dviejų taškų įvertinimui.

Tipinio uždavinio, kuriam naudojama regresinė analizė pavyzdys: Aktuarinėje
(draudimo) matematikoje reikia turėti įver�ius, pasakan�ius kokia tikimybė, kad
žmogus vienokio ar kitokio amžiaus mirs. Tam yra naudojamos taip vadinamos 
mirtingumo lentelės. Jose duomenys aprašo kiek ir kokio amžiaus žmonių
kažkuriais metais mirė, pvz. 2010 metais Lietuvoje mirė 1000 20 metų amžiaus 
žmonių. Detalesni duomenys nėra naudojami, nes per daug sudėtinga juos apdoroti.
Kadangi aktuarai nori apskai�iuoti draudimo kainą, jiems reikia įvertinti
riziką, kada žmogus mirs, tai jie naudodamiesi regresinės analizės metodais 
paskai�iuoja tikėtiniausią reikšmę, kad pvz. yra 3\% tikimybė, kad žmogus  mirs
dvidešimtaisiais savo gyvenimo metais. Kitais žodžiais tariant, iš turimų
duomenų mes sukursime tolydžią funkcija, kuri mums pasakys reikšmes taškuose,
kurių mes neturime.

\paragraph{Klasifikavimas ir regresija}

Abiejų mokymosi su mokytoju rūšių tikslas yra pagal mokymosi duomenis sukurti
modelį, kuriuo remiantis būtų galima identifikuoti naujų objektų
savybes.\cite{markhall99} Å iame darbe negrinÄ—sime klasifikavimo problemÄ….

\subsubsection{Mokymasis be mokytojo}

Mokymasis be mokytojo tai toks mokymasis, kai turime mokymosi duomenis, kuriems
nėra priskirtas teisingas atsakymas. Kitaip tariant, mes sprendžiame
uždavinį, kuriam atsakymo galime pasitikrinti. Mokymosi be mokytojo principas - 
maksimizuoti objektų, esan�ių vienoje grupėje, tarpusavio panašumą ir 
minimizuoti tarpgrupinį objektų panašumą.

Mokymosi su mokytoju metu galima išmatuoti gauto modelio tikslumą įvairiais metodais, pvz.
kryžminiu patikrinimu. Mokymesi be mokytojo mes tokių tiesioginio patikrinimo
procedūrų neturime. Todėl yra sunkiau išsiaiškinti patikimumą išvadų, gautų pagal
daugumos mokymosi be mokytojo algoritmų darbo rezultatus. 

Yra mažiausiai penkios pagrindinės priežastys, kodėl mums gali būti įdomūs
mokymosi be mokytojo algoritmai:

\begin{enumerate}
	\item Turime labai daug nesužymėtų (angl. unlabelled) duomenų, o jų
	sužymėjimas rankomis būtų labai brangus. 
	\item Norime apsimokyti su dideliu kiekiu sąlyginai ,,pigių`` duomenų tam,
	kad paskui galÄ—tume	pasitelkti mokymosi su mokytoju algoritmus, ir tada
	detaliau ištirti duomenis.
	\item Duomenų struktūros šablonas yra nuolat kintantis, ir jei tą kitimą
	galėtume sekti mokymosi be mokytojo režimu, tai būtų galima padidinti 
	mūsų programos našumą.
	\item Galima panaudoti mokymosi be mokytojo algoritmus, kad surastume
	duomenų savybes, kurias vėliau panaudosime duomenų kategorizavimui.
	\item Pradinėje duomenų analizės stadijoje pasinaudoję mokymosi be mokytojo
	metodais galime geriau pažinti turimus duomenis.
\end{enumerate}

Mokymosi be mokytojo algoritmų pagrindinis privalumas – gebėjimas atpažinti grupavimo
struktūrą be jokios išankstinės informacijos.

%% JG: neprižiūrimų mokymosi metodų yra visokių: association rule mining,
% clustering, ir t.t. Zr ESL knygos 14 skyrių.
%% DJ: Nurašinėjau nuo Duda knygos tą vietą, kur mokymas be mokytojo ir 
% klasterizavimas yra sinonimai.

%% Kartais šiokia tokia informacija žinoma. Pvz., klasterių kiekis nurodomas
% k-means algoritme. Arba galima daryti prielaidas apie klasterių struktūrą:
% k-means ieško apvalių klasterių. Esminis dalykas yra tas, kad teisingas
% atsakymas nėra žinomas.

%% JG: algoritmas turi atrasti grupes duomenyse, jos nėra iš anksto žinomos.

\subsubsection{Mokymosi su mokytoju ir mokymosi be mokytojo skirtumai}

Pagrindiniai skirtumai tarp mokymosi su mokytoju ir mokymosi be mokytojo yra:
\begin{itemize}
	\item mokymosi duomenys - mokymosi su mokytoju algoritmų įeities duomenyse
	yra	išreikštinai pasakyta, kokio rezultato mes laukiame, o mokymosi be
	mokytojo įeities duomenyse tokios papildomos informacijos nėra.
	\item  naudojimo tikslai - mokymasis su mokytoju siekia iš pavyzdžių
	išmokti vertinti naujus duomenis, o mokymasis be mokytojo siekia atrasti
	vidinę duomenų struktūrą.
\end{itemize}

Aptarkime pavyzdį: nuotraukų apdorojimas.

Mokymosi su mokytoju programai kaip įeities duomenis paduotume keletą 
nuotraukų su žymėmis pasakan�iomis, ar nuotraukoje yra žmogaus veidas ar jo ten
nėra, kitaip tariant, duotume keletą pavyzdžių su teisingais atsakymais.
Programa peržvelgs visas nuotraukas ir susikurs klasifikatorių (modelį), kuris
kažkokiu tikslumu galės atskirti nuotraukas su žmogaus veidu. Tokiu būdu mūsų
mokymosi programa ,,išmoks`` nuotraukose atpažinti veidus.

Mokymosi be mokytojo programai kaip įeities duomenis paduotume keletą
nuotraukų be jokių papildomų žymių. Žinoma, mūsų programa pati nesugebės
,,išrasti``, kas yra žmogaus veidas, ta�iau ji tikriausiai sugrupuos nuotraukas
su žmonių veidais ir tarkim peizažais į skirtingas grupes. Kitaip tariant,
nuotraukų su žmonių veidais vidinė struktūra mūsų mokymosi be mokytojo programai
bus nepanaši į nuotraukų su peizažais vidinę struktūrą, todėl ji į vieną grupę
sudės nuotraukas, kurios jai atrodo tarpusavyje panašiausios: vienoje
grupėje nuotraukos su žmonių veidais, o kitoje su gamtos peizažais.

Abu mokymo procesai yra panašūs savo esme (siekia išgauti žinias apie turimus
duomenis), bet jų panaudojimas skiriasi iš esmės (mokymo su mokytoju atveju mes
kuriame modelį apibūdinantį kaip buvo sukurti mokymo duomenys, kad galėtume
spėti naujų objektų savybes, o mokymo be mokytojo atveju siekiame susipažinti
su vidine mokymo duomenų struktūra, kai nėra kaip pamatuoti ar geri ar blogi
klasteriai buvo rasti).

%% JG: aš nesutinku, kad abiem procesais siekiama tų pa�ių tikslų. Vienu atveju 
% siekiama išmokti iš pavyzdžių. Kitu atveju siekiama atrasti nežinomas
% struktūras turimuose duomenyse. Procesai yra panašūs savo esme, bet jų 
% panaudojimas skiriasi iš esmės.

%% JG: iš vikipedijos: In machine learning, unsupervised learning refers to the 
% problem of trying to find hidden structure in unlabeled data. Since the
% examples given to the learner are unlabeled, there is no error or reward
% signal to evaluate a potential solution. This distinguishes unsupervised 
% learning from supervised learning and reinforcement learning.

%% JG: visą šitą skyrių reikia pateikti koncentruotai. Esminiai teiginiai ir grafiniai pavyzdžiai. 

%% DJ: Turiu pripažint, kad šitam pavyzdyje prigrybavau stipriai. Nurašinėjau
% pavyzdį kur prastai paaiškino skirtumą, bet užtat man pavyzdys patiko. Dabar
% labiau į temą surašyta.
 % Supervised and Unsupervised Learning
\subsection{Bajeso teorija}

Kuris terminas geresnis: ,,Bajeso teorija`` ar ,,Bajeso išvadų teorija``?
Arba kaip reikėtų versti ``Bayesian Decision Theory''?
Kaip reikėtų suprasti P(error|x)?
% Sensitivity: measures difficulty of task
% Bias: measures strategy of subject

Bajeso teorija\cite{duda2001pattern} yra statistinis požiūris į modelių (angl.
pattern) klasifikavimo problemą. Bajeso teorija paremta klasifikavimo sprendimų kompromisų (angl.
tradeoff) matavimų tikimybėmis ir tų sprendimų svoriais. Bajeso teorija remiasi
prielaida, kad sprendimą galima aprašyti tikimybiniais terminais ir kad
susijusios tikimybės yra iš anksto žinomos.

Kitaip tariant, Bajeso teorija padeda atsakyti į klausimą, kokią tikimybė, kad
objektas priklauso kažkokiai klasei $\omega_j$ ir jis turi kažkokią savybę $x$.
Tai galime užrašyti taip: 

\begin{equation}
p(\omega_j, x) = P(\omega_j | x) \times p(x) = p(x|\omega_j) \times P(\omega_j)
\end{equation}

O jei pergrupuosime pirmąją formulę, tai gausime taip vadinamąją Bajeso formulę:

\begin{equation}
P(\omega_j | x) = \frac{p(x | \omega_j) \times P(\omega_j)}{p(x)}, kur
\end{equation}

$P(\omega_j | x)$ (angl. posterior probability) - sąlyginė tikimybė,
kad objektas priklauso $\omega_j$ klasei, kai kažkuri jo savybė yra $x$;

$p(x | \omega_j)$ (angl. likelihood) - sąlyginė tikimybė, kad $\omega_j$ klasei
priklausantis objektas turės kažkokią savybę x;

$P(\omega_j)$ (angl. prior probability) - tikimybė, kad objektas priklauso
$\omega_j$ klasei (nusatoma iš istorinių duomenų);

$p(x) = \sum_{j=1}^{n} p(x | \omega_j) \times P(\omega_j)$ (angl. evidence) -
perskaičiavimo faktorius (angl. scale factor), kuris parodo tai, kaip dažnai mes
įvertinsime modelį su savybe $x$. Jis užtikrina, kad visų $P(\omega_j | x)$
sąlyginių tikimybių suma bus lygi 1.

Remdamiesi Bajeso teorija laikysime, kad objektas priklauso klasei $\omega_i$,
kai $P(\omega_i | x) > P(\omega_j | x)$ ir atvikščiai. Tai yra taip vadinama
Bajeso sprendimo taisyklė (angl. Bayes Decision Rule).
 %Bayesian decision theory}

\subsection{Klasifikavimas ``artimiausio kaimyno'' metodu}
\subsection{Klasifikavimas ``mažiausių kvadratų metotu''}
\subsection{Naive Bayesian classifTitleier}
\subsection{Bias and variance tradeoff}
\subsection{Klasifikavimo metodo įvertinimas}
\subsubsection{Klasifikavimo metodo įvertinimas ``cross-validation'' metodu}
\subsubsection{Klasifikavimo metodo įvertinimas ``bootstrapping'' metodu}
\subsection{Atraminių vektorių klasifikatoriai}

Atraminių vektorių klasifikatoriai (angl. \textit{support vector machines}, SVM) - tai mašininio mokymosi  algoritmas, kuris gali būti taikomas tiek klasifikavime, tiek regresinėje analizėje. Jis priskiriamas prie mokymosi su mokytoju algoritmų \cite{vapnik2000nature}.

Atraminių vektorių klasifikatorių algoritmo idėja yra duomenų vektorių erdvėje surasti hiperplokštumą (sprendimo ribą (angl. \textit{decision boundary})), kurios atstumas nuo skirtingoms klasėms priklausančių objektų būtų didžiausias, galimai pašalinant triukšmą bei išimtis. Kitaip tariant, yra ieškoma hiperplokštuma, kuri geriausiai atskiria objektus priklausančius skirtingoms klasėms. 

Tarkime, kad turime $L$ mokymosi objektų, kurių kiekvienas objektas $x_i$ turi $D$ matų ir priklauso vienai iš dviejų klasių $y_i=-1$ arba $y_i=+1$. Taigi turime mokymosi duomenis, kurių pavidalas yra:
\begin{equation}
 \{x_i, y_i\}, kur\; i=1..L, y_i \in \{-1,1\}, x \in \Re^D
\end{equation}
Tarkime, kad duomenys yra tiesiškai atskiriami. Tai reiškia, kad galima nupiešti tiesę grafe $x_1$ ir $x_2$, kuri atskiria dvi klases, kai $D=2$ ir hiperplokštumą grafuose $x_1, x_2,...x_D$, kai $D > 2$. Hiperplokštuma apibrėžta $w\cdot x + b = 0$, kur $w$ -- hiperplokštumos normalės vektorius, $\frac{b}{||w||}$ -- statmens einančio nuo hiperplokštumos iki koordinačių pradžios taško ilgis.

Atraminiai vektoriai (angl. \textit{support vectors}) yra duomenų objektai esantys arčiausiai atskiriančiosios hiperplokštumos. Atraminių vektorių klasifikatorių algoritmo tikslas yra orientuoti hiperplokštumą tokiu, būdu, kad atstumas tarp jos ir artimiausių objektų iš abiejų klasių \cite{cortes1995support}. Tai pavaizduota ~\ref{fig:support_vector_machines} pav. Taigi, atraminių vektorių klasifikatorių sukūrimas yra parametrų $w$ ir $b$ tenkinančių minėtas sąlygas radimas. Tai galima užrašyti tokia nelygybe:
\begin{equation}
 \label{svm_separable}
 y_i(x_i \cdot w + b) - 1 > 0
\end{equation}
Jei abiejų klasių objektai nėra tiesiškai atskiriami, reikia ,,atpalaiduoti'' (\ref{svm_separable}) salygą:
\begin{equation}
 \label{svm_non_separable}
 y_i(x_i \cdot w + b) - 1 + \xi_i > 0, kur\; \xi_i \geq 0, \;  \forall_i,
\end{equation}
kur $\xi_i$ yra baudos už neteisingai klasei priskirtą objektą dydis.
\begin{figure}
 \centering
 \includegraphics[width=.7\textwidth]{images/support_vector_machines.jpg}
 \caption{Hiperplokštuma nubrėžta per dvi tiesiškai atskiriamas klases.}
 \label{fig:support_vector_machines}
\end{figure}

Atraminių vektorių klasifikatoriai gerai tinka uždaviniams, kai turima labai maža mokymosi duomenų aibė. Biomedicininiai duomenys ir pasižymi tuo, kad mokymosi duomenų aibė yra maža palyginus su turimų matų skaičiumi. Todėl atraminių vektorių klasifikatorių naudojimas dirbui su biomedicininiais duomenimis yra tapęs standartiniu pasirinkimu. 


%% JG: cituoti turi originalų darbą:
%% JG: C. Cortes and V. Vapnik, Support-Vector Networks, Machine Learning, 20(3):273-297, September 1995
%% JG: Vladimir N. Vapnik. The Nature of Statistical Learning Theory. Springer, New York, 1995


%SVM is a type of machine learning algorithm derived from statistical learning
%[theory](http://download.oracle.com/docs/cd/B14117_01/text.101/b10729/classify.htm).

%% JG: nepamiršksio daugiamatiškumo erdvę, o ten juos galima atskirti tiesiškai.
\subsection{Random forests}
\subsection{Kuo ypatingas daugiamačių duomenų klasifikavimas}
\subsubsection{the curse of dimensionality}

\section{Susiję darbai}
\section{Klasifikavimo metodų palyginimo karkasas}
\section{Klasifikavimo metodų palyginimo rezultatai}

\addcontentsline{toc}{section}{REZULTATAI IR IŠVADOS}
\section*{REZULTATAI IR IŠVADOS}
%% Rezultatų ir išvadų dalyje išdėstomi pagrindiniai darbo rezultatai (kažkas išanalizuota, kažkas sukurta, kažkas diegta), pateikiamos išvados (daromi nagrinėtų problemų sprendimo metodų palyginimai, siūlomos rekomendacijos, akcentuojamos naujovės).

\section*{REZULTATAI IR TOLIMESNIŲ TYRIMŲ KRYPTYS}

Šiame darbe analizuota vis didesnį susidomėjimą kelianti daugiamačių duomenų klasifikavimo problematika ypatingą dėmesį kreipiant į matų atrinkimo problematiką. 

Eksperimentai parodė, kad nėra vieno universaliai geriausio matų atrinkimo metodo. Reikia atsižvelgti į turimus duomenis bei į darbui keliamus tikslus.

Šiame darbe sukaupta patirtis gali būti panaudota kaip tolimesnių daugiamačių biomedicininių duomenų tyrimų pagrindas. Tolimesnių tyrimų kryptys galėtų būti stabilių matų, kurie maksimaliai padidina klasifikavimo tikslumą, atrinkimo metodo paiešką.

Čia bus rezultatai ir išvados. Ir nutarta, kad čia bus aprašytos tolimesnių tyrimų kryptys.

\addcontentsline{toc}{section}{SĄVOKŲ APIBRĖŽIMAI}
\section*{SĄVOKŲ APIBRĖŽIMAI}
Mokymasis su mokytoju (angl. supervised learning) - % TODO

Mokymasis be mokytojo (angl. unsupervised learning) - % TODO

Mašininis\cite{mamcenko08} (kompiuterinis, sistemos\cite{martisiute08})
mokymasis (angl. machine learning) - tai mokslas siekiantis įgalinti
kompiuterius atlikti tam tikrus darbus be išreikštinio programavimo.

Hiperplokštuma (angl. hyperplane) - plokštumos generalizacija daugiadimensėje
erdvėje.

Atraminių vektorių klasifikatoriai (angl. support vector machines, SVM) - yra
klasifikavimo su mokymu metodas, taikomas ir klasifikavime, ir regresinei
analizei.\cite{bernataviciene08}

Regrèsija [lot. regressio – grįžimas, traukimasis]: tikimybių teorijoje ir mat.
statistikoje – atsitiktinio dydžio vidurkio priklausomybės nuo kt. dydžio (kelių
dydžių) išraiška;\cite{tzz2010}



\addcontentsline{toc}{section}{LITERATŪRA} 
\bibliographystyle{alpha}
\bibliography{literatura.bib}
\end{comment}


\begin{abstract}
Mano abstractas.

\end{abstract}

\section{}

\end{document}
