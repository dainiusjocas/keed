\subsection{Atraminių vektorių metodai}

Atraminių vektorių klasifikatorius\cite{vapnik2000nature} (angl. support vector
machines) - tai mašininio mokymosi (angl. machine learning) algoritmas išvestas iš
statistinio mokymosi. Jis priskiriamas mokymuisi su mokytoju. Metodas taikomas
ir klasifikavime, ir regresinėje analizėje.

%% JG: cituoti turi originalų darbą:
%% JG: C. Cortes and V. Vapnik, Support-Vector Networks, Machine Learning, 20(3):273-297, September 1995
%% JG: Vladimir N. Vapnik. The Nature of Statistical Learning Theory. Springer, New York, 1995

Naudojant atraminių vektorių klasifikatorių, yra sukuriama hiperplokštuma,
atskirianti duomenis į dvi klases. Hiperplokštuma parenkama tokia, kad atstumas
tarp skirtingų klasių artimiausių elementų ir hiperplokštumos būtų didžiausias.

Konstruojant hiperplokštumą yra spendžiamas optimizavimo su ribojimais
algoritmas.

% TODO išsiaiškinti, kas ten per matematika

Gali būti ir taip, kad ieškoma hiperplokštuma gali ir neegzistuoti pavyzdžiui,
kai klasės stipriai persidengia. Tada įvedamas parametras ir pasikeičia
optimizavimo uždavinys. % TODO dar daugiau matematikos

% Transponavimas - matricos eilučių sukeitimas vietomis su stulpeliais


%SVM is a type of machine learning algorithm derived from statistical learning
%[theory](http://download.oracle.com/docs/cd/B14117_01/text.101/b10729/classify.htm).

Viena iš atraminių vektorių metodų klasifikavimo ypatybių yra gebėjimas mokytis
iš labai mažos mokymosi duomenų aibės.


%% JG: nepamiršksio daugiamatiškumo erdvę, o ten juos galima atskirti tiesiškai.