\section{EKSPERIMENTAI}

Čia aprašysiu eksperimentus

\subsection{Naudoti duomenys}

Šiame darbe eksperimentai buvo atliekami su biomedicininiais viešai prieinamais
genų ekspresijos duomenų rinkiniais.
\begin{longtable}{|p{4cm}|p{1.4cm}|p{2.5cm}|p{1.6cm}|p{1cm}|}
\captionsetup{labelsep=period}
\caption{Darbe naudoti duomenų rinkiniai\label{table:datasets}}\\
%This is the header for the first page of the table...
\hline \hline
{\textbf{Pavadinimas}} &
{\textbf{Šaltinis}} &
{\textbf{Objektų skaičius (+/-)}}&
{\textbf{Dimensijų skaičius}}&
{\textbf{ODS}}\\
\hline
\endfirsthead
%This is the header for the remaining page(s) of the table...
\multicolumn{3}{c}{{\tablename} \thetable{} -- Tęsinys} \\[0.5ex]
\hline \hline
{\textbf{Pavadinimas}} &
{\textbf{Šaltinis}} &
{\textbf{Objektų skaičius (+/-)}}&
{\textbf{Dimensijų skaičius}}&
{\textbf{ODS}}\\
\hline
\endhead
%This is the footer for all pages except the last page of the table...
\multicolumn{3}{l}{{Lentelės tęsinys kitame puslapyje\ldots}} \\
\endfoot
%This is the footer for the last page of the table...
\hline \hline
\endlastfoot
\hline 
Gaubtinės žarnos auglys (angl. Colon) 
& 
\cite{alon1999broad} 
& 
62 (40/22) 
& 
2000 
& 
3,1\% \\
\hline
Centrinės nervų sistemos auglys (CNS) 
& 
\cite{pomeroy2002prediction} 
& 
60 (39 AAL / 21 AML) 
& 
7129 
& 
0.84\% \\
\hline
Prostatos auglys 
& 
\cite{singh2002gene} 
& 
102 (52/50) 
& 
6033 
& 
1.7\% \\
\hline
Šizofrenija ir maniakinė depresija
&
\cite{altara}
&
90 (bp\footnote{bp (angl. Bipolar disorder) - maniakine depresija sergantys
pacientai.}:
sz\footnote{sz (angl. Schizophrenia) - šizofrenija sergantys pacientai.}:
cc\footnote{cc (angl. Control Crowd) - kontrolinė grupė.} =30:31:29)
&
22283
&
0.403\% \\
\hline
\end{longtable}
Duomenų rinkinius apibūdinantis dydis ODS, kuris turimiems duomenims tesiekia
nuo  0,403\% iki 3,01\% procento, parodo, kad turime labai retus (angl. sparce) duomenis, o tai labai apsunkina
mokymosi procesą ir gali sukelti persimokymo (angl. overfitting) problemą.

\subsection{Metodologija}

\subsection{Dimensijų atrinkimo metodų sparta}

\subsection{Dimensijų atrinkimo stabilumas}

\subsection{Pavienių dimensijų atrinkimo metodų stabilumas}

Šiame skyrelyje apžvelgsime pavienių dimensijų atrinkimo motodų stabilumą. 
Stabilumas visiems metodams buvo matuojamas atsižvelgiant į klasifikavimo
rezultatus.

\subsubsection{\textit{Fisher} dimensijų atrinkimo metodas}

\subsubsection{\textit{Relief} dimensijų atrinkimo metodas}

Šis metodas yra vienas nestabiliausių, nes pasikeitimai duomenų rinkinyje 
stipriai įtakoja rezultatus.