\addcontentsline{toc}{section}{SĄVOKŲ APIBRĖŽIMAI}
\section*{SĄVOKŲ APIBRĖŽIMAI}

Klasifikavimas --

Klasifikatorius -- 

Persimokymas (angl. \textit{overfitting}) -- reiškinys, kai klasifikavimo algoritmas per daug prisitaiko prie treniravimosi duomenų. Kitaip tariant, sukurtas klasifikatorius pasižymi aukštu klasifikavimo tikslumu dirbant su treniravimosi duomenimis, tačiau klasifikavimo tikslumas yra žemas dirbant su testiniais duomenimis.

Genėjimas (klasifikavimo medžių)  (angl. \textit{pruning}) - technika, kuri iš klasifikavimo medžių pašalina mazgus, kurie turi salyginai mažą atskiriamąją galią.

Hiperplokštuma (angl. \textit{hyperplane}) -- plokštumos generalizacija daugiamatėje erdvėje.

Normalės vektorius - vektorius, kuris yra statmenas tiesei arba plokštumai.

Triukšmas (angl. \textit{noise}) -- pašaliniai atsitiktiniai signalai, patekę į informaciją nešančių signalų srautą.

Išimtis (angl. \textit{outlier}) -- obejektas, kuris savo skaitine daug didesnis arba daug mažesnis už imties vidurkį.

Mašininis mokymasis (angl. \textit{machine learning}) yra dirbtinio intelekto šaka, kurios tyrėjai siekia įgalinti kompiuterius tobulinti savo elgseną (mokytis) empirinių duomenų atžvilgiu.

Atraminių vektorių klasifikatoriai (angl. support vector machines, SVM) -- mokymosi su mokytoju metodas, taikomas ir klasifikavime, ir regresinei analizei.

Regresija [lot. regressio –- grįžimas, traukimasis] -- tikimybių teorijoje ir mat. statistikoje – atsitiktinio dydžio vidurkio priklausomybės nuo kt. dydžio (kelių dydžių) išraiška \cite{tzz2010}.

Mokymosi duomenys (angl. \textit{training data}) -- duomenys, su kuriais dirbs mašininio mokymosi algoritmai.

Testavimo duomenys (angl. \textit{testing data}) -- duomenys, kuriais bus validuojamas sukurtas klasifikatorius.
