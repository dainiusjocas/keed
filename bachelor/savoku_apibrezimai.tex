Prižiūrimas mokymasis (angl. supervised learning) - % TODO

Neprižiūrimas mokymasis (angl. unsupervised learning) - % TODO

Mašininis\cite{mamcenko08} (kompiuterinis, sistemos\cite{martisiute08})
mokymasis (angl. machine learning) - tai mokslas siekiantis priversti
kompiuterius atlikti tam tikrą darbą be išreikšto programavimo.
\begin{comment}
Machine learning is the science of getting computers to act without being
explicitly programmed.
\end{comment}

Hiperplokštuma (angl. hyperplane) - plokštumos generalizacija daugiadimansėje
erdvėje.

Atraminių vektorių klasifikatoriai (angl. support vector machines, SVM) - yra
klasifikavimo su mokymu metodas, taikomas ir klasifikavime, ir regresinei
analizei.\cite{bernataviciene08}

Regrèsija [lot. regressio – grįžimas, traukimasis]: tikimybių teorijoje ir mat.
statistikoje – atsitiktinio dydžio vidurkio priklausomybės nuo kt. dydžio (kelių
dydžių) išraiška;\cite{tzz2010}
