Klasifikacija \textit{(angl. classification)} -- objektų skirstymas į grupes pagal tam tikrus požymius.
Klasifikatorius - 
Triukšmas \textit{(angl. noise)} -- pašaliniai atsitiktiniai signalai, patekę į informaciją nešančių signalų srautą.
Retas masyvas \textit{(angl. sparse array)} -- duomenų masyvas, kurio dauguma elementų yra nuliai arba nuliui ekvivalenčios reikšmės.
Mokymasis su mokytoju (angl. supervised learning) -- % TODO

Mokymasis be mokytojo (angl. unsupervised learning) -- % TODO

Mašininis\cite{mamcenko08} (kompiuterinis, sistemos\cite{martisiute08})
mokymasis (angl. machine learning) -- tai mokslas siekiantis įgalinti
kompiuterius atlikti tam tikrus darbus be išreikštinio programavimo.

Hiperplokštuma (angl. hyperplane) -- plokštumos generalizacija daugiamatėje erdvėje.

Atraminių vektorių klasifikatoriai (angl. support vector machines, SVM) -- yra
klasifikavimo su mokymu metodas, taikomas ir klasifikavime, ir regresinei
analizei.\cite{bernataviciene08}

Regrèsija [lot. regressio –- grįžimas, traukimasis] -- tikimybių teorijoje ir mat.
statistikoje – atsitiktinio dydžio vidurkio priklausomybės nuo kt. dydžio (kelių
dydžių) išraiška \cite{tzz2010};

Mokymosi duomenys (angl. sample data) -- duomenys,
kurie yra paruošti naudoti algoritmams, kuries kurs modelius (pvz.
klasifikatorius

Normalės vektorius - vektorius, kuris yra statmenas tiesei arba plokštumai.

Genėjimas (klasifikavimo medžių)  (angl. \textit{pruning}) - technika, kuri iš klasifikavimo medžių pašalina mazgus, kurie turi salyginai mažą atskiriamąją galią.

Persimokymas (angl. \textit{overfitting}) -- efektas, kai klasifikavimo algoritmas per daug prisitaiko prie treniravimosi duomenų. Kitaip tariant, sukurtas klasifikatorius pasižymi aukštu klasifikavimo tikslumu dirbant su treniravimosi duomenimis, tačiau klasifikavimo tikslumas yra žemas dirbant su testiniais duomenimis.
