\addcontentsline{toc}{section}{SĄVOKŲ APIBRĖŽIMAI}
\section*{SĄVOKŲ APIBRĖŽIMAI}
\label{savoku_apibrezimai}

Klasifikavimas -- procesas, kurio metu sukonstruojama funkcija, kuri pagal mėginių nepriklausomus kintamuosius apskaičiuoja priklausomus kintamuosius.

Klasifikatorius -- funkcija, kuri pagal mėginių nepriklausomus kintamuosius apskaičiuoja priklausomus kintamuosius.

Jautrumas (angl. \textit{sensitivity}) -- įvertis, kuris parodo testo/modelio gebėjimą dignozuoti susirgimą, jeigu asmuo iš tikrųjų serga (ligonis identifikuojamas ligoniu).

Specifiškumas (angl. \textit{specificity}) -- įvertis, parodantis testo/modelio gebėjimą nustatyti, jog susirgimo nėra, kai jo iš tikrųjų nėra (sveikas identifikuojamas sveiku).

Persimokymas (angl. \textit{overfitting}) -- reiškinys, kai klasifikavimo algoritmas per daug prisitaiko prie treniravimosi duomenų. Sukurtas klasifikatorius pasižymi aukštu klasifikavimo tikslumu dirbant su treniravimosi duomenimis, tačiau klasifikavimo tikslumas yra žemas dirbant su testiniais duomenimis.

Genėjimas (klasifikavimo medžių)  (angl. \textit{pruning}) - mazgų, kurie turi sąlyginai mažą atskiriamąją galią, pašalinimas iš klasifikavimo medžio.

Hiperplokštuma (angl. \textit{hyperplane}) -- plokštumos generalizacija daugiamatėje erdvėje.

Triukšmas (angl. \textit{noise}) -- pašaliniai atsitiktiniai signalai, patekę į informaciją nešančių signalų srautą.

Išimtis (angl. \textit{outlier}) -- objektas, kuris savo skaitine reikšme daug didesnis arba daug mažesnis už imties vidurkį.

Mašininis mokymasis (angl. \textit{machine learning}) -- dirbtinio intelekto šaka, kurios tyrėjai siekia įgalinti kompiuterius tobulinti savo elgseną (mokytis) empirinių duomenų atžvilgiu.

Regresija (lot. \textit{regressio} -- grįžimas, traukimasis) -- tikimybių teorijoje ir mat. statistikoje -- atsitiktinio dydžio vidurkio priklausomybės nuo kt. dydžio (kelių dydžių) išraiška.

Mokymosi duomenys (angl. \textit{training data}) -- duomenys, pagal kuriuos yra kuriamas klasifikatorius.

Testavimo duomenys (angl. \textit{testing data}) -- duomenys, kuriais bus validuojamas sukurtas klasifikatorius.
