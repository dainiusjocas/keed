\section*{SUMMARY}
\label{summary}

This bachelor thesis is dedicated to analysis of high-dimensional small-sample biomedical data. This thesis has overview of machine learning theory which is necessary to do desired analysis. First, because analysis of high-dimensional data rarely done without feature selection, a special emphasis is put on basic and multicriterion feature selection methods. Second, when working with high-dimensional data computation time is always an issue, so, feature selection methods are compared according to their computational performance. Third, classification of high-dimensional data is analyzed by impact of selected features to the classification accuracy and robustness of feature selection. Analysis showed that there is no such thing like the best classification strategy for high-dimensional data biomedical data. That why this work is an introduction for the further studies.


Keywords: high-dimensional data, machine learning, classification, feature selection.
