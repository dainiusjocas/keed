Duomenų kiekis pasaulyje labai sparčiai didėja. Dar daugiau, tie duomenys sudėtingėja. 
Duomenų sudėtingėjimas suponuoja atributų skaičiaus augimą - didėja duomenų
dimensijų skaičius. Dimensijų skaičiaus ,,sprogimas`` ypač pastebimas 
biologiniuose duomenyse. Taip yra todėl, nes atsiranda vis naujesni būdai 
apdoroti biologinius duomenis, pavyzdžiui, genomo sekvenavimas ar epigenitinė
analizė. Sekvenuojant genomą moderniomis priemonėmis galima gauti net
keletą milijonų dimensijų vienam genui(!). Todėl naujiems duomenims klasifikuoti
reikia ir naujų klasifikavimo strategijų, nes tradicinės yra tiesiog nepajėgios dirbti
su tokiu milžinišku dimensijų skaičiumi - drastiškai didėja skaičiavimo laikas, 
mažėja klasifikavimo tikslumas.
 
Viena iš strategijų dirbti su didelio dimensiškumo duomenis yra naudoti įvairius
dimensijų atrinkimo\cite{guyon2003introduction} (angl. feature selection) metodus.
Tokie metodai nėra naujiena, jie jau kuris laikas
yra naudojami mašininio mokymosi taikymuose, bet išradingai naudojant jie gali itin svariai 
prisidėti prie naujų klasifikavimo strategijų, reikalingų darbui su daugiamačiais 
duomenimis. Dimensijų atrinkimo etapas yra
svarbus duomenų paruošimui. Jis ne tik leidžia sukurti tikslesnius klasifikavimo
modelius, taip pat jis padeda geriau suprasti ir vizualizuoti tais duomenimis
apibūdinamus procesus, sumažinti atminties poreikį, pagreitinti mokymosi ir 
sumažinti klasifikatorių darbo laiką. Tačiau ir dimensijų atrinkimo metodai
turi būti pakankamai našūs, todėl ne visi yra naudotini.

Gyvybėms mokslų tyrėjams itin svarbu fokusuoti į mažesnį dimensijų skaičių,
nes tai itin paspartina jų tyrimus - jiems reikia tirti mažesnį skaičių 
mėginių. Mažesnio skaičiaus mėginių tyrimas ir kainuoja mažiau, nes mažiau 
reikia ir cheminių reagentų ir darbo laiko. Bet čia iškyla naujas dimensijų atrinkimo 
kriterijus - stabilumas (ang. robustness). Stabilumas didina tikimybę, kad atrinktosios 
dimensijos yra tikrai susijusios su nagrinėjama problema. Jei dimensijų atrinkimo
rezultatai labai stipriai varijuos, tai tada reikės atlikinėti bandymus su visa
duomenų aibe, kas yra labai neefektyvu. Pastebėtina, kad stabilumo matavimai turi
būti atliekami būtinai atsižvelgiant į klasifikavimo tikslumą.

Dar viena problema dirbant su daugiamačiais duomenimis yra tai, kad dažnai
turimas labai ribotas skaičius duomenų objektų. Dimensijų - objektų santykis 
neretai skiriasi visomis eilėmis. Atrodytų, tik laiko klausimas, kada bus paruošta
daugiau objektų, bet žvelgiant į duomenų gavybos tendencijas pasidaro aišku, kad
objektų skaičius niekada nepavys dimensijų skaičiaus. Todėl reikia labai apgalvoti,
kaip bus į tai atsižvelgta kuriant klasifikavimo modelius. Nes esant mažam
objektų skaičiui kyla grėsmė susidurti su persimokymo problema (angl. 
overfitting). Tokiu atveju klasifikatorius tampa bevertis.

Duomenyse daugėja dimensijų, todėl neišvengiamai daugėja ir triukšmo. 
Triukšmas atsiranda
dėl įvairių priežasčių, pavyzdžiui, cheminiai preparatai buvo ne visai tinkamai
paruošti. Triukšmas duoda atsitiktinius rezultatus, iš kurių naudos nedaug.
Atsiranda poreikis identifikuoti triukšmingus duomenis ir juos išmesti iš 
klasifikavimo proceso.

Taigi, žvelgiant iš paukščio skrydžio, galime pastebėti, kad norint optimaliai
dirbti su daugiamačiais duomenimis vienu metu reikia atsižvelgti į eilę kriterijų:
\begin{enumerate}
 \item Klasifikavimo tikslumas - tipinė klasifikavimo užduotis yra atskirti
 sergančius pacientus nuo sveikų.;
 \item Dimensijų atrinkimo stabilumas, atsižvelgiant į klasifikavimo rezultatus;
 \item Didelis dimensijų-objektų santykis;
 \item Triukšmo lygis duomenyse;
 \item Skaičiavimo išteklių optimalus naudojimas.
\end{enumerate}
Reikalavimas vienu metu atsižvelgti į keletą kriterijų labai apsunkina užduotį.
Daugiamačių duomenų klasifikavime pragrindinė problema yra surasti optimalų 
metodą, kuris geriausiai atsižvelgia į aukščiau minėtus kriterijus.

Šiame darbe dirbsime su viešai prieinamomais genų ekspresijos duomenų rinkiniais.
\begin{longtable}{|p{4cm}|p{1.4cm}|p{2.5cm}|p{1.6cm}|p{1cm}|}
\captionsetup{labelsep=period}
\caption{Darbe naudoti duomenų rinkiniai\label{table:datasets}}\\
%This is the header for the first page of the table...
\hline \hline
{\textbf{Pavadinimas}} &
{\textbf{Šaltinis}} &
{\textbf{Objektų skaičius (+/-)}}&
{\textbf{Dimensijų skaičius}}&
{\textbf{ODS}}\\
\hline
\endfirsthead
%This is the header for the remaining page(s) of the table...
\multicolumn{3}{c}{{\tablename} \thetable{} -- Tęsinys} \\[0.5ex]
\hline \hline
{\textbf{Pavadinimas}} &
{\textbf{Šaltinis}} &
{\textbf{Objektų skaičius (+/-)}}&
{\textbf{Dimensijų skaičius}}&
{\textbf{ODS}}\\
\hline
\endhead
%This is the footer for all pages except the last page of the table...
\multicolumn{3}{l}{{Lentelės tęsinys kitame puslapyje\ldots}} \\
\endfoot
%This is the footer for the last page of the table...
\hline \hline
\endlastfoot
\hline 
Gaubtinės žarnos auglys (angl. Colon) 
& 
\cite{alon1999broad} 
& 
62 (40/22) 
& 
2000 
& 
3,1\% \\
\hline
Centrinės nervų sistemos auglys (CNS) 
& 
\cite{pomeroy2002prediction} 
& 
60 (39 AAL / 21 AML) 
& 
7129 
& 
0.84\% \\
\hline
Prostatos auglys 
& 
\cite{singh2002gene} 
& 
102 (52/50) 
& 
6033 
& 
1.7\% \\
\hline
Šizofrenija ir maniakinė depresija
&
\cite{altara}
&
90 (bp\footnote{bp (angl. Bipolar disorder) - maniakine depresija sergantys pacientai.}:
sz\footnote{sz (angl. Schizophrenia) - šizofrenija sergantys pacientai.}:
cc\footnote{cc (angl. Control Crowd) - kontrolinė grupė.} =30:31:29)
&
22283
&
0.403\% \\
\hline
\end{longtable}
Duomenų rinkinius apibūdinantis dydis ODS\footnote{ODS - Objektų dimensijų santykis},
kuris turimiems duomenims tesiekia nuo  0,403\% iki 3,01\% procento,
parodo, kad turime labai retus (angl. scarce) duomenis, o tai labai apsunkina
mokymosi procesą ir gali sukelti persimokymo (angl. overfitting) problemą.

Skaičiavimo išteklius, kurių reikėjo ne taip ir mažai, suteikė VU MIF ITTC.
Programavimo darbai buvo atlikti naudojant R programavimo kalbą. Didžiąją
dalį eksperimentų atlikau profesinės praktikos MII metu.

Su teorine dalykinės srities medžiaga susipažinau skaitydamas darbo vadovo 
rekomenduotus mokslinius straipsnius, bei naudodamasis internetine paieška.

Šio darbo tikslas yra išanalizuoti darbo su daugiamačiais duomenis ypatybes.

Šiam darbui yra keliamos tokios užduotys:
\begin{enumerate}
 \item Apžvelgti esamus klasifikavimo metodus;
 \item Išanalizuoti populiariausių dimensijų atrinkimo metodų spartą;
 \item Išanalizuoti populiariausių dimensijų atrinkimo metodų stabilumą;
 \item Išanalizuoti kaip dimensijų atrinkimo metodai įtakoja klasifikatorių tikslumą.
\end{enumerate}

