Duomenų kiekiai labai sparčiai didėja. Dar daugiau, tie duomenys sudėtingėja. 
Duomenų sudėtingėjimas suponuoja atributų skaičiaus augimą - didėja duomenų
dimensijų skaičius. Dimensijų skaičiaus ,,sprogimas`` ypač pastebimas 
biologiniuose duomenyse. Taip yra todėl, nes atsiranda vis naujesni būdai 
apdoroti biologinius duomenis, pavyzdžiui, genomo sekvenavimas ar epigenitinė
analizė. Sekvenuojant genomą moderniomis priemonėmis galima gauti net
keletą milijonų dimensijų vienam genui(!). Todėl naujiems duomenims klasifikuoti
reikia ir naujų klasifikavimo strategijų, nes tradicinės yra nepajėgios dirbti
su tokiu dideliu dimensijų skaičiumi - drastiškai didėja skaičiavimo laikas, 
mažėja klasifikavimo tikslumas.
 
Viena iš strategijų dirbti su didelio dimensiškumo duomenis yra naudoti įvairius
dimensijų atrinkimo metodus. Tokie metodai nėra naujiena, jie jau kuris laikas
yra naudojami mašininio mokymosi taikymuose, bet jie gali itin svariai 
prisidėti prie naujų klasifikavimo strategijų, reikalingų darbui su daugiamačiais 
duomenimis. Dimensijų atrinkimo etapas yra
svarbus duomenų paruošimui. Jis ne tik leidžia sukurti tikslesnius klasifikavimo
modelius, taip pat jis padeda geriau suprasti ir vizualizuoti tais duomenimis
apibūdinamus procesus, sumažinti atminties poreikį, pagreitinti mokymosi ir 
klasifikatorių naudojimo laiką. 

Kadangi duomenų kiekiai yra didžiuliai, tai jiems apdoroti reikia labai daug
skaičiavimo resursų. Atsiranda natūralus poreikis kažkaip optimizuoti 
skaičiavimus. Klasifikavimo procesas yra imlus skaičiavimo resursams, todėl 
sumažinus dimensijų skaičių galime gauti nemažą spartdidina tikimybę, kad atrinktosios 
dimensijos yra tikrai susijusios su nagrinėjama problema. os prieaugį, tačiau ir 
dimensijų atrinkimo metodai turi būti pakankamai našūs, todėl ne visi yra
naudotini.

Gyvybėms mokslų tyrėjams itin svarbu fokusuoti į mažesnį dimensijų skaičių,
nes tai itin paspartina jų tyrimus - jiems reikia tirti mažesnį skaičių 
mėginių. Mažesnio skaičiaus mėginių tyrimas ir kainuoja mažiau, nes mažiau 
reikia ir cheminių reagentų ir laiko. Bet čia iškyla naujas dimensijų atrinkimo 
kriterijus - stabilumas (ang. robustness). Stabilumas didina tikimybę, kad atrinktosios 
dimensijos yra tikrai susijusios su nagrinėjama problema. Jei dimensijų atrinkimo
rezultatai labai stipriai varijuos, tai tada reikės atlikinėti bandymus su visa
duomenų aibe, kas yra labai neefektyvu. Dar daugiau, stabilumo matavimai turi
būti atliekami būtinai atsižvelgiant į klasifikavimo tikslumą.

Dar viena problema dirbant su daugiamačiais duomenimis yra tai, kad dažnai
turimas labai ribotas skaičius duomenų objektų. Dimensijų - objektų santykis 
neretai skiriasi visomis eilėmis. Atrodytų, laiko klausimas, kada bus paruošta
daugiau objektų, bet žvelgiant į duomenų gavybos tendencijas pasidaro aišku, kad
objektų skaičius neikada nepavis dimensijų skaičiau. Todėl reikia labai apgalvoti,
kaip bus į tai atsižvelgta kuriant klasifikavimo modelius. Nes esant mažam
objektų skaičiui kyla grėsmė susidurti su persimokymo problema (angl. 
overfitting). Tokiu atveju klasifikatorius tampa bevertis.

Duomenyse daugėja dimensijų - neišvengiamai daugėja ir triukšmo. Triukšmas atsiranda
dėl įvairių priežasčių, pavyzdžiui, cheminiai preparatai buvo ne visai tinkamai
paruošti. Triukšmas duoda atsitiktinius rezultatus, iš kurių naudos nedaug.
Atsiranda poreikis identifikuoti triukšmingus duomenis ir juos išmesti iš 
klasifikavimo proceso.

Taigi, žvelgiant iš paukščio skrydžio, galime pastebėti, kad norint optimaliai
dirbti su daugiamačiais duomenimis vienu metu reikia atsižvelgti į eilę kriterijų:
\begin{enumerate}
 \item klasifikavimo tikslumas - tipinė klasifikavimo užduotis yra atskirti
 sergančius pacientus nuo sveikų.;
 \item dimensijų atrinkimo stabilumas, atsižvelgiant į klasifikavimo rezultatus;
 \item didelis dimensijų-objektų santykis;
 \item triukšmo lygis duomenyse;
 \item skaičiavimo išteklių naudojimas.
\end{enumerate}
Reikalavmas vienu metu atsižvelgti į keletą kriterijų labai apsunkina užduotį.
Daugiamačių duomenų klasifikavime pragrindinė problema yra surasti optimalų 
metodą, kuris geriausiai atsižvelgia į aukščiau minėtus kriterijus.

Šiame darbe dirbsime su viešai prieinamomis genų ekspresijos duomenų bazėmis.
\textcolor{red}{***TODO: lentelė apie datasetus***}
Skaičiavimo išteklius, kurių reikėjo ne taip ir mažai, suteikė VU MIF ITTC.
Programavimo darbai buvo atlikti naudojant R programavimo kalbą. Didžiąją
dalį eksperimentų atlikau profesinės praktikos MII metu.

Su teorine dalykinės srities medžiaga susipažinau skaitydamas darbo vadovo 
rekomenduotus mokslinius straipsnius, bei naudodamasis internetine paieška.

Šio darbo tikslas yra išanalizuoti darbo su daugiamačiais duomenis ypatybes.

Šiam darbui yra keliamos tokios užduotys:
\begin{enumerate}
 \item Apžvelgti esamus klasifikavimo metodus;
 \item Išanalizuoti populiariausių dimensijų atrinkimo metodų spartą;
 \item Išanalizuoti populiariausių dimensijų atrinkimo metodų stabilumą;
 \item Išanalizuoti kaip dimensijų atrinkimo metodai įtakoja klasifikatorių tikslumą.
\end{enumerate}



