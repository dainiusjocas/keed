%% Įvade aprašomi darbo tikslai, nurodomas temos aktualumas, aptariamos teorinės darbo prielaidos bei metodika, apibrėžiamas tiriamasis objektas, apibūdinami su tema susiję literatūros ar kitokie šaltiniai, temos analizės tvarka, darbo atlikimo aplinkybės, pateikiama žinių apie naudojamus instrumentus (programas ir kt.). Darbo įvadas neturi būti dėstymo santrauka. Įvado apimtis 3–4 puslapiai.

\newpage
\section*{ĮVADAS}

%% JG: Idėja - naudok terminus "matai" ir "mėginiai". Kiekvienam mėginiui atliekama labai daug matavimų, iš to ir "daugiamatiškumas"

% JG: Paminėk, kad šiame darbe mes gilinamės į biomedicinoje kaupiamų genetinių duomenų analizės specifika. Šie duomenys ypatingi dėl daugiamatiškumo, mažo mėginių kiekio, triukšmingų matavimų.

Nuolat vystosi technologijos skirtos gauti biomedicininius duomenis, pvz. genomo sekvenavimas \cite{pettersson2009generations}, tai reiškia, kad didėja gaunamų duomenų detalumas. Detalumas reiškia, kad daugėja biomedicininius duomenis abibūdinančių faktorių arba matų skaičius. Duomenys, kurių kiekvienas mėginys aprašomas dideliu skaičiumi matų, yra vadinami daugiamačiais duomenimis.

Šiame darbe yra nagrinėjama biomedicinoje kaupiamų genetinių daugiamačių duomenų analizės specifika. Šie duomenys yra specifiški tuo, kad jie turi šimtus kartų daugiau matų nei mėginių. Kadangi mėginio gavimo kaina yra aukšta, turimas mažas mėginių skaičius. Biomedicininių duomenų analizę apsunkina ir tai, kad matavimai, kuriais tie duomenys gaunami, yra triukšmingi. Triukšmas matavimo metu atsiranda dėl cheminių reakcijų netikslumo, tiriamo organizmo sudėtingumo. Kai duomenys yra triukšmingi, tai didėjant juos apibūdinančių matų skaičiui, didėja tikimybė duomenyse, kad bus rasta atsitiktinių priklausomybių. Tai yra pagrindinė priežastis, kodėl biomedicininių duomenų analizės procesas yra sudėtingas.

%% JG: Būdai neatsiranda, o vystosi technologija. Jie nėra tikslesni, bet detalesni, t.y. kiekvienam mėginiui atliekama daugiau matavimų. 

%% JG: Nors matavimų kiekis didėja, mėginio kaina išlieka gana aukšta. Todėl biomedicinos eksperimentuose gaunami duomenys ypatingi tuo, kad matų visados ženkliai daugiau nei mėginių.

Biomedicininių duomenų klasifikavimo užduotis yra atskirti sveikųjų pacientų mėginius nuo sergančiųjų. Klasifikavimu siekiama nustatyti, kurie matai, veikdami drauge, geriausiai paaiškina skirtumą tarp ligos paveiktų ir nepaveiktų mėginių. Labiausiai ligą paaiškinančių matų nustatymas galėtų palengvinti tiriamų ligų diagnozės ir gydymo metodų kūrimą. Klasifikavimu yra vadinamas duomenų analizės procesas, kurio metu yra sukonstruojama funkcija, atskirianti duomenis į grupes (arba klases) pagal jų matus \cite{fisher1936use}. Sukonstruotos funkcijos yra vadinamos klasifikatoriais, o jų konstravimo algoritmai -- klasifikavimo algoritmais. Klasifikatoriai paruošiami naudojant turimus mėginius -- treniravimosi duomenis -- ir informaciją apie jų būklę (sveikas ar sergantis). Klasifikatoriaus ruošimo procesas yra vadinamas apmokymu. Klasifikatoriai yra validuojami su testiniais duomenimis, o naudojami nustatant naujų, dar nematytų, mėginių būklę.

Dėl ,,daugiamatiškumo prakeiksmo`` (angl. \textit{the curse of dimentionality}) didėjant matų kiekiui mėginiai pasidaro panašūs, todėl bandymas juos klasifikuoti tolygus spėliojimui \cite{bellman1966adaptive}. Biomedicininių duomenų kontekste galima daryti prielaidą, kad ne visi matai yra susiję su tiriama problema, pvz. gaubtinės žarnos vėžiu, dėl to, kad duomenys yra daugiamačiai. Paprastai nagrinėjamai problemai svarbus yra mažas, palyginus su visu, matų kiekis.  Todėl biomedicininių duomenų daugiamatiškumui sumažinti yra naudojami informatyviausių matų atrinkimo (angl. \textit{feature selection}) metodai \cite{guyon2003introduction}. Pagal tai, kaip susiję su klasifikatoriumi, matų atrinkimo metodai skirstomi į tris kategorijas \cite{saeys2008robust}: filtruojantys (angl. \textit{filter}), prisitaikantys (angl. \textit{wrapper}) ir įterptiniai (angl. \textit{embedded}) metodai. Filtruojančiais metodais pirmiausia yra atrenkami informatyviausi matai, o tada su jais apmokomas klasifikatorius. 
Prisitaikančiųjų 
metodų atveju, pirma, apmokomas klasifikatorius su visais matais, antra, parenkamas matų poaibis ir apmokomas klasifikatorius, tada po daugkartinio matų aibių įvertinimo pagal klasifikavimo rezultatus yra nusprendžiama, kuris matų poaibis yra labiausiai tinkamas klasifikavimui. Įterptinių metodų atveju matų atrinkimo procesas yra neatsiejamas nuo klasifikavimo proceso -- pats klasifikatorius įvertina matus.

Matų atrinkimas yra svarbi biomedicininių duomenų apdorojimo (angl. \textit{preprocessing}) etapo dalis. Naudojant matų atrinkimo metodus, galima kovoti su daugiamatiškumo prakeiksmu matų skaičių priartinant prie mėginių skaičiaus. Todėl svarbu yra pasirinkti geriausiai tinkančią matų atrinkimo strategiją. Kadangi pačių matų atrinkimo metodų veikimas priklauso nuo konkrečių duomenų, taip pat matų atrinkimo algoritmą reikia derinti ir prie sprendžiamo uždavinio, tai paties matų atrinkimo metodo pasirinkimas yra sudėtinga užduotis. 

Dirbant su biomediciniais duomenimis dažniausiai turime tik kelias dešimtis mėginių, todėl, norint geriau įvertinti klasifikatoriaus tikslumą, yra naudojami pakartotinio mėginių poaibio atrinkimo (angl. \textit{resampling}) metodai: kryžminio patikrinimo (angl. \textit{cross-validation}) arba įkelčių (angl. \textit{bootstrap\footnote{Terminas \textit{bootstrap} ,,įkelties`` prasme pradėtas naudoti dar Rudolfo Ericho Raspės knygoje ,,Barono Miunchauzeno nuotykiai``(1785), kurioje Baronas Minchauzenas užkėlė save ant arklio tempdamas į viršų savo batų raištelius (angl. \textit{bootstraps}).}}). Šių metodų naudojimas su duomenimis, kurių tikrasis pasiskirstymas nėra žinomas, padeda įvertinti klasifikavimo rezultatų variabilumą (angl. \textit{variance}) ir sisteminį nuokrypį (angl. \textit{bias}).

Naudojant kryžminio patikrinimo metodą, daug kartų sudaromos skirtingos treniravimosi ir testinės mėginių imtys. Taikant atskirą šio metodo variantą, kryžminį patikrinimą paliekant vieną mėginį (angl. \textit{leave-one-out cross-validation}), iš mokymosi duomenų išimamas vienas (testinis) mėginys, o su likusiais apmokomas klasifikatorius, kuris klasifikuoja išbrauktąjį mėginį. Procesas tęsiamas tol, kol suklasifikuojami visi objektai. Kitais kryžminio patikrinimo metodo variantais iš treniravimosi mėginių yra išimama po keletą mėginių. Pagal tai, kiek testinių mėginių klasifikatorius priskyrė klaidingai kategorijai, yra nustatoma vidutinė klaidingo klasifikavimo tikimybė. Šiuo metodu gauti įverčiai pasižymi dideliu klasifikavimo rezultatų variabilumu \cite{braga2004cross}.

Naudojant įkelčių metodą, iš $N$ dydžio mėginių aibės yra paimama tokio pačio dydžio atsitiktinių mėginių imtis su pasikartojimais, kuri vadinama įkelties treniravimosi imtimi. Į šią imtį nepaimti mėginiai yra priskiriami testavimo imčiai. Naudojant įkelties treniravimosi mėginių imtį yra apmokomas klasifikatorius, kuris klasifikuoja testavimo imtį. Procesą kartojant gaunama klasifikavimo nuostolių įverčių imtis. Šios imties vidurkis yra vidutinis klasifikavimo nuostolio įvertis. Dažniausiai naudojamas ,,0.623 įkelčių`` (angl. \textit{0.623\footnote{0.623 yra tikimybė mėginiui būti įtrauktam į treniravimosi imtį.} bootstrap}) metodas. Šiuo metodu gautas vidutinio klasifikavimo nuostolio įvertis pasižymi mažu variabilumu \cite{michie1994machine}.

%% JG: Kodėl klasifikuojama? Norima nustatyti, kokie matai, veikdami drauge, geriausiai paaiškina skirtumą tarp ligos paveiktų ir sveikų mėginių.

%% JG: Kokios yra tradicinės klasifikavimo strategijos ir kodėl jos neveikia daugiamačių duomenų atveju? Skaičiavimo laikas nėra problema - Random forests veikia visai neblogai tokiais atvejais. Daugiamatiškumas veda prie the curse fo dimensionality.

% JG: sumažinus naudojamų matų kiekį, matų kiekis priartėja prie mėginių kiekio ir tokiu būdu apeinamas daugiamatiškumo prakeiksmo problema. Biomedicinos duomenų kontekste, galima daryti prielaidą, kad dauguma matų yra beprasmiai, pvz., tik kai kurie genai įtakoja ligą, todėl matų mažinimimas yra prasmingas. Taip pat, kuriant medicininius diagnostikos įrankius, naudojamų matų kiekis įtakoja įrankio kainą. Todėl pageidautina turėti kuo mažiau matų.

Kadangi biomedicininiuose duomenyse reikšmingų matų kiekis tiriamai problemai yra nedidelis, todėl tyrėjams norint geriau suprasti nagrinėjamus biomedicininius duomenis yra svarbu orientuotis į mažesnį matų poaibį, kuris yra svarbus nagrinėjamai problemai. Tokioje situacijoje tampa svarbu, kaip varijuoja atrenkamų matų aibė, kai matų atrinkimas vykdomas su vis kitu mėginių poaibiu. Matai, kurie keičiant mėginių, naudojamų matų atrinkime, poaibį yra vėl ir vėl atrenkami, yra vadinami stabiliais matais \cite{devijver1982pattern}. Parametras, parodantis kaip stabiliai yra atrenkami matai, yra vadinamas stabilumu \textit{(angl. robustness)}. Tačiau skirtingi matų atrinkimo metodai tiems patiems mėginiams gali atrinkti skirtingus matus. Taip pat, suskaidžius duomenis į persidengiančius poaibius ir atrinkus tą patį kiekį matų tuo pačiu metodu, gaunamas skirtingas matų poaibis. Matų aibės sumažinimas paspartina biomedicininių duomenų tyrimus -- tyrėjams reikia atlikinėti bandymus su mažesniu mėginių skaičiumi, taip 
pat kuriant medicininius diagnostikos įrankius, naudojamų matų kiekis įtakoja įrankio kainą. Todėl stabilių matų atrinkimas dirbant su biomedicininiais duomenimis yra svarbus.

% JG: Žiūrėkim į stabilumą kaip į šalutinį matų atrinkimo efektą, kurį svarbu pažaboti. Stabilumas svarbus, nes, analizuojant duomenis norima ne tik nustatyti, koks būtų vidutinis klasifikatoriaus tikslumas, bet ir sukurti tą vidutinį klasifikatorių. Norint pastarąjį sukurti, reikia žinoti, kuriuos konkrečius matus naudoti. 

% JG: Šitam paragrafe suplaki daug svarbių dalykų į krūvą ir juos turėtum būti paaiškinęs jau anksčiau. Mėginių trūkumas nėra tavo sprendžiama problema. Tačiau, kuo mažiau duomenų, tuo nestabilesni atrenkami matai. 

%Taigi, dirbant su daugiamačiais duomenimis, reikia atsižvelgti į keletą kriterijų:
%\begin{enumerate}
% \item Klasifikavimo tikslumą;
% \item Matų atrinkimo stabilumą, atsižvelgiant į klasifikavimo rezultatus;
% \item Triukšmo lygį duomenyse;
% \item Skaičiavimo išteklių naudojimo racionalumą.
%\end{enumerate}
%Reikalavimas vienu metu atsižvelgti į keletą kriterijų užduotį daro sudėtinga. Klasifikuojant daugiamačius duomenis uždavinys yra surasti geriausius rezultatus duodančią strategiją, kuri geriausiai atsižvelgia į minėtus kriterijus.

%Darbo eksperimentinei daliai reikalingus skaičiavimo išteklius, suteikė VU MIF skaitmeninių tyrimų ir skaičiavimų centras \cite{mif2012stsc}. Eksperimentuose buvo naudojami laisvai internete prieinami biomedicininių duomenų rinkiniai (angl. \textit{datasets}). Biomedicininių duomenų apdorojimo algoritmų implementavimui buvo naudojama R \cite{r2012statistics} programavimo kalba. Eksperimentai atlikti profesinės praktikos MII metu.

Matų atrinkimo stabilumo problemą Yang ir Mao \cite{yang2011robust} siūlė spręsti reitinguojant matus remiantis keletos matų atrinkimo metodų rezultatais. Galutinis matų reitingo sąrašas gaunamas, kai po kiekvieno matų atrinkimo yra išmetama viena žemiausią reitingą turintis matas iš matų aibės, ir matų atrinkimas yra kartojamas tol, kol nebelieka matų. Tačiau matų atrinkimo metodų kiekis yra ribotas ir skirtingų metodų dažnai negalima vykdyti išskirstytų skaičiavimų aplinkoje. Tai riboja šio metodo pritaikomumą daugiamačių duomenų analizėje.

Matų atrinkimo stabilumo problemą siūlyta spręsti surandant matų grupių tankio centrus ir naudoti matus, kurie artimiausi tiems centrams \cite{yu2008stable}. Pasiūlytas grupių tankių algoritmas užtrunka $O(\lambda n^2m)$ laiko, kur $n$ yra matų kiekis, o $m$ - mėginių skaičius. Vėliau Loscalzo ir kt. pasiūlė mokymo duomenis skaidyti poaibiais ir kiekviename poaibyje ieškoti tankių grupių, o tada imti sprendimą balsavimo principu \cite{loscalzo2009consensus}. Nors šie metodai siūlo stabilų matų atrinkimą, tačiau šių metodų panaudojamumą daugiamačiuose duomenyse riboja skaičiavimo sudėtingumas.

Šiame bakalauriniame darbe remiantis Yang, Mao bei Loscalzo darbuose pateiktomis įžvalgomis, bus stengiamasi pasiūlyti tyrimų kryptis, kurios galėtų padėtų sukurti metodus, skirtus spręsti stabilių matų atrinkimo problemą. Idėja yra sugrupuoti matus pagal greitą klasterizacijos algoritmą, išrinkti reprezentatyviausius matus, transformuoti matų erdvę ir joje vykdyti matų atrinkimą remiantis keletu matų atrinkimo metodų.

Šio darbo tikslas yra išanalizuoti daugiamačių duomenų klasifikavimo ypatybes. Šiam darbui yra keliamos tokios užduotys:
\begin{enumerate}
 \item Susipažinti su naujausiais klasifikavimo ir matų atrinkimo metodais;
 \item Atlikti matų atrinkimo metodų palyginamuosius eksperimentus;
 \item Pasiūlyti kryptis, kaip dabartiniai metodai gali būti patobulinti ir paruošti naujųjų metodų prototipus.
\end{enumerate}

Tolimesnė darbo struktūra yra tokia: skyriuje nr.\ref{darbu_apzvalga} apžvelgiami su atliekamu tyrimu susiję darbai bei reikalinga teorinė medžiaga; skyriuje nr.\ref{pagrindiniai_matu_atrinkimo_motodai} gilinamasi į pagrindinius matų atrinkimo metodus; skyriuje nr.\ref{stabiliu_matu_atrinkimo_metodai} apžvelgiamos matų atrinkimo stabilumą didinančios matų atrinkimo strategijos; skyriuje nr.\ref{eksperimentai} aprašyti daugiamačių duomenų analizės metu gauti duomenys.

