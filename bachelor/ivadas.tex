\newpage
\section*{ĮVADAS}

%% JG: Idėja - naudok terminus "matai" ir "mėginiai". Kiekvienam mėginiui atliekama labai daug matavimų, iš to ir "daugiamatiškumas"


% JG: Paminėk, kad šiame darbe mes gilinamės į biomedicinoje kaupiamų genetinių duomenų analizės specifika. Šie duomenys ypatingi dėl daugiamatiškumo, mažo mėginių kiekio, triukšmingų matavimų.

Atsiranda vis tikslesni būdai, pvz. genomo sekvenavimas, gauti 
biomedicininius duomenis. Tikslumas šiame kontekste reiškia, kad didėja 
atributų, dažniau
vadinamų duomenų dimensijomis, skaičius. Sekvenuojant genomą moderniomis 
priemonėmis galima gauti net keletą milijonų dimensijų vienam genui(!).
Duomenys turintys daug dimensijų yra vadinami daugiamačiais duomenimis.


%% JG: Būdai neatsiranda, o vystosi technologija. Jie nėra tikslesni, bet detalesni, t.y. kiekvienam mėginiui atliekama daugiau matavimų. 

%% JG: Nors matavimų kiekis didėja, mėginio kaina išlieka gana aukšta. Todėl biomedicinos eksperimentuose gaunami duomenys ypatingi tuo, kad matų visados ženkliai daugiau nei mėginių.

Norint daugiamačius duomenis suskirstyti į pageidaujamas kategorijas, pvz. 
atskirti sergančius nuo nesergančių pacientų, pagal vidinę duomenų struktūrą 
(šis procesas vadinamas klasifikavimu) reikia specifinių klasifikavimo 
strategijų. To reikia todėl,
kad tradiciniai klasifikavimo būdai yra nepajėgūs dirbti su daugiamačiais
duomenimis. Taikant tradicines klasifikavimo strategijas daugiamačiams 
duomenims skaičiavimo laikas pasidaro nebepriimtinas, mažėja klasifikavimo
tikslumas. 
 
%% JG: Kodėl klasifikuojama? Norima nustatyti, kokie matai, veikdami drauge, geriausiai paaiškina skirtumą tarp ligos paveiktų ir sveikų mėginių.

%% JG: Kokios yra tradicinės klasifikavimo strategijos ir kodėl jos neveikia daugiamačių duomenų atveju? Skaičiavimo laikas nėra problema - Random forests veikia visai neblogai tokiais atvejais. Daugiamatiškumas veda prie the curse fo dimensionality.

Viena iš strategijų dirbti su daugiamačiais duomenimis yra mažinti duomenų 
dimensijų skaičių - naudoti dimensijų atrinkimo \textit{(angl. feature
selection)}  metodus. Naudojant dimensijų atrinkimo metodus galima 
supaprastinti duomenis atsirenkant tik tas dimensijas, kurios yra svarbios 
konkrečiai tiriamai problemai, pvz. nustatant ligos priežastis. Dimensijų 
atrinkimas yra svarbi duomenų apdorojimo \textit{(angl. preprocessing)} etapo
dalis. Pasirinkus našius dimensijų atrinkimo metodus galima sukurti tikslesnius
klasifikavimo modelius, sumažinti skaičiavimams reikalingų resursų poreikį,
pagreitinti klasifikavimo modelio kūrimo - mokymosi - procesą, taip pat jis 
padeda vizualizuoti bei geriau suprasti tais duomenimis apibūdinamus procesus.

% JG: sumažinus naudojamų matų kiekį, matų kiekis priartėja prie mėginių kiekio ir tokiu būdu apeinamas daugiamatiškumo prakeiksmo problema. Biomedicinos duomenų kontekste, galima daryti prielaidą, kad dauguma matų yra beprasmiai, pvz., tik kai kurie genai įtakoja ligą, todėl matų mažinimimas yra prasmingas. Taip pat, kuriant medicininius diagnostikos įrankius, naudojamų matų kiekis įtakoja įrankio kainą. Todėl pageidautina turėti kuo mažiau matų.

Norint geriau suprasti biomedicininius duomenis itin svarbu fokusuoti dėmesį į
sąlyginai nedidelį dimensijų poaibį. Dimensijų aibės sumažinimas paspartina
biomedicininių duomenų tyrimus - tyrėjams reikia atlikinėti bandymus su
mažesniu mėginių skaičiumi. Mažesnio skaičiaus mėginių tyrimas kainuoja
mažiau, nes mažiau reikia žmonių darbo laiko, mažiau reikia ir cheminių
reagentų. Tačiau tyrėjams ne mažiau svarbu yra žinoti kaip tų atrinktųjų
dimensijų aibė varijuoja priklausomai nuo nedidelių pasikeitimų pačioje
duomenų aibėje - šis kriterijus yra vadinamas dimensijų atrinkimo stabilumu
\textit{(angl. robustness)}.

% JG: Žiūrėkim į stabilumą kaip į šalutinį matų atrinkimo efektą, kurį svarbu pažaboti. Stabilumas svarbus, nes, analizuojant duomenis norima ne tik nustatyti, koks būtų vidutinis klasifikatoriaus tikslumas, bet ir sukurti tą vidutinį klasifikatorių. Norint pastarąjį sukurti, reikia žinoti, kuriuos konkrečius matus naudoti. 

Jei atrinktųjų dimensijų poaibis, kurį naudojant duomenų objektai yra išskirstomi
teisingoms kategorijoms (klasifikavimo rezultatai yra tikslūs),
yra stabilus, tai reiškia, kad tikslinga detalius bandymus su duomenimis 
pradėti nuo atrinktojo stabilaus dimensijų poaibio. Kitu atveju bandymus 
atlikinėti reikia su visa duomenų aibe, kas yra mažiau efektyvu. Pastebėtina, kad
stabilumo matavimus tikslinga atlikinėti tik atsižvelgiant į klasifikavimo
tikslumą.

Dar viena problema dirbant su daugiamačiais duomenimis yra tai, kad dažnai
turimas labai ribotas skaičius duomenų objektų \textit{(angl. tuple)}. 
Objektų - dimensijų santykis (ODS) gali skirtis keliais šimtais kartų. 
Atrodytų, tik laiko klausimas, kada bus paruošta
daugiau duomenų objektų, bet žvelgiant į duomenų gavybos tendencijas pasidaro
aišku, kad
objektų skaičius niekada nepavys dimensijų skaičiaus. Todėl reikia apgalvoti,
kaip bus į tai atsižvelgta kuriant klasifikavimo modelius. Nes esant mažam
objektų skaičiui kyla grėsmė, kad klasifikatorius bus sukurtas toks, kuris
gerai veiks tik, su tais duomenimis, kuriais remiantis jis buvo sukurtas, 
tačiau netiksliai klasifikuos naujus duomenis. Tokia problema yra vadinama 
persimokymo problema (angl. overfitting).

% JG: Šitam paragrafe suplaki daug svarbių dalykų į krūvą ir juos turėtum būti paaiškinęs jau anksčiau. Mėginių trūkumas nėra tavo sprendžiama problema. Tačiau, kuo mažiau duomenų, tuo nestabilesni atrenkami matai. 

Išgaunamuose biomedicininiuose duomenyse didėja dimensijų skaičius, todėl 
daugėja ir nereikalingų ar klaidingų duomenų - triukšmo. Triukšmas atsiranda
dėl įvairių priežasčių, pvz, cheminiai preparatai buvo netinkamai paruošti. 
Duomenyse esant triukšmui su jais tampa sudėtinga dirbti, prastėja klasifikavimo
rezultatai. Atsiranda poreikis duomenų apdorojimo etape identifikuoti 
triukšmingus duomenis ir juos pašalinti iš duomenų rinkinio.

Taigi, dirbant su daugiamačiais duomenimis, reikia atsižvelgti į keletą 
kriterijų:
\begin{enumerate}
 \item Klasifikavimo tikslumą;
 \item Dimensijų atrinkimo stabilumą, atsižvelgiant į klasifikavimo rezultatus;
 \item Triukšmo lygį duomenyse;
 \item Skaičiavimo išteklių naudojimo racionalumą.
\end{enumerate}
Reikalavimas vienu metu atsižvelgti į keletą kriterijų apsunkina užduotį.
Klasifikuojant daugiamačius duomenis uždavinys yra surasti geriausius 
rezultatus duodančią strategiją, kuri geriausiai atsižvelgia į aukščiau minėtus
kriterijus.

Darbo eksperimentinei daliai reikalingus skaičiavimo išteklius, suteikė VU MIF 
ITC. Duomenų apdorojimo algoritmų implementavimui buvo naudojama R programavimo
kalba. Eksperimentai atlikti profesinės praktikos MII metu.

Šio darbo tikslas yra išanalizuoti darbo su daugiamačiais duomenis ypatybes.

Šiam darbui yra keliamos tokios užduotys:
\begin{enumerate}
 \item Apžvelgti esamus klasifikavimo metodus;
 \item Išanalizuoti bazinių dimensijų atrinkimo metodų spartą;
 \item Išanalizuoti bazinių dimensijų atrinkimo metodų stabilumą;
 \item Išanalizuoti kaip dimensijų atrinkimo metodai įtakoja klasifikatorių
 tikslumą;
 \item Pasiūlyti naują metodą daugiamačių duomenų klasifikavimui.
\end{enumerate}

Bakalaurinis darbas suskirstytas į 3 skyrius:
\begin{enumerate}
 \item Pirmajame skyriuje apžvelgsiu nagrinėjamą dalykinę sritį;
 \item Antrajame skyriuje pristatysiu siūlomą metodą daugiamačių duomenų 
 klasifikavimui;
 \item Trečiajame skyriuje aprašysiu darbo metu atliktus eksperimentus.
\end{enumerate}

