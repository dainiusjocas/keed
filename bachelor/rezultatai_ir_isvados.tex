%% Rezultatų ir išvadų dalyje išdėstomi pagrindiniai darbo rezultatai (kažkas išanalizuota, kažkas sukurta, kažkas diegta), pateikiamos išvados (daromi nagrinėtų problemų sprendimo metodų palyginimai, siūlomos rekomendacijos, akcentuojamos naujovės).

\section*{REZULTATAI IR TOLIMESNIŲ TYRIMŲ KRYPTYS}

Šiame darbe analizuota vis didesnį susidomėjimą kelianti daugiamačių duomenų klasifikavimo problematika ypatingą dėmesį kreipiant į matų atrinkimo problematiką. Atliekant analizę buvo susipažinta su daugiamačių duomenų klasifikavimo ir matų atrinkimo geriausiomis praktikomis ir atlikti matų atrinkimo metodų palyginimo eksperimentus.



Matų atrinkimo metodų palyginamųjų eksperimentų metu gauti rezultatai parodė, kad nėra vieno universaliai geriausio matų atrinkimo metodo. Pasirenkant matų atrinkimo metodą reikia atsižvelgti į turimus duomenis bei į darbui keliamus tikslus. 

Šiame darbe sukaupta patirtis gali būti panaudota kaip tolimesnių daugiamačių biomedicininių duomenų tyrimų pagrindas. Tolimesnių tyrimų kryptys galėtų būti stabilių matų, kurie maksimaliai padidina klasifikavimo tikslumą, atrinkimo metodo paiešką.
