%% Rezultatų ir išvadų dalyje išdėstomi pagrindiniai darbo rezultatai (kažkas išanalizuota, kažkas sukurta, kažkas diegta), pateikiamos išvados (daromi nagrinėtų problemų sprendimo metodų palyginimai, siūlomos rekomendacijos, akcentuojamos naujovės).

\section*{REZULTATAI IR TOLIMESNIŲ TYRIMŲ KRYPTYS}
\label{rezultatai_ir_tolimesniu_tyrimu_kryptys}

Šiame darbe analizuota vis didesnį susidomėjimą kelianti daugiamačių biomedicininių duomenų klasifikavimo problematika ypatingą dėmesį kreipiant informatyviausių matų atrinkimo aspektui. Atliekant analizę buvo susipažinta su daugiamačių duomenų klasifikavimo ir matų atrinkimo geriausiomis praktikomis ir eksperimentiškai palyginti matų atrinkimo metodai.

Matų atrinkimo metodų palyginamųjų eksperimentų metu gauti rezultatai parodė, kad nėra vieno universaliai geriausio matų atrinkimo metodo tinkančio genų išraiškos duomenų klasifikavimo analizei. Pasirenkant matų atrinkimo metodą visada reikia atsižvelgti į turimus duomenis bei į analizei keliamus tikslus, pvz., ligos diagnostikos įrankio sukūrimas. 

Matų atrinkimas daugiamačiams duomenims yra labai svarbus. Ieškant naujų kokybiškų matų atrinkimo metodų dvi pagrindinės kryptys yra naudoti multikriterinius matų atrinkimo ir panašių matų grupavimo metodus. Panašių matų grupavimui kol kas yra pasiūlytas tik CGS metodas, kuris netinka daugiamačiams duomenims, nes yra per lėtas. Multikriteriniai matų reitingavimo metodai kenčia situacijose, kai kurie nors iš kriterijų tam tikram duomenų rinkiniui demonstruoja prastus rezultatus, pavyzdžiui, duomenyse yra daug koreliuojančių matų, o matų reitingavimo metodai į tai neatsižvelgia. Kuo daugiau matų turi duomenys, tuo labiau pasireiškia minėtos problemos.

Šiame darbe sukaupta patirtis gali būti panaudota kaip tolimesnių daugiamačių biomedicininių duomenų tyrimų pagrindas. Toliau ieškant stabilių matų atrinkimo metodų, kurie maksimaliai padidina klasifikavimo tikslumą, reikėtų daugiau dėmesio kreipti į tinkmo matų grupavimo algoritmo paiešką. Tinkamas matų grupavimo algoritmas turėtų atsižvelgti į matų tarpusavio koreliacijas bei būti pakankamai spartus.
