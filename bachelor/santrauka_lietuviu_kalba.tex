\section*{SANTRAUKA}
\label{santrauka}

Šiame baigiamajame bakalauro studijų darbe yra analizuojamas daugiamačių, mažai mėginių turinčių, triukšmingų biomedicininių duomenų klasifikavimas. Tokie duomenys sutinkami genomo, genų išraiškos, epigenetikos tyrimuose. Darbe pateikiama mašininio mokymosi teorijos, reikalingos daugiamačių duomenų klasifikavimui, apžvalga. Daugiamačių duomenų analizė neapsieina be matų atrinkimo metodų taikymo, todėl darbe detaliai nagrinėjami pagrindiniai ir multikriteriniai matų atrinkimo metodai. Taikant šiuos metodus daugiamačių duomenų klasifikavimo analizei, svarbus skaičiavimo laikas, apmokyto klasifikatoriaus tikslumas ir atrenkamų matų stabilumas. Atlikus eksperimentus ir palyginus skirtingus metodus paaiškėjo, kad nėra universaliai geriausios klasifikavimo strategijos, tinkamos daugiamačiams biomedicininiams duomenims. Todėl šis darbas yra įžanga į tolimesnius tyrimus. Darbo išvadose pateikiamos tolesnių tyrimų gairės.

\textbf{Raktiniai žodžiai:} daugiamačiai duomenys, mašininis mokymasis, klasifikavimas, matų atrinkimas, matų stabilumas. 